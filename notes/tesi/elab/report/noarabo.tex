\spzlessottomano{^Ab.Ud.Int.I}{Abudenti}{Abudenti}
,\ np.\ a.
Abudenti.
\begin{subvocedue}
\item[Rif.:] \sezione{sec:abudenti}
\end{subvocedue}
\spzlessottomano{^Erken}{ėrken}{di gran mattino, di buon ora}
,\ avv.\ t.
di gran mattino, di buon ora.
\begin{subvocedue}
\item[Rif.:] \spzcite[25]{kiefferbianchi18351}
\end{subvocedue}
\spzlessottomano{:CsetAn.H}{asitane}{soglia,   porta, corte; Istanbul; nel testo compare una  volta   seguito  da  \RL{sa`Adat}   e  una  volta   seguito  da \RL{dawlat}}
,\ n.\ p.
soglia,   porta, corte; Istanbul; nel testo compare una  volta   seguito  da  \RL{sa`Adat}   e  una  volta   seguito  da \RL{dawlat}.
\begin{subvocedue}
\item[Pron. (1.0):] \spzcite{redhouse1997}
\item[Rif.:] \spzcite[31]{kiefferbianchi18351}
\end{subvocedue}
\spzlessottomano{:CsUd.H}{asude}{quieto, tranquillo}
,\ agg.\ p.
quieto, tranquillo.
\begin{subvocedue}
\item[Pron. (1.0):] \spzcite{redhouse1997}
\item[Rif.:] \spzcite[42]{kiefferbianchi18351}, \spzcite[61]{steingass1992}
\end{subvocedue}
\spzlessottomano{eker}{eğer}{se}
,\ cong.\ p.
se.
\begin{subvocedue}
\item[Pron. (1.0):] \spzcite[328]{redhouse1997}
\item[Rif.:] \spzcite[78]{kiefferbianchi18351}
\end{subvocedue}
\spzlessottomano{:GlB}{al-}{prendere; acquistare}
,\ v.\ t.
prendere; acquistare.
\begin{subvocedue}
\item[Pron. (1.0):] \tabella{tab:verbiturchi}
\item[Rif.:] \spzcite[92]{kiefferbianchi18351}, \tabella{tab:verbiturchi}
\end{subvocedue}
\spzlessottomano{alt.Un}{altun}{oro, moneta d'oro}
,\ n.\ t.
oro, moneta d'oro.
\begin{subvocedue}
\item[Pron. (1.0):] \sezione{sec:turconomi}
\item[Rif.:] \sezione{sec:turconomi}, \spzcite[87]{kiefferbianchi18351}
\end{subvocedue}
\spzlessottomano{umId}{ümid, ümit}{speranza}
,\ n.\ p.
speranza.
\begin{subvocedue}
\item[Pron. (1.0):] \spzcite[1208]{redhouse1997}
\item[Rif.:] \spzcite[101]{kiefferbianchi18351}
\end{subvocedue}
\spzlessottomano{:W^c}{üç}{tre}
,\ num.\ t.
tre.
\spzlessottomano{:Wzer}{üzer}{sopra, contro}
,\ post.\ t.
sopra, contro.
\begin{subvocedue}
\item[Pron. (1.0):] \spzcite[1212]{redhouse1997}
\item[Rif.:] \spzcite[129]{kiefferbianchi18351}
\end{subvocedue}
\spzlessottomano{:QzenB}{özen}{afferrare, rompere a forza di tirare, strappare}
,\ v.\ t.
afferrare, rompere a forza di tirare, strappare.
\begin{subvocedue}
\item[Rif.:] \spzcite[131]{kiefferbianchi18351}
\end{subvocedue}
\spzlessottomano{.Ql}{ol}{forma arcaica di o}
,\ pron.\ t.
forma arcaica di o.
\spzlessottomano{.QlB}{ol-}{essere, diventare}
,\ va.\ t.
essere, diventare.
\begin{subvocedue}
\item[Pron. (1.0):] \tabella{tab:verbiturchi}
\item[Rif.:] \tabella{tab:verbiturchi}
\end{subvocedue}
\spzlessottomano{.Qn}{on}{dieci}
,\ num.\ t.
dieci.
\spzlessottomano{.JB}{i-}{essere (difettivo)}
,\ va.\ t.
essere (difettivo).
\spzlessottomano{:EtB}{ėt-}{fare}
,\ va.\ t.
fare.
\begin{subvocedue}
\item[Pron. (1.0):] \tabella{tab:verbiturchi}
\item[Rif.:] \sezione{sec:turcoverbiausiliari}
\end{subvocedue}
\spzlessottomano{.J^c:Un}{içün}{per}
,\ post.\ t.
per.
\spzlessottomano{Idir}{eider, ider}{adesso, qui}
,\ avv.\ p.
adesso, qui.
\begin{subvocedue}
\item[Rif.:] \spzcite[154]{kiefferbianchi18351}, \spzcite[129]{steingass1992}
\end{subvocedue}
\spzlessottomano{^EyleB}{eyle-}{fare}
,\ va.\ t.
fare.
\begin{subvocedue}
\item[Pron. (1.0):] \tabella{tab:verbiturchi}
\item[Rif.:] \sezione{sec:turcoverbiausiliari}, \spzcite[160]{kiefferbianchi18351}
\end{subvocedue}
\spzlessottomano{.Jl.H}{ile}{con}
,\ post.\ t.
con.
\spzlessottomano{^Ey.U}{eyu}{buono, bello, bene}
,\ agg.\ t.
buono, bello, bene.
\begin{subvocedue}
\item[Pron. (1.0):] \sezione{sec:aggettiviturchi}
\item[Rif.:] \sezione{sec:aggettiviturchi}, \spzcite[163]{kiefferbianchi18351}
\end{subvocedue}
\spzlessottomano{bAd}{bad}{imperativo: sia, fiat}
,\ v.\ p.
imperativo: sia, fiat.
\begin{subvocedue}
\item[Rif.:] \spzcite[167]{kiefferbianchi18351}
\end{subvocedue}
\spzlessottomano{b^Ar^c:H}{barça}{nave da guerra, grande galera}
,\ n.\ m.
nave da guerra, grande galera.
\begin{subvocedue}
\item[Rif.:] \sezione{sec:barca}
\end{subvocedue}
\spzlessottomano{b^Ayl.Os}{baylos}{Bailo}
,\ n.\ l.
Bailo.
\begin{subvocedue}
\item[Rif.:] \spzcite[186]{kiefferbianchi18351}
\end{subvocedue}
\spzlessottomano{bir}{bir}{uno}
,\ num.\ t.
uno.
\spzlessottomano{birl.H}{birle}{con}
,\ post.\ t.
con.
\begin{subvocedue}
\item[Rif.:] \spzcite[205]{kiefferbianchi18351}
\end{subvocedue}
\spzlessottomano{biz}{biz}{noi}
,\ pron.\ t.
noi.
\spzlessottomano{be^s}{beş}{cinque}
,\ num.\ t.
cinque.
\spzlessottomano{beg}{beg, bey}{principe, signore, governatore di distretto; filgli di paşa; capitani; stranieri di considerazione; qualcosa meno di paşa, una sola coda di cavallo}
,\ n.\ c.
principe, signore, governatore di distretto; filgli di paşa; capitani; stranieri di considerazione; qualcosa meno di paşa, una sola coda di cavallo.
\begin{subvocedue}
\item[Rif.:] \spzcite[220]{kiefferbianchi18351}
\end{subvocedue}
\spzlessottomano{bilB}{bil-}{conoscere, comprendere, sapere; guardare, stimare}
,\ v.\ t.
conoscere, comprendere, sapere; guardare, stimare.
\begin{subvocedue}
\item[Pron. (1.0):] \tabella{tab:verbiturchi}
\item[Rif.:] \spzcite[226-227]{kiefferbianchi18351}, \tabella{tab:verbiturchi}
\end{subvocedue}
\spzlessottomano{b.U}{bu}{questo}
,\ aggp.\ t.
questo.
\spzlessottomano{b.UlB}{bul-}{trovare, scoprire, inventare}
,\ v.\ t.
trovare, scoprire, inventare.
\begin{subvocedue}
\item[Pron. (1.0):] \tabella{tab:verbiturchi}
\item[Rif.:] \sezione{sec:turcoverbi}, \spzcite[247]{kiefferbianchi18351}
\end{subvocedue}
\spzlessottomano{b.I^n}{bin}{mille}
,\ num.\ t.
mille.
\spzlessottomano{buy.UrB}{buyur-}{ordinare, ordinare  di fare,  decretare, dare mandato per iscritto; ordinare in senso cortese; al posto di ėt- quando ci si rivolge ai superiori o si chiede un favore; pregare di}
,\ va.\ t.
ordinare, ordinare  di fare,  decretare, dare mandato per iscritto; ordinare in senso cortese; al posto di ėt- quando ci si rivolge ai superiori o si chiede un favore; pregare di.
\begin{subvocedue}
\item[Pron. (1.0):] \tabella{tab:verbiturchi}
\item[Rif.:] \sezione{sec:turcoverbiausiliari}, \spzcite[266]{kiefferbianchi18351}
\end{subvocedue}
\spzlessottomano{pAde^sAh}{Padişah}{Padishah, Sultano dell'impero Ottomano}
,\ n.\ p.
Padishah, Sultano dell'impero Ottomano.
\spzlessottomano{por}{pür}{molto; pieno, abbondante, numeroso}
,\ avv./agg.\ p.
molto; pieno, abbondante, numeroso.
\begin{subvocedue}
\item[Rif.:] \spzcite[197]{kiefferbianchi18351}, \spzcite[239]{steingass1992}
\end{subvocedue}
\spzlessottomano{penb.H}{penbe}{cotone}
,\ n.\ m.
cotone.
\begin{subvocedue}
\item[Rif.:] \spzcite[230]{kiefferbianchi18351}
\end{subvocedue}
\spzlessottomano{tAbAn}{taban}{luminoso, risplendente; meglio (sostantivo)}
,\ agg.\ p.
luminoso, risplendente; meglio (sostantivo).
\begin{subvocedue}
\item[Rif.:] \spzcite[269]{kiefferbianchi18351}, \spzcite[271-272]{steingass1992}
\end{subvocedue}
\spzlessottomano{^gAygIr}{caygir}{che occorre, stabilito, posto, che occupa, tiene, penetra}
,\ agg.\ p.
che occorre, stabilito, posto, che occupa, tiene, penetra.
\begin{subvocedue}
\item[Rif.:] \sezione{sec:aggettivipersiani}, \spzcite[363]{kiefferbianchi18351}
\end{subvocedue}
\spzlessottomano{^ga`afir}{Cafer}{Cafer}
,\ np.\ a.
Cafer.
\begin{subvocedue}
\item[Rif.:] \sezione{sec:cafer}
\end{subvocedue}
\spzlessottomano{^gUy}{cuy}{che vuole, che desidera}
,\ agg.\ p.
che vuole, che desidera.
\begin{subvocedue}
\item[Rif.:] \sezione{sec:aggettivipersiani}, \spzcite[405]{kiefferbianchi18351}
\end{subvocedue}
\spzlessottomano{^c^A:vu^s}{çavuş}{çavuş}
,\ n.\ t.
çavuş.
\begin{subvocedue}
\item[Rif.:] \sezione{sec:titoli}, \spzcite[361]{kiefferbianchi18351}
\end{subvocedue}
\spzlessottomano{^cinq.U||s^A:v.I}{Cinque Savii}{Cinque Savii}
,\ np.\ i.
Cinque Savii.
\begin{subvocedue}
\item[Rif.:] \sezione{sec:cinquesavi}
\end{subvocedue}
\spzlessottomano{^co:vAl}{çuval}{sacco}
,\ n.\ p.
sacco.
\begin{subvocedue}
\item[Rif.:] \spzcite[398]{kiefferbianchi18351}, \spzcite[401]{steingass1992}
\end{subvocedue}
\spzlessottomano{_h_wAh}{hah}{colui che desidera; desiderio, volontà}
,\ agg.\ p.
colui che desidera; desiderio, volontà.
\begin{subvocedue}
\item[Rif.:] \spzcite[490]{kiefferbianchi18351}, \spzcite[481]{steingass1992}
\end{subvocedue}
\spzlessottomano{_h.Od}{hod}{(congiunzione) ma, ora, quanto a}
,\ post.\ t.
(congiunzione) ma, ora, quanto a.
\begin{subvocedue}
\item[Rif.:] \spzcite[491]{kiefferbianchi18351}
\end{subvocedue}
\spzlessottomano{d.eB}{dė-}{dire, chiamare}
,\ v.\ t.
dire, chiamare.
\begin{subvocedue}
\item[Pron. (1.0):] \tabella{tab:verbiturchi}
\item[Rif.:] \sezione{sec:turcoverbi}, \spzcite[571]{kiefferbianchi18351}
\end{subvocedue}
\spzlessottomano{da_h.Y}{dahi}{anche, ancora, e}
,\ cong.\ t.
anche, ancora, e.
\begin{subvocedue}
\item[Pron. (1.0):] \sezione{sec:congturche}
\item[Rif.:] \sezione{sec:congturche}, \spzcite[512]{kiefferbianchi18351}
\end{subvocedue}
\spzlessottomano{dur}{dur}{copula (è)}
,\ va.\ t.
copula (è).
\spzlessottomano{doro_h^sAn}{dirahşan}{chiaro, brillante, lampeggiante}
,\ agg.\ p.
chiaro, brillante, lampeggiante.
\begin{subvocedue}
\item[Rif.:] \spzcite[515]{kiefferbianchi18351}, \spzcite[510]{steingass1992}
\end{subvocedue}
\spzlessottomano{darUnI}{deruni}{interno, spirituale}
,\ agg.\ p.
interno, spirituale.
\begin{subvocedue}
\item[Rif.:] \spzcite[520]{kiefferbianchi18351}, \spzcite[515]{steingass1992}
\end{subvocedue}
\spzlessottomano{derI.g}{diriγ}{rifiuto, repulsione; mancanza, omissione; dispiacere per una mancanza}
,\ n.\ p.
rifiuto, repulsione; mancanza, omissione; dispiacere per una mancanza.
\begin{subvocedue}
\item[Rif.:] \spzcite[522]{kiefferbianchi18351}, \spzcite[401]{steingass1992}
\end{subvocedue}
\spzlessottomano{daftar}{defter}{registro,  inventario,  quaderno,  anche nel senso di pubblico registro}
,\ n.\ p.
registro,  inventario,  quaderno,  anche nel senso di pubblico registro.
\begin{subvocedue}
\item[Rif.:] \spzcite[529-530]{kiefferbianchi18351}
\end{subvocedue}
\spzlessottomano{dak.In}{degin, değin}{con dativo o con forme in \spzrl{Bin^g.H}: fino a, finché}
,\ post.\ t.
con dativo o con forme in \spzrl{Bin^g.H}: fino a, finché \verificare.
\begin{subvocedue}
\item[Rif.:] \spzcite[536]{kiefferbianchi18351}
\end{subvocedue}
\spzlessottomano{de^nl:U}{deŋlü}{particella che marca la quantità di una cosa}
,\ num.\ t.
particella che marca la quantità di una cosa.
\begin{subvocedue}
\item[Pron. (1.0):] \sezione{sec:quantificatori}
\item[Rif.:] \sezione{sec:quantificatori}, \spzcite[534]{kiefferbianchi18351}
\end{subvocedue}
\spzlessottomano{del.H}{dile}{cuore, anima}
,\ n.\ p.
cuore, anima.
\begin{subvocedue}
\item[Rif.:] \spzcite[541]{kiefferbianchi18351}, \spzcite[533]{steingass1992}
\end{subvocedue}
\spzlessottomano{d.O^z}{doj}{doge}
,\ n.\ l.
doge.
\spzlessottomano{dUst}{dost}{amico}
,\ n.\ p.
amico.
\begin{subvocedue}
\item[Rif.:] \spzcite[553]{kiefferbianchi18351}, \spzcite[544]{steingass1992}
\end{subvocedue}
\spzlessottomano{d.U^sB}{düş-}{arrivare, avere luogo, incontrarsi, convenire, concernere}
,\ v.\ t.
arrivare, avere luogo, incontrarsi, convenire, concernere.
\begin{subvocedue}
\item[Pron. (1.0):] \tabella{tab:verbiturchi}
\item[Rif.:] \sezione{sec:turcoverbi}, \spzcite[555]{kiefferbianchi18351}
\end{subvocedue}
\spzlessottomano{rUz}{ruz}{giorno, dì}
,\ n.\ p.
giorno, dì.
\begin{subvocedue}
\item[Rif.:] \sezione{sec:nomipersiani}, \spzcite[604]{kiefferbianchi18351}
\end{subvocedue}
\spzlessottomano{rU^san}{ruşen}{chiaro, manifesto, luminoso, splendente}
,\ agg.\ p.
chiaro, manifesto, luminoso, splendente.
\begin{subvocedue}
\item[Rif.:] \spzcite[606]{kiefferbianchi18351}, \spzcite[595]{steingass1992}
\end{subvocedue}
\spzlessottomano{sAy.H}{saye}{ombra, protezione, favore}
,\ n.\ p.
ombra, protezione, favore.
\begin{subvocedue}
\item[Rif.:] \spzcite[644]{kiefferbianchi18351}, \spzcite[645]{steingass1992}
\end{subvocedue}
\spzlessottomano{sepAre^s}{sipariş}{ordine, commissione, raccomandazione}
,\ n.\ p.
ordine, commissione, raccomandazione.
\begin{subvocedue}
\item[Rif.:] \spzcite[650]{steingass1992}
\end{subvocedue}
\spzlessottomano{sotUd.H}{sütude}{lodabile,lodato}
,\ agg.\ p.
lodabile,lodato.
\begin{subvocedue}
\item[Rif.:] \spzcite[651]{kiefferbianchi18351}, \spzcite[656]{steingass1992}
\end{subvocedue}
\spzlessottomano{siz}{siz}{senza (+ablativo o gerundio in -E)}
,\ post.\ t.
senza (+ablativo o gerundio in -E).
\begin{subvocedue}
\item[Rif.:] \spzcite[670]{kiefferbianchi18351}
\end{subvocedue}
\spzlessottomano{seks^En}{seksen}{ottanta}
,\ num.\ t.
ottanta.
\spzlessottomano{.so^nr:H}{soŋra}{dopo + ablativo}
,\ post.\ t.
dopo + ablativo.
\begin{subvocedue}
\item[Rif.:] \spzcite[115]{kiefferbianchi18352}
\end{subvocedue}
\spzlessottomano{.s:in.g.Ur}{sıngur}{assicurazione}
,\ n.\ m.
assicurazione.
\begin{subvocedue}
\item[Rif.:] \sezione{sec:sengur}
\end{subvocedue}
\spzlessottomano{.toq.Uz}{dokuz}{nove}
,\ num.\ t.
nove.
\spzlessottomano{`AlimpenAh}{alempenah}{Rifugio del Mondo (titolo del Sultano)}
,\ sc.\ ap.
Rifugio del Mondo (titolo del Sultano).
\begin{subvocedue}
\item[Rif.:] \spzcite[223]{kiefferbianchi18352}, \spzcite[229]{kiefferbianchi18352}
\end{subvocedue}
\spzlessottomano{farAmU^s}{feramûş}{dimenticanza}
,\ n.\ p.
dimenticanza.
\begin{subvocedue}
\item[Rif.:] \spzcite[361]{kiefferbianchi18352}, \spzcite[914]{steingass1992}
\end{subvocedue}
\spzlessottomano{q^Abil.O}{Capello}{Capello}
,\ np.\ i.
Capello.
\begin{subvocedue}
\item[Rif.:] \sezione{sec:girolamocapello}
\end{subvocedue}
\spzlessottomano{q^AlB}{kal-}{restare, restare indietro}
,\ v.\ t.
restare, restare indietro.
\begin{subvocedue}
\item[Pron. (1.0):] \tabella{tab:verbiturchi}
\item[Rif.:] \sezione{sec:turcoverbi}, \spzcite[424-425]{kiefferbianchi18352}
\end{subvocedue}
\spzlessottomano{qubris}{Kıbrıs}{Cipro}
,\ np.
Cipro.
\begin{subvocedue}
\item[Rif.:] \sezione{sec:qubris}
\end{subvocedue}
\spzlessottomano{q:ilB}{kıl-}{fare, operare}
,\ va.\ t.
fare, operare.
\begin{subvocedue}
\item[Pron. (1.0):] \tabella{tab:verbiturchi}
\item[Rif.:] \sezione{sec:turcoverbiausiliari}, \spzcite[502]{kiefferbianchi18352}
\end{subvocedue}
\spzlessottomano{qan.tAr}{kantar}{cantaro, unità di peso}
,\ n.\ m.
cantaro, unità di peso.
\begin{subvocedue}
\item[Rif.:] \sezione{sec:qantar}
\end{subvocedue}
\spzlessottomano{q.Ut}{kut}{carisma}
,\ n.\ t.
carisma.
\begin{subvocedue}
\item[Pron. (1.0):] \tabella{tab:nomiturchi}
\item[Rif.:] \sezione{sec:qut}
\end{subvocedue}
\spzlessottomano{kimesn.H}{kimesne}{come kimse: qualcuno, alcuno}
,\ pron.\ t?.
come kimse: qualcuno, alcuno.
\spzlessottomano{kend:U}{kendü}{stesso (kendi)}
,\ pron.\ t?.
stesso (kendi).
\spzlessottomano{k^H}{ki}{che}
,\ cong.\ p.
che.
\spzlessottomano{g:alB}{ǵel-}{arrivare, venire}
,\ v.\ t.
arrivare, venire.
\begin{subvocedue}
\item[Pron. (1.0):] \tabella{tab:verbiturchi}
\item[Rif.:] \spzcite[629]{kiefferbianchi18352}, \tabella{tab:verbiturchi}
\end{subvocedue}
\spzlessottomano{gomAn}{güman}{dubbio, incertezza}
,\ n.\ p.
dubbio, incertezza.
\begin{subvocedue}
\item[Rif.:] \spzcite[636]{kiefferbianchi18352}, \spzcite[1097]{steingass1992}
\end{subvocedue}
\spzlessottomano{gen.H}{gene}{di nuovo, nientemeno}
,\ avv.\ t.
di nuovo, nientemeno.
\begin{subvocedue}
\item[Pron. (1.0):] \sezione{sec:turcoavv}
\item[Rif.:] \sezione{sec:turcoavv}, \spzcite[649]{kiefferbianchi18352}
\end{subvocedue}
\spzlessottomano{g:Oz}{göz}{occhio}
,\ n.\ t.
occhio.
\begin{subvocedue}
\item[Pron. (1.0):] \tabella{tab:nomiturchi}
\item[Rif.:] \sezione{sec:turconomi}, \spzcite[660]{kiefferbianchi18352}
\end{subvocedue}
\spzlessottomano{g:OnderB}{ǵönder-}{inviare}
,\ v.\ t.
inviare.
\begin{subvocedue}
\item[Pron. (1.0):] \sezione{sec:turcoverbi}
\item[Rif.:] \sezione{sec:turcoverbi}, \spzcite[672]{kiefferbianchi18352}
\end{subvocedue}
\spzlessottomano{g.IdB}{ǵit-}{andare, partire}
,\ va.\ t.
andare, partire.
\begin{subvocedue}
\item[Pron. (1.0):] \sezione{sec:turcoverbiausiliari}
\item[Rif.:] \sezione{sec:turcoverbiausiliari}, \spzcite[680]{kiefferbianchi18352}
\end{subvocedue}
\spzlessottomano{lodr:H}{lodra}{lodra (unità di misura)}
,\ n.
lodra (unità di misura).
\spzlessottomano{m^Arq.O||d:H^Ald.I}{Marco d'Aldi}{Marco d'Aldi}
,\ np.\ i.
Marco d'Aldi.
\begin{subvocedue}
\item[Rif.:] \sezione{sec:marcodaldi}
\end{subvocedue}
\spzlessottomano{mord}{murd}{morto}
,\ n.\ p.
morto.
\begin{subvocedue}
\item[Rif.:] \spzcite[864]{kiefferbianchi18352}, \spzcite[1212]{steingass1992}
\end{subvocedue}
\spzlessottomano{m.Us^A||ma^g^A|.Qd}{Mosè Magiaod}{Mosè Magiaod}
,\ np.
Mosè Magiaod.
\begin{subvocedue}
\item[Rif.:] \sezione{sec:mosemagiaod}
\end{subvocedue}
\spzlessottomano{nAm}{nam}{nome, fama, onore}
,\ n.\ p.
nome, fama, onore.
\begin{subvocedue}
\item[Rif.:] \spzcite[1084]{kiefferbianchi18352}, \spzcite[1378]{steingass1992}
\end{subvocedue}
\spzlessottomano{ni^c.H}{nice}{come? in che modo?}
,\ pron.\ t.
come? in che modo?.
\begin{subvocedue}
\item[Rif.:] \sezione{sec:pronomiturchi}, \spzcite[1096-1097]{kiefferbianchi18352}
\end{subvocedue}
\spzlessottomano{nehAn}{nihan}{segreto, interno; nascosto, occulto}
,\ agg.\ p.
segreto, interno; nascosto, occulto.
\begin{subvocedue}
\item[Rif.:] \spzcite[1150]{kiefferbianchi18352}, \spzcite[1438]{steingass1992}
\end{subvocedue}
\spzlessottomano{nIk}{nik}{buono, eccellente, fortunato; bene, fortuna; come avverbio, molto}
,\ agg./avv.\ p.
buono, eccellente, fortunato; bene, fortuna; come avverbio, molto.
\begin{subvocedue}
\item[Rif.:] \spzcite[1155]{kiefferbianchi18352}, \spzcite[1443]{steingass1992}
\end{subvocedue}
\spzlessottomano{:vened.Ik}{Venedik}{Venezia}
,\ np.
Venezia.
\begin{subvocedue}
\item[Rif.:] \sezione{sec:venedik}
\end{subvocedue}
\spzlessottomano{:v.ErB}{vėr-}{dare (anche ausiliare)}
,\ v.\ t.
dare (anche ausiliare).
\begin{subvocedue}
\item[Pron. (1.0):] \tabella{tab:verbiturchi}
\item[Rif.:] \spzcite[1201]{kiefferbianchi18352}, \tabella{tab:verbiturchi}
\end{subvocedue}
\spzlessottomano{her}{her}{ogni, ognuno}
,\ aggp.\ p.
ogni, ognuno.
\begin{subvocedue}
\item[Rif.:] \spzcite[1212]{kiefferbianchi18352}
\end{subvocedue}
\spzlessottomano{hamAn}{hemen}{solamente, non più; anche, lo stesso; così, esattamente così; sempre; adesso}
,\ avv.\ p.
solamente, non più; anche, lo stesso; così, esattamente così; sempre; adesso.
\begin{subvocedue}
\item[Rif.:] \spzcite[1224]{kiefferbianchi18352}, \spzcite[1507]{steingass1992}
\end{subvocedue}
\spzlessottomano{y^Aqom.O||by^Az.I}{Giacomo Biasii}{Giacomo Biasii}
,\ np.\ i.
Giacomo Biasii.
\begin{subvocedue}
\item[Rif.:] \sezione{sec:giacomobiasii}
\end{subvocedue}
\spzlessottomano{y^An}{yan}{lato, fianco, profilo}
,\ n.\ t.
lato, fianco, profilo.
\begin{subvocedue}
\item[Pron. (1.0):] \tabella{tab:nomiturchi}
\item[Rif.:] \spzcite[1254-1255]{kiefferbianchi18352}, \tabella{tab:nomiturchi}
\end{subvocedue}
\spzlessottomano{yed.I}{yedi}{sette}
,\ num.\ t.
sette.
\spzlessottomano{yer}{yėr}{terra, posto}
,\ n.\ t.
terra, posto.
\begin{subvocedue}
\item[Pron. (1.0):] \sezione{sec:turconomi}
\item[Rif.:] \sezione{sec:turconomi}, \spzcite[1262-1263]{kiefferbianchi18352}
\end{subvocedue}
\spzlessottomano{yar^AB}{yara-}{essere servizievole, utile, adatto; valere, avere valore}
,\ v.\ t.
essere servizievole, utile, adatto; valere, avere valore.
\begin{subvocedue}
\item[Pron. (1.0):] \tabella{tab:verbiturchi}
\item[Rif.:] \spzcite[1264]{kiefferbianchi18352}, \tabella{tab:verbiturchi}
\end{subvocedue}
\spzlessottomano{yan^A^sB}{yanaş-}{avvicinare}
,\ v.\ t.
avvicinare.
\begin{subvocedue}
\item[Pron. (1.0):] \tabella{tab:verbiturchi}
\item[Rif.:] \spzcite[1284]{kiefferbianchi18352}, \spzcite[5611]{meninski1680dc}, \tabella{tab:verbiturchi}
\end{subvocedue}
\spzlessottomano{yin.H}{yine}{ancora, di nuovo, ma ancora, inoltre; nonostante}
,\ avv.\ t.
ancora, di nuovo, ma ancora, inoltre; nonostante.
\begin{subvocedue}
\item[Rif.:] \spzcite[1285]{kiefferbianchi18352}
\end{subvocedue}
\spzlessottomano{y:Uz}{yüz}{cento}
,\ num.\ t.
cento.
\spzlessottomano{y.Ol}{yol}{via, cammino, strada; canale; mezzo, modo}
,\ n.\ t.
via, cammino, strada; canale; mezzo, modo.
\begin{subvocedue}
\item[Pron. (1.0):] \tabella{tab:nomiturchi}
\item[Rif.:] \spzcite[1293-1294]{kiefferbianchi18352}, \tabella{tab:nomiturchi}
\end{subvocedue}
\spzlessottomano{yahUdI}{yahudi}{ebreo}
,\ n.
ebreo.
\begin{subvocedue}
\item[Rif.:] \spzcite[1298]{kiefferbianchi18352}
\end{subvocedue}
