\spzlessottomano{^El}{el}{mano}

\spzlessottomano{altUn}{altın}{oro, moneta d'oro}

\spzlessottomano{beg}{beg, bey}{principe, signore, governatore di distretto; filgli di paşa; capitani; stranieri di considerazione; qualcosa meno di paşa, una sola coda di cavallo}

\spzlessottomano{^c^A:vu^s}{çavuş}{çavuş}

\spzlessottomano{q.Ut}{kut}{carisma}

\spzlessottomano{g:Oz}{göz}{occhio}

\spzlessottomano{y^An}{yan}{lato, fianco, profilo}

\spzlessottomano{yer}{yer}{terra, posto}

\spzlessottomano{y.Ol}{yol}{via, cammino, strada; canale; mezzo, modo}
