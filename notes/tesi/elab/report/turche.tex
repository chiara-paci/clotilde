\begin{glossario}{Origine: turco}
\item[{\color{colorsologlossario}{\bf (g)}}] {\color{colorsologlossario}\spzrl{AtB}}, {\sf at-},\ v.\ t.:\ lanciare, gettare.
\item[{\color{colorlowref}\spzrl{^Erken}},] {\sf ėrken},\ avv.\ t.:\ di gran mattino, di buon ora.
\begin{subvocedue}
\item[Rif.:] \spzcite[25]{kiefferbianchi18351}
\end{subvocedue}
\begin{subvocedue}
\item[(var)] \spzrl{^Er}, {\sf ėr}\begin{subvocedue}
\item[Rif.:] \spzcite[19]{kiefferbianchi18351}
\end{subvocedue}
\item[(var)] \spzrl{:Er}, {\sf ėr}\begin{subvocedue}
\item[Rif.:] \spzcite[155]{kiefferbianchi18351}
\end{subvocedue}
\item[\subglossariobullet] \spzrl{:Er gI^c}, {\sf ėr geç}:\ presto o tardi \verificare[trascrizione].
\begin{subvocedue}
\item[Rif.:] \spzcite[155]{kiefferbianchi18351}
\end{subvocedue}
\item[(simil:1.0)] \spzrl{:Er}  22 (0)
\end{subvocedue}
\item[{\color{colorlowref}\spzrl{^El}},] {\sf el},\ n.\ t.:\ mano.
\begin{subvocedue}
\item[Pron. (1.0):] \spzcite[332-335]{redhouse1997}
\item[Rif.:] \spzcite[82-83]{kiefferbianchi18351}
\end{subvocedue}
\begin{subvocedue}
\item[\subglossariobullet] \spzrl{^Eld.H :EtB}, {\sf elde ėt-},\ v.\ t.:\ ottenere, acquistare il possesso.
\begin{subvocedue}
\item[Pron. (1.0):] \spzcite[333]{redhouse1997}
\item[Rif.:] \spzcite[333]{redhouse1997}
\end{subvocedue}
\item[\subglossariobullet] \spzrl{^Eld.H qAlB}, {\sf elde kal-},\ v.\ t.:\ essere lasciato (soldi o beni a qualcuno).
\begin{subvocedue}
\item[Pron. (1.0):] \spzcite[334]{redhouse1997}
\item[Rif.:] \spzcite[334]{redhouse1997}
\end{subvocedue}
\item[\subglossariobullet] \spzrl{^Elind.H dur}, {\sf elinde dur}:\ è in suo potere, può farlo.
\begin{subvocedue}
\item[Rif.:] \spzcite[83]{kiefferbianchi18351}
\end{subvocedue}
\item[(radice)] \spzrl{^Elllerind.H}  10 (7)
\end{subvocedue}


Contesti:
\begin{subvocedue}
\item[(riga 10)] \spzrl{^Elllerind.H q^Alinm^A.gl:H}
\end{subvocedue}
\item[{\color{colorlowref}\spzrl{:GlB}},] {\sf al-},\ v.\ t.:\ prendere; acquistare.
\begin{subvocedue}
\item[Pron. (1.0):] \spzcite[41]{redhouse1997}
\item[Rif.:] \spzcite[92]{kiefferbianchi18351}
\end{subvocedue}
\begin{subvocedue}
\item[(radice)] \spzrl{^Al.Ub}  7 (5)
\end{subvocedue}


Contesti:
\begin{subvocedue}
\item[(riga 7)] \spzrl{^Al.Ub g.Id.Ub}
\end{subvocedue}
\item[{\color{colorlowref}\spzrl{altUn}},] {\sf altın},\ n.\ t.:\ oro, moneta d'oro.
\begin{subvocedue}
\item[Pron. (1.0):] \spzcite[53]{redhouse1997}
\item[Rif.:] \spzcite[87]{kiefferbianchi18351}
\end{subvocedue}
\begin{subvocedue}
\item[(var)] \spzrl{altUn}, {\sf altun}\begin{subvocedue}
\item[Pron. (1.0):] \spzcite[87]{kiefferbianchi18351}
\end{subvocedue}
\item[(radice)] \spzrl{altUnliq}  7 (16)
\end{subvocedue}
\item[{\color{colorlowref}\spzrl{:W^c}},] {\sf üç},\ num.\ t.:\ tre.
\begin{subvocedue}
\item[(simil:1)] \spzrl{:W^c}  7 (11)
\end{subvocedue}
\item[{\color{colorlowref}\spzrl{:WzB}},] {\sf üz-},\ v.\ t.:\ afferrare, rompere a forza di tirare, strappare.
\begin{subvocedue}
\item[Pron. (1.0):] \spzcite[1212]{redhouse1997}
\item[Rif.:] \spzcite[131]{kiefferbianchi18351}
\end{subvocedue}
\begin{subvocedue}
\item[\subglossariobullet] \spzrl{:WzinB}, {\sf üzün-},\ v.\ t.:\ desiderare con passione, farsi prendere (?), essere strappato (?).
\begin{subvocedue}
\item[Rif.:] \spzcite[131]{kiefferbianchi18351}
\end{subvocedue}
\item[(radice)] \spzrl{:Wzin.H}  22 (16)
\end{subvocedue}
\item[{\color{colorlowref}\spzrl{:Wzer}},] {\sf üzer},\ post.\ t.:\ sopra, contro.
\begin{subvocedue}
\item[Pron. (1.0):] \spzcite[1212]{redhouse1997}
\item[Rif.:] \spzcite[129]{kiefferbianchi18351}
\end{subvocedue}
\begin{subvocedue}
\item[\subglossariobullet] \spzrl{:Wzer.H}, {\sf üzere},\ post.\ t.:\ sopra, secondo, contro, a condizione che (con forme nominali dei verbi).
\begin{subvocedue}
\item[Pron. (1.0):] \spzcite[1212]{redhouse1997}
\item[Rif.:] \spzcite[129]{kiefferbianchi18351}
\end{subvocedue}
\item[(radice)] \spzrl{:Wzer.H}  3 (17) 4 (12) 14 (3)
\item[(radice)] \spzrl{:Wzer.Yn.H}  20 (7)
\end{subvocedue}


Contesti:
\begin{subvocedue}
\item[(riga 3)] \spzrl{.tawiyyat :Wzer.H}
\item[(riga 4)] \spzrl{^gUyliq :Wzer.H}
\item[(riga 14)] \spzrl{.sadAqat :Wzer.H}
\item[(riga 20)] \spzrl{mazbUri^n :Wzer.Yn.H}
\end{subvocedue}
\item[{\color{colorlowref}\spzrl{.Ql}},] {\sf ol},\ pron.\ t.:\ forma arcaica di o.
\begin{subvocedue}
\item[(var)] \spzrl{andan}, {\sf andan}, ablativo\item[(simil:1.0)] \spzrl{andan}  11 (6)
\end{subvocedue}
\item[{\color{colorlowref}\spzrl{.QlB}},] {\sf ol-},\ va.\ t.:\ essere, diventare.
\begin{subvocedue}
\item[(simil:1.0)] \spzrl{.Ql}  11 (16) 15 (5)
\item[(radice)] \spzrl{.Ql^An}  2 (15) 3 (18) 8 (20) 10 (17) 12 (4) 13 (5) 17 (4) 19 (14) 20 (16)
\item[(radice)] \spzrl{.QldIlar}  17 (0)
\item[(radice)] \spzrl{.QlmAmi^s}  7 (23)
\item[(radice)] \spzrl{.QlmasI^cUn}  14 (14)
\item[(radice)] \spzrl{.Qlma.gil.h}  8 (7)
\item[(radice)] \spzrl{.Qlma.gIn}  7 (3) 12 (1)
\item[(radice)] \spzrl{.Qlmanmaq}  23 (15)
\item[(radice)] \spzrl{.Qlun^An}  2 (21) 9 (5) 18 (6) 19 (16) 23 (18)
\item[(radice)] \spzrl{.Qlund.I}  25 (6)
\item[(radice)] \spzrl{.QlunmAdU.g.I}  5 (22)
\item[(radice)] \spzrl{.Qlunmaq}  16 (4)
\item[(radice)] \spzrl{.QlunmayUb}  25 (1)
\item[(radice)] \spzrl{.QlunUrsah}  18 (16)
\item[(radice)] \spzrl{.QlUr}  21 (12)
\item[(radice)] \spzrl{.Ql:H}  20 (20) 24 (13)
\item[(radice)] \spzrl{.Ql:Hlar}  4 (13)
\item[(radice)] \spzrl{.Ql.Id.I}  8 (0)
\end{subvocedue}
\item[{\color{colorlowref}\spzrl{.Qn}},] {\sf on},\ num.\ t.:\ dieci.
\begin{subvocedue}
\item[(simil:1)] \spzrl{.Qn}  9 (19)
\end{subvocedue}
\item[{\color{colorlowref}\spzrl{.JB}},] {\sf i-},\ va.\ t.:\ essere (difettivo).
\begin{subvocedue}
\item[(radice)] \spzrl{.Jdi:k}  13 (17)
\item[(radice)] \spzrl{.JdindUkmiz}  11 (11)
\item[(radice)] \spzrl{.Jdinmi^s}  13 (16)
\item[(radice)] \spzrl{.JdUb}  11 (3) 18 (2) 19 (5)
\item[(radice)] \spzrl{.Jd.H}  4 (19)
\item[(radice)] \spzrl{.Jd.Ik^H}  3 (14) 8 (12) 9 (12)
\item[(radice)] \spzrl{.Jken}  6 (24) 15 (13)
\end{subvocedue}
\item[{\color{colorlowref}\spzrl{:EtB}},] {\sf ėt-},\ va.\ t.:\ fare.
\begin{subvocedue}
\item[(var)] \spzrl{_EtB}, {\sf ėt-}\item[(var)] \spzrl{:Ed}, {\sf ėd-}, desinenze in vocale/sonora\item[\subglossariobullet] \spzrl{:EdinB}, {\sf ėdin-}:\ farsi; rendersi.
\begin{subvocedue}
\item[Rif.:] \spzcite[154]{kiefferbianchi18351}
\end{subvocedue}
\item[(radice)] \spzrl{_EtdUki^niz}  23 (23)
\item[(radice)] \spzrl{_EtdUkumuzd:H}  11 (8)
\item[(radice)] \spzrl{_Etmi^s}  15 (12)
\item[(radice)] \spzrl{:Etd.Urulm:as.I}  13 (10)
\item[(radice)] \spzrl{:Etd.Ukimiz}  18 (10)
\item[(radice)] \spzrl{:Etd.Ukumuzden}  5 (4)
\item[(radice)] \spzrl{:Etd.Ukumuz:H}  3 (5)
\item[(radice)] \spzrl{:Etm:a:k}  7 (8)
\item[(radice)] \spzrl{:Etm:akl.H}  14 (5)
\item[(radice)] \spzrl{:Etm:ak.H}  14 (23)
\item[(radice)] \spzrl{:Etm:akY}  13 (14)
\item[(radice)] \spzrl{:Etm:ak.In}  14 (27) 16 (8)
\item[(radice)] \spzrl{:Etm.Iyad.I}  9 (2)
\end{subvocedue}
\item[{\color{colorlowref}\spzrl{.J^c:Un}},] {\sf içün},\ post.\ t.:\ per.
\begin{subvocedue}
\item[(simil:1)] \spzrl{.J^c:Un}  7 (9) 13 (11) 15 (4) 16 (17)
\end{subvocedue}
\item[{\color{colorlowref}\spzrl{^EyleB}},] {\sf eyle-},\ va.\ t.:\ fare.
\begin{subvocedue}
\item[Rif.:] \spzcite[160]{kiefferbianchi18351}
\end{subvocedue}
\begin{subvocedue}
\item[(radice)] \spzrl{^Eyledi:k}  20 (3)
\item[(radice)] \spzrl{^Eyley.Hl:ar}  19 (12)
\end{subvocedue}


Contesti:
\begin{subvocedue}
\item[(riga 20)] \spzrl{ta`yiyin ^Eyledi:k}
\item[(riga 19)] \spzrl{:cy.sAl ^Eyley.Hl:ar}
\end{subvocedue}
\item[{\color{colorlowref}\spzrl{.Jl.H}},] {\sf ile},\ post.\ t.:\ con.
\begin{subvocedue}
\item[(simil:1)] \spzrl{.Jl.H}  2 (12) 13 (1) 18 (20) 19 (13) 22 (12) 25 (4)
\end{subvocedue}
\item[{\color{colorlowref}\spzrl{^Ey.U}},] {\sf eyu},\ agg.\ t.:\ buono, bello, bene.
\begin{subvocedue}
\item[Rif.:] \spzcite[163]{kiefferbianchi18351}
\end{subvocedue}
\begin{subvocedue}
\item[(simil:1)] \spzrl{^Ey.U}  8 (4)
\end{subvocedue}
\item[{\color{colorlowref}\spzrl{bir}},] {\sf bir},\ num.\ t.:\ uno.
\begin{subvocedue}
\item[(simil:1)] \spzrl{bir}  9 (9) 14 (16) 15 (8,20) 17 (19) 18 (18)
\end{subvocedue}
\item[{\color{colorlowref}\spzrl{birl.H}},] {\sf birle},\ post.\ t.:\ con.
\begin{subvocedue}
\item[Rif.:] \spzcite[205]{kiefferbianchi18351}
\end{subvocedue}
\begin{subvocedue}
\item[(simil:1)] \spzrl{birl.H}  21 (8)
\end{subvocedue}
\item[{\color{colorlowref}\spzrl{biz}},] {\sf biz},\ pron.\ t.:\ noi.
\begin{subvocedue}
\item[(simil:1)] \spzrl{biz}  15 (0)
\item[(radice)] \spzrl{bizim}  19 (2)
\item[(radice)] \spzrl{biz.H}  14 (8,24) 19 (7)
\end{subvocedue}
\item[{\color{colorlowref}\spzrl{be^s}},] {\sf beş},\ num.\ t.:\ cinque.
\begin{subvocedue}
\item[(simil:1)] \spzrl{be^s}  9 (16)
\end{subvocedue}
\item[{\color{colorlowref}\spzrl{beg}},] {\sf beg, bey},\ n.\ t.:\ principe, signore, governatore di distretto; filgli di paşa; capitani; stranieri di considerazione; qualcosa meno di paşa, una sola coda di cavallo.
\begin{subvocedue}
\item[Rif.:] \spzcite[220]{kiefferbianchi18351}
\end{subvocedue}
\begin{subvocedue}
\item[(radice)] \spzrl{begleri^n}  12 (16)
\end{subvocedue}
\item[{\color{colorlowref}\spzrl{bilB}},] {\sf bil-},\ v.\ t.:\ conoscere, comprendere, sapere; guardare, stimare.
\begin{subvocedue}
\item[Rif.:] \spzcite[226-227]{kiefferbianchi18351}
\end{subvocedue}
\begin{subvocedue}
\item[(var)] \spzrl{bIlB}, {\sf bil-}\begin{subvocedue}
\item[Rif.:] \spzcite[226-227]{kiefferbianchi18351}
\end{subvocedue}
\item[(radice)] \spzrl{bilmAsin}  11 (9)
\end{subvocedue}
\item[{\color{colorlowref}\spzrl{b.U}},] {\sf bu},\ aggp.\ t.:\ questo.
\begin{subvocedue}
\item[\subglossariobullet] \spzrl{bUndan aqdam}, {\sf bundan akdem}:\ prima, prima di questo.
\item[(simil:1)] \spzrl{b.U}  2 (22,28) 3 (13) 4 (0,14) 6 (19) 8 (11,15) 10 (9) 11 (4) 13 (21) 14 (6) 16 (12) 17 (17) 18 (21) 19 (18) 21 (1) 25 (2)
\item[(radice)] \spzrl{b.Undan}  6 (17) 10 (11) 17 (16)
\end{subvocedue}
\item[{\color{colorlowref}\spzrl{b.UlB}},] {\sf bul},\ va.\ t.:\ trovare, scoprire, inventare.
\begin{subvocedue}
\item[Rif.:] \spzcite[247]{kiefferbianchi18351}
\end{subvocedue}
\begin{subvocedue}
\item[(radice)] \spzrl{b.Ulmas.I}  8 (23)
\end{subvocedue}
\item[{\color{colorlowref}\spzrl{b.I^n}},] {\sf bin},\ num.\ t.:\ mille.
\begin{subvocedue}
\item[(simil:1)] \spzrl{b.I^n}  7 (12)
\end{subvocedue}
\item[{\color{colorlowref}\spzrl{buy.UrB}},] {\sf buyur-},\ va.\ t.:\ ordinare, ordinare  di fare,  decretare, dare mandato per iscritto; ordinare in senso cortese; al posto di ėt- quando ci si rivolge ai superiori o si chiede un favore; pregare di.
\begin{subvocedue}
\item[Rif.:] \spzcite[266]{kiefferbianchi18351}
\end{subvocedue}
\begin{subvocedue}
\item[\subglossariobullet] \spzrl{buy.UrulB}, {\sf buyurul-},\ v.\ t.:\ essere ordinato, essere  ordinato che  sia fatto, essere governato.
\item[(radice)] \spzrl{buy.Urilmay.H}  22 (6)
\item[(radice)] \spzrl{buy.Urulmay.H}  6 (15)
\item[(radice)] \spzrl{buy.Ur.Ilm.H}  20 (13)
\end{subvocedue}
\item[{\color{colorlowref}\spzrl{^c^A:vu^s}},] {\sf çavuş},\ n.\ t.:\ çavuş.
\begin{subvocedue}
\item[Rif.:] \spzcite[361]{kiefferbianchi18351}
\end{subvocedue}
\begin{subvocedue}
\item[(simil:1)] \spzrl{^c^A:vu^s}  15 (9)
\item[(simil:1)] \spzrl{^cAwu^s}  15 (17)
\end{subvocedue}
\item[{\color{colorlowref}\spzrl{_h.Od}},] {\sf hod},\ post.\ t.:\ (congiunzione) ma, ora, quanto a.
\begin{subvocedue}
\item[Rif.:] \spzcite[491]{kiefferbianchi18351}
\end{subvocedue}
\begin{subvocedue}
\item[(simil:1)] \spzrl{_hUd}  21 (26)
\end{subvocedue}


Contesti:
\begin{subvocedue}
\item[(riga 21)] \spzrl{q:ilinmAq _hUd}
\end{subvocedue}
\item[{\color{colorlowref}\spzrl{d.eB}},] {\sf dė-},\ v.\ t.:\ dire, chiamare.
\begin{subvocedue}
\item[Rif.:] \spzcite[571]{kiefferbianchi18351}
\end{subvocedue}
\begin{subvocedue}
\item[\subglossariobullet] \spzrl{d.ey:U}, {\sf dėyü},\ \ t.:\ forma arcaica per diye; dicendo, proponendosi; regge frasi principali o simili, quasi come un avverbio postposto.
\begin{subvocedue}
\item[Rif.:] \spzcite[572-573]{kiefferbianchi18351}
\end{subvocedue}
\item[(radice)] \spzrl{d.ey:U}  16 (6)
\end{subvocedue}
\item[{\color{colorlowref}\spzrl{da_h.Y}},] {\sf dahi},\ avv.\ t.:\ anche, ancora, e.
\begin{subvocedue}
\item[Rif.:] \spzcite[512]{kiefferbianchi18351}
\end{subvocedue}
\begin{subvocedue}
\item[(var)] \spzrl{d^A_h.I}, {\sf dahi}\begin{subvocedue}
\item[Rif.:] \spzcite[500]{kiefferbianchi18351}
\end{subvocedue}
\item[(var)] \spzrl{d^A_h.Y}, {\sf dahi}\begin{subvocedue}
\item[Rif.:] \spzcite[500]{kiefferbianchi18351}
\end{subvocedue}
\item[(var)] \spzrl{da_h.I}, {\sf dahi}\item[(simil:1.0)] \spzrl{d^A_h.I}  5 (10)
\item[(simil:1.0)] \spzrl{d^A_h.Y}  15 (1)
\item[(simil:1)] \spzrl{da_h.I}  3 (23) 22 (26)
\end{subvocedue}
\item[{\color{colorlowref}\spzrl{dur}},] {\sf dur},\ va.\ t.:\ copula (è).
\begin{subvocedue}
\item[(simil:1)] \spzrl{dur}  2 (23) 6 (6) 9 (22) 24 (5)
\end{subvocedue}
\item[{\color{colorlowref}\spzrl{dak.In}},] {\sf degin, değin},\ post.\ t.:\ con dativo o con forme in \spzrl{Bin^g.H}: fino a, finché \verificare.
\begin{subvocedue}
\item[Rif.:] \spzcite[536]{kiefferbianchi18351}
\end{subvocedue}
\begin{subvocedue}
\item[(var)] \spzrl{d^rk.In}, {\sf dakin}\item[(simil:1)] \spzrl{d^rkIn}  21 (16)
\end{subvocedue}
\item[{\color{colorlowref}\spzrl{de^nl:U}},] {\sf deŋlü},\ avv.\ t.:\ particella che marca la quantità di una cosa.
\begin{subvocedue}
\item[Rif.:] \spzcite[534]{kiefferbianchi18351}
\end{subvocedue}
\begin{subvocedue}
\item[\subglossariobullet] \spzrl{n.H de^nl:U?}, {\sf ne deŋlü?}:\ quanto?.
\begin{subvocedue}
\item[Rif.:] \spzcite[534]{kiefferbianchi18351}
\end{subvocedue}
\item[\subglossariobullet] \spzrl{.Q de^nl:U}, {\sf aw deŋlü}:\ tanto (quella quantità).
\begin{subvocedue}
\item[Rif.:] \spzcite[534]{kiefferbianchi18351}
\end{subvocedue}
\item[(simil:1)] \spzrl{de^nl:U}  21 (2)
\end{subvocedue}
\item[{\color{colorsologlossario}{\bf (g)}}] {\color{colorsologlossario}\spzrl{dil}}, {\sf dil},\ n.\ t.:\ lingua, linguaggio, dialetto.
\begin{subvocedue}
\item[Rif.:] \spzcite[536]{kiefferbianchi18351}
\end{subvocedue}
\item[{\color{colorsologlossario}{\bf (g)}}] {\color{colorsologlossario}\spzrl{dileB}}, {\sf dile-},\ v.\ t.:\ volere, desiderare, domandare.
\begin{subvocedue}
\item[Rif.:] \spzcite[540]{kiefferbianchi18351}
\end{subvocedue}
\item[{\color{colorlowref}\spzrl{d.U^sB}},] {\sf düş-},\ v.\ t.:\ arrivare, avere luogo, incontrarsi, convenire, concernere.
\begin{subvocedue}
\item[Rif.:] \spzcite[555]{kiefferbianchi18351}
\end{subvocedue}
\begin{subvocedue}
\item[\subglossariobullet] \spzrl{dU^sin umur^nzdah}, {\sf düşen umurüŋüzde}:\ nei vostri eventuali affari, negli affari che vi capiteranno.
\item[(radice)] \spzrl{d.U^sen}  23 (7)
\end{subvocedue}
\item[{\color{colorlowref}\spzrl{siz}},] {\sf siz},\ post.\ t.:\ senza (+ablativo o gerundio in -E).
\begin{subvocedue}
\item[Rif.:] \spzcite[670]{kiefferbianchi18351}
\end{subvocedue}
\begin{subvocedue}
\item[(simil:1)] \spzrl{siz}  22 (17)
\end{subvocedue}


Contesti:
\begin{subvocedue}
\item[(riga 22)] \spzrl{:Wzin.H siz}
\end{subvocedue}
\item[{\color{colorlowref}\spzrl{seks^En}},] {\sf seksen},\ num.\ t.:\ ottanta.
\begin{subvocedue}
\item[(simil:1)] \spzrl{seks^En}  9 (8,15)
\end{subvocedue}
\item[{\color{colorlowref}\spzrl{.so^nr:H}},] {\sf soŋra},\ post.\ t.:\ dopo + ablativo.
\begin{subvocedue}
\item[Rif.:] \spzcite[115]{kiefferbianchi18352}
\end{subvocedue}
\begin{subvocedue}
\item[(simil:1)] \spzrl{.so^nr:H}  2 (18)
\end{subvocedue}
\item[{\color{colorlowref}\spzrl{.toq.Uz}},] {\sf dokuz},\ num.\ t.:\ nove.
\begin{subvocedue}
\item[(simil:1)] \spzrl{.toq.Uz}  9 (20)
\end{subvocedue}
\item[{\color{colorlowref}\spzrl{q^AlB}},] {\sf kal-},\ va.\ t.:\ restare, restare indietro.
\begin{subvocedue}
\item[Rif.:] \spzcite[424-425]{kiefferbianchi18352}
\end{subvocedue}
\begin{subvocedue}
\item[(radice)] \spzrl{q^Alinm^A.gl:H}  10 (8)
\end{subvocedue}


Contesti:
\begin{subvocedue}
\item[(riga 10)] \spzrl{^Elllerind.H q^Alinm^A.gl:H}
\end{subvocedue}
\item[{\color{colorlowref}\spzrl{q:ilB}},] {\sf kıl-},\ va.\ t.:\ fare, operare.
\begin{subvocedue}
\item[Rif.:] \spzcite[502]{kiefferbianchi18352}
\end{subvocedue}
\begin{subvocedue}
\item[(radice)] \spzrl{q:ilinmAq}  21 (25)
\item[(radice)] \spzrl{q:ilinmalar.I}  5 (15)
\end{subvocedue}


Contesti:
\begin{subvocedue}
\item[(riga 21)] \spzrl{firAmU^s q:ilinmAq}
\item[(riga 5)] \spzrl{mar.di_Y||-al-bAl q:ilinmalar.I}
\end{subvocedue}
\item[{\color{colorlowref}\spzrl{q.Ut}},] {\sf kut},\ n.\ t.:\ carisma.
\begin{subvocedue}
\item[(simil:1)] \spzrl{q.Ut}  12 (20)
\end{subvocedue}
\item[{\color{colorlowref}\spzrl{kimesn.H}},] {\sf kimesne},\ pron.\ t?.:\ come kimse: qualcuno, alcuno.
\begin{subvocedue}
\item[(simil:1)] \spzrl{kimesn.H}  8 (6) 14 (18) 15 (23)
\item[(radice)] \spzrl{kimesn.Hni_n}  10 (6)
\end{subvocedue}
\item[{\color{colorlowref}\spzrl{kend:U}},] {\sf kendü},\ pron.\ t?.:\ stesso (kendi).
\begin{subvocedue}
\item[(simil:1)] \spzrl{kend:U}  23 (20)
\item[(radice)] \spzrl{kend:Us.I}  18 (17)
\item[(radice)] \spzrl{kend:Uler.I}  11 (14)
\item[(radice)] \spzrl{kend:Uy.H}  13 (12)
\end{subvocedue}
\item[{\color{colorlowref}\spzrl{g:alB}},] {\sf gel-},\ v.\ t.:\ arrivare, venire.
\begin{subvocedue}
\item[Rif.:] \spzcite[629]{kiefferbianchi18352}
\end{subvocedue}
\begin{subvocedue}
\item[\subglossariobullet] \spzrl{.Ql:H g:alB}, {\sf ola gel-}:\ avere costume di essere fatto.
\begin{subvocedue}
\item[Rif.:] \spzcite[629]{kiefferbianchi18352}
\end{subvocedue}
\item[\subglossariobullet] \spzrl{:Ed.H g:alB}, {\sf ede gel-}:\ aver costume di fare.
\begin{subvocedue}
\item[Rif.:] \spzcite[629]{kiefferbianchi18352}
\end{subvocedue}
\item[(radice)] \spzrl{g:aldikl:ar.I}  4 (20)
\end{subvocedue}


Contesti:
\begin{subvocedue}
\item[(riga 4)] \spzrl{.Jd.H g:aldikl:ar.I}
\end{subvocedue}
\item[{\color{colorlowref}\spzrl{gin.H}},] {\sf gine},\ avv.\ t.:\ di nuovo, nientemeno.
\begin{subvocedue}
\item[Rif.:] \spzcite[649]{kiefferbianchi18352}
\end{subvocedue}
\begin{subvocedue}
\item[(simil:1)] \spzrl{gin.H}  22 (2)
\end{subvocedue}
\item[{\color{colorlowref}\spzrl{g:Oz}},] {\sf göz},\ n.\ t.:\ occhio.
\begin{subvocedue}
\item[Rif.:] \spzcite[660]{kiefferbianchi18352}
\end{subvocedue}
\begin{subvocedue}
\item[\subglossariobullet] \spzrl{g:Oz :EtB}, {\sf göz ėt-}:\ fare attenzione.
\begin{subvocedue}
\item[Rif.:] \spzcite[660]{kiefferbianchi18352}
\end{subvocedue}
\item[(simil:1)] \spzrl{g:Oz}  5 (9) 23 (2)
\end{subvocedue}
\item[{\color{colorlowref}\spzrl{g:OnderB}},] {\sf gönder-},\ v.\ t.:\ inviare.
\begin{subvocedue}
\item[Rif.:] \spzcite[672]{kiefferbianchi18352}
\end{subvocedue}
\begin{subvocedue}
\item[(var)] \spzrl{g:UnderB}, {\sf günder-}\begin{subvocedue}
\item[Rif.:] \spzcite[672]{kiefferbianchi18352}
\end{subvocedue}
\item[(radice)] \spzrl{g:UnderilmIwb}  15 (18)
\item[(radice)] \spzrl{g:UndermakY}  15 (10)
\item[(radice)] \spzrl{g:UnderUb}  19 (0)
\end{subvocedue}
\item[{\color{colorlowref}\spzrl{g.IdB}},] {\sf git-},\ v.\ t.:\ andare, partire.
\begin{subvocedue}
\item[Rif.:] \spzcite[680]{kiefferbianchi18352}
\end{subvocedue}
\begin{subvocedue}
\item[(var)] \spzrl{gitB}, {\sf git-}\begin{subvocedue}
\item[Rif.:] \spzcite[567]{kiefferbianchi18352}
\end{subvocedue}
\item[(radice)] \spzrl{g.Id.Ub}  7 (6)
\end{subvocedue}
\item[{\color{colorlowref}\spzrl{ni^c.H}},] {\sf nice},\ avv.\ t.:\ come? in che modo?.
\begin{subvocedue}
\item[Rif.:] \spzcite[1096-1097]{kiefferbianchi18352}
\end{subvocedue}
\begin{subvocedue}
\item[(simil:1.0)] \spzrl{ni^cah}  22 (21)
\end{subvocedue}
\item[{\color{colorlowref}\spzrl{:v.ErB}},] {\sf vėr-},\ v.\ t.:\ dare (anche ausiliare).
\begin{subvocedue}
\item[Rif.:] \spzcite[1201]{kiefferbianchi18352}
\end{subvocedue}
\begin{subvocedue}
\item[(simil:1.0)] \spzrl{:v.Er}  20 (11)
\item[(radice)] \spzrl{:v.Erdikda-n.su^nraH}  11 (13)
\item[(radice)] \spzrl{:v.ErUb}  7 (18)
\end{subvocedue}
\item[{\color{colorlowref}\spzrl{y^An}},] {\sf yan},\ n.\ t.:\ lato, fianco, profilo.
\begin{subvocedue}
\item[Rif.:] \spzcite[1254-1255]{kiefferbianchi18352}
\end{subvocedue}
\begin{subvocedue}
\item[\subglossariobullet] \spzrl{y^An w.ErB}, {\sf yan ver}:\ mettersi da parte.
\begin{subvocedue}
\item[Rif.:] \spzcite[1254-1255]{kiefferbianchi18352}
\end{subvocedue}
\item[\subglossariobullet] \spzrl{y^Anind.H}, {\sf yaninde}:\ da lui.
\begin{subvocedue}
\item[Rif.:] \spzcite[1254-1255]{kiefferbianchi18352}
\end{subvocedue}
\item[\subglossariobullet] \spzrl{y^Anind.H k^H}, {\sf yanindeki}:\ colui che è da me.
\begin{subvocedue}
\item[Rif.:] \spzcite[1254-1255]{kiefferbianchi18352}
\end{subvocedue}
\item[\subglossariobullet] \spzrl{.Q y^An:H b.U y^An:H yelB}, {\sf oyana bu yana yel-}:\ correre qua e là.
\begin{subvocedue}
\item[Rif.:] \spzcite[1254-1255]{kiefferbianchi18352}
\end{subvocedue}
\item[(radice)] \spzrl{y^An:H}  22 (23)
\end{subvocedue}
\item[{\color{colorlowref}\spzrl{yed.I}},] {\sf yedi},\ num.\ t.:\ sette.
\begin{subvocedue}
\item[(simil:1)] \spzrl{yed.I}  7 (13)
\end{subvocedue}
\item[{\color{colorlowref}\spzrl{yer}},] {\sf yer},\ n.\ t.:\ terra, posto.
\begin{subvocedue}
\item[Rif.:] \spzcite[1262-1263]{kiefferbianchi18352}
\end{subvocedue}
\begin{subvocedue}
\item[(radice)] \spzrl{yerlerden}  13 (6)
\end{subvocedue}
\item[{\color{colorlowref}\spzrl{yar^AB}},] {\sf yara-},\ v.\ t.:\ essere servizievole, utile, adatto; valere, avere valore.
\begin{subvocedue}
\item[Rif.:] \spzcite[1264]{kiefferbianchi18352}
\end{subvocedue}
\begin{subvocedue}
\item[(radice)] \spzrl{yar^Ar}  15 (21)
\end{subvocedue}
\item[{\color{colorlowref}\spzrl{yan^A^sB}},] {\sf yanaş-},\ v.\ t.:\ avvicinare.
\begin{subvocedue}
\item[Rif.:] \spzcite[1284]{kiefferbianchi18352}
\end{subvocedue}
\begin{subvocedue}
\item[\subglossariobullet] \spzrl{yanA^sdirB}, {\sf yanaştır-}:\ far avvicinare.
\begin{subvocedue}
\item[Rif.:] \spzcite[1284]{kiefferbianchi18352}
\end{subvocedue}
\item[(radice)] \spzrl{yanA^sdirmaq}  16 (20)
\end{subvocedue}
\item[{\color{colorlowref}\spzrl{y:Uz}},] {\sf yüz},\ num.\ t.:\ cento.
\begin{subvocedue}
\item[(simil:1)] \spzrl{yUz}  7 (14)
\end{subvocedue}
\item[{\color{colorlowref}\spzrl{y.Ol}},] {\sf yol},\ n.\ t.:\ via, cammino, strada; canale; mezzo, modo.
\begin{subvocedue}
\item[Rif.:] \spzcite[1293-1294]{kiefferbianchi18352}
\end{subvocedue}
\begin{subvocedue}
\item[(radice)] \spzrl{y.Old:H}  7 (21)
\end{subvocedue}
\end{glossario}
