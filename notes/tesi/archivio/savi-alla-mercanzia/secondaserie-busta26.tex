\section{Seconda serie - Busta 26 - Parte prima}

Console Veneto in Cipro 1586-1733 (buste 26 e 27)

I documenti della busta 26 sono raccolti in varie filze numerate.

\setcounter{docnumber}{13}

\subsecdocument{I V savi al Console Cadido di Barbieri}{Venezia}{12 luglio 1600}

\begin{center}
1600: 12 Luglio

À D. Candido di Barbieri Console di Cipro
\end{center}

Con tutto, che sappiamo, che vi sia molto ben nota la causa per la quale
fù levato l'una,  e mera? per miavo?, che  scodeva? delli dannari quel
Consolato, non dimeno intendendo  Noi, che voi costringete li scrivani
à consegnarvi li groppi delli  dannari, et li libretti di carico delle
mercanzie, che vengono à quello scallo con questa operazione contraria
alla nostra  Commissione, andate operando solo  per vostri particolari
fini  et  interessi,  che  cedono  à maleficio  del  Negozio,  et  dei
Mercanti, impedendovi?  anco nelle  boccole?, et gravezze  spettanti à
quel  Bassà, cose  che sono  immediate  contrarie alle  leggi, et  che
devono esser in vero? lontane dal vostro carico, il quale così come ??
diligente di esercitare à vostro  commodo, et beneficio, ma con danno,
et maleficio altrui, così doverete al contrario opperando procurar per
ogni via, che  la Mercantia fosse desgravata?, et  li homeni sollevati
da  queste  spese  estraordinarie,  et  non lecite,  che  si  conviene
dolendosi  specialmente  li  Mercanti  di  non ricever  da  Voi  alcun
fruttuoso  servizio  che  si  deve  però, se  bene  havvessimo  potuto
procedere in altra maniera con  voi, non habbiamo però voluto per hora
venir  ad altra deliberazione,  che all'acclusa  copia della  quale vi
mandiamo, accioche  sia da Voi, et  successori vostri invidiabilmente?
esseguita avvertendovi ad esercitar il nostro carico, à servizio della
Merc.zia  secondo il vostro  obligo, et  per servizio  publico, perche
operando in  contrario saremo costretti di far  quelle provisioni, che
saranno da  Noi giudicate  necessarie contro il  nostro Carico,  et la
persona  vostra, per  il  debito di  Giustizia,  dandoci avviso  della
ricevuta,  et particolar esecuzione  delle presenti,  con le  quali vi
mandiamo anco copia per  vostra maggior istruzione della deliberazione
per nostri  Preccessori? fatta  ?? 30 Agosto  1597, in  tal proposito,
dandone di piu avviso particolare di quello, che haverete tenuto dalli
Marineri? della  Nave per conto  del Consolato, et  boccole?, accioche
possiamo deliberar quello, che ne parerà conveniente in tal proposito.

Li Cinque Savij

\section{Seconda serie - Busta 26 - 3}

\setcounter{docnumber}{17}

\subsecdocument{I V savi al Console Cadido di Barbieri}{}{13 marzo 1599}

\begin{center}
1599. 13. Marzo
\end{center}

Havendone esposto  il Fedelissimo Marchio spinelli fù  Console in quel
Regno che  volendosi partir per  repatriar, et provveder  alli bisogni
della sua infirmità  li fù levata ??? da Sciaban  Bassà di quel Regjno
per causa della quale se volse partirsi fù costretto prima che montare
in Nave di esborsare  ori 250. Si come anco appare di  ciò per fede di
D. Ant. Ciuvano  nostro suo successore de 9  Agosto 1598 et ricercando
lui di  dover esser refatto  di detto denaro havendone  supplicato che
intorno ciò  dovessimo scriver  lettere à Voi  ... Candido  di Barbari
designato console di quest'Isola,  però non essendo ancora Voi partito
per questo  Viaggio. Noi Angiolo?  Basadona? e Collega Savij  sopra la
mercanzia  dicemo à  Voi Console  designato che  giunto che  sarete in
Cipro dobbiate  procurare, et usar  ogni diligentia per quelle  vie, e
con quelli mezzi che sono ordinarij acciò il Spinelli sopradetto possi
esser  reintegrato  di  detti  denari  esborsati per  causa  di  detta
... informandomi prima particolarmente di tutto questo fatto, accioche
possi  esser deliberato  di quel  Cons.o? di  XII quanto  sarà stimato
conveniente, et di ??? ? tal proposito.

\begin{center}
1599. 18. Marzo
\end{center}

?? D.  Alberto et  haver intimati  à .. Candido  di Barbari  il sop.to
ordine, et ... nelle mani  una copia autentica mandato ... ad istanzia
del sop.to Spinelli.
