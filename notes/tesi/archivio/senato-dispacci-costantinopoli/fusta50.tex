\section{Fusta 50}

6 marzo - 26 agosto 1600

% fotografata fino al 17 inclusa e il 20

\subsecdocument{Il bailo Girolamo Cappello al Doge}{Costantinopoli}{4 settembre 1599}

\subsubsecallegato{Lettera di Sinan}{}{}

Al più  glorioso sig.  nella religion del  Messia potente  fra' grandi
nella schera  di Christo  il Bailo  di venezia il  fine del  quale sia
felice.

Dopo  le  salutazioni  date  in  quanto  comporta  l'amicitia  si  fa'
intendere  come hò  riceuuto la  ??  vostra, et  dalla sua  continenza
inteso quanto  mi signifiate. Arrivato, ch'io  fui à ??  ?? il negotio
spettante all'Isola di Timo; et quando anco io andarò à Santa Maura mi
adoperarò come potrò, ne mi risparmierò nelli negotii che appartengono
a Venezia,  dove à Dio piacendo farò  fare diligentissima inquisitione
sopra li schiavi sudditi Ven.ni presi ingiustamente, et tenuti da Bei,
et  altri, li  quali  spero di  liberare.  In sostanza  fino, che  si
conserverà l'amicitia  con il felicissimo  Imperatore si commineranno?
li negotij  nostri giusta la volontà  di sua Maestà  Imperiale. Non vi
scordarete  adunque di  quanto vi  ho  detto quando  io arrivarò  alli
Castelli soggetti à Venetiani.

Il povero Sinan.

\testoright{Tradotta dall'Alberti}

\setcounter{docnumber}{9}

\subsecdocument{Il bailo Girolamo Cappello al Doge}{Costantinopoli}{16 ottobre 1599}

\tuttocifrato

[...]

Hoggi sono comparsi  in Divano due Capigì espediti  da Giafer Bassà di
Tauris,  i quali  avvisano,  che esso  Bassà  era andato  a Ca???  per
incaminarsi poi con le sue  genti, et invadere il paese dei Georgiani,
et per procurarsi di debellare  Simon Principe di quel luogo, il quale
si dice, che se ne stà armato con buon numero di gente, et pronto alla
difesa.

\stopcifrato

Dalle Vigne di Pera a 16 di ottobre 1599

\testoright{Girolamo Cappello Bailo}

\setcounter{docnumber}{15}

\subsecdocument{Il bailo Girolamo Cappello al Doge}{Costantinopoli}{13 settembre 1599}

\docnota{Parla di Giafer verso la fine ma non si capisce}

\setcounter{docnumber}{18}

\subsecdocument{I baili Girolamo Cappello e Vincenzo Gradenigo al Doge}{Costantinopoli}{20 settembre 1599}

[...]

\startcifrato

[...]

L'altra nova è  portata da due Capigì, che  sono ultimamente venuti da
gli  ultimi confini  di Persia,  quali riferiscono  che  Giaffer Bassà
habbia in  certa fattione preso  vivo Simon Giorgiano, et  dicono, che
hauendolo  essi ??? à  condurre alla  presenza del  Bassà, se  ne sono
partiti immediatamente,  et hanno usata  estraordinaria diligenzia nel
viaggio per portare  qui questa buona nova in  confirmazione. Ma quale
sin hora non  si ha altro di  più, che una lettera scritta  di Ar??? à
persona  principale alla  Porta, la  traduttion della  quale  sarà qui
acclusa. L'una,  et l'altra di queste  nove, et più la  seconda che la
prima hà portato gran consolazione à tutta la Porta aspettandosene con
desiderio la confirmatione da lettere di Giaffer Bassà; piaccia à Dio,
che riesca vano, poiché se fusse  vera sarebbe levato à questi un gran
stimolo, et travaglio; con tutto che si ??ede che in questo caso il Re
di Persia non starebbe otioso,  et per vendicare l'offesa del suocero,
et per  impedire à  Turchi maggiori progressi  in quelle  parti. Delli
successi de sollevati nell'Asia non si ha altro avviso, ma per lettere
di mercanti di Aleppo de 18 M passato (?) scritte qua con occasione di
messo estraordinario  à gli ?? mercanti  nostri per lo  ??? di lettere
per Venezia,  s'intende che quel  Bassà haveva sei giorni  prima fatto
serrar le porte  di quella città, et che  sin'all'hora non erano state
aperte  per il  timore, che  esso Bassà  haveva delli  Giannizzeri? di
Damasco sollevati contra di lui, et che si trovano al numero di ??? in
quei contorni,  facendo diversi danni  et per ciò non  permetteva esso
Bassà, che  si aprissero le  porte, per non  esporre et la sua  vita à
manifesto  pericolo, et  quella  Città à  gli  accidenti, che  possono
portare seco  simili sollevationi, e  tumulti; dando anco  li medesimi
mercanti avviso, che Mehmet Bassà destinato contra Hussein? si trovava
in quelle parti ???

[...]

\stopcifrato

[..]

Dalle Vigne di Pera a 20 di settembre 1599

\testoright{Girolamo Cappello Bailo}

\testoright{Vincenzo Gradenigo Bailo}

\subsubsecallegato{Traduzione di lettera scritta al Chiaus Bassà}{}{}

Dopò le salutazioni, amichevolmente vi  si fà sapere, come alli 14 del
presente mese  di Rabia?  Alehir?  mandassimo huomo  per le  porte? di
Andinon?, il quale  ritorno alli 17 del suddetto mese,  cioè alli 4 di
Novembre, et portò vera nova, come il ?? Generale ?? ?? et patrone (la
cui grandezza  conservi l'Altissimo Dio per sempre)  era andato contra
il maledetto  Simon, et l'ha  combattuto, e tagliato à  pezzi tremille
suoi  maledetti,  et  dannati  soldati,  et preso,  et  fatto  schiavo
l'istesso maledetto  Simon, il quale  di ritrova al  presente prigione
appresso  l'istesso  Generale, et  che  per  gratia della  provvidenza
Divina  non   è  restato  morto,  ne   ferito  nessuno  dell'essercito
Mussulmano, et che  è stato messo sottosopra il  paese di esso Simone,
essendo fuggiti  con la sola  vita quelli, che scapolarono  dal taglio
delle spade  Mussulmane; onde speriamo,  che sempre gli  inimice della
nobile  fede  resteranno  sconquassati  con  la testa  bassa,  et  che
continuamente  l'esercito  Mussulmano  si  conserverà  vittorioso,  et
trionfale Iddio sà, et per  essere successo questo fatto, li Persiani,
et altri  nemici si sono  grandemente impauriti, et  spaventati. Acmat
Agà Capigì  del sig.re  Generale nostro padrone,  con una  quantità di
gente, et  con il nipote di  Alessandro, preso nella  battaglia et con
altri infedeli prigioni è per venire alla eccelsa Porta, il quale alli
17 del mese sudetto è gionto da Cars in Arghiron?Ardinon?, il quale se
ne viene per portar la buona  nova del maledetto Simon; però per farvi
anticipatamente  intender questi  habbiamo mandato  Day?  et Mustaffà,
etaltri nostri uomini. Nel? resto?, ???

Il vostro affezionato amico Mustaffà.

\testoright{Tradotta dal Bonis??}

\setcounter{docnumber}{19}

\subsecdocument{Il bailo Girolamo Cappello al Doge}{Costantinopoli}{}

\docnota{Parla di Giafer verso la fine ma non si capisce}

