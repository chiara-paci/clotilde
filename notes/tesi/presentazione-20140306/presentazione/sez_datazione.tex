\mysection{Datazione}

\newcommand\oredaa[2]{\only<#1->{\only<#1>{\color{red}}#2}}

\myframe{Date}{frame:date}{
  \resizebox{\textwidth}{!}{
    \only<1-11,14-17>{
  \begin{tabular}{rl}
    1595-1599 & Incarico di Girolamo Capello \\
    \oredaa{9}{1597-99 & Ca\Ayn{}fer consegna il cotone \\}
    \oredaa{7}{16 aprile 1599 & Destituzione di Ca\Ayn{}fer \\}
    \oredaa{8}{21 ottobre 1599 & 
    Emo (console in Siria) incontra Ca\Ayn{}fer ad Aleppo\\}
    \oredaa{2}{fine 1599 & Arrivo di Vincenzo Gradenigo a Costantinopoli\\}
    \oredaa{10}{1 luglio 1600 & Partenza di Ca\Ayn{}fer per il Mar Nero\\}
    \oredaa{3}{22 febbraio 1600 & Morte di Vincenzo Gradenigo \\}
    \oredaa{4}{inizio luglio 1600 & Arrivo di Agostino Nani \\}
    \oredaa{5}{14 settembre 1600 & Partenza di Capello da Costantinopoli \\}
    \oredaa{11}{24 novembre 1600 & Ca\Ayn{}fer incontra «uno degli  Dragomanni di casa» \\}
    \oredaa{14}{6 gennaio 1600 & Incontro tra Ca\Ayn{}fer e Nani\\}
    \oredaa{15}{7 gennaio 1600 & Spedizione della posta\\}
    \oredaa{16}{17 gennaio 1600 & Rientro a Costantinopoli dei messi\\}
    \oredaa{17}{20 gennaio 1600 & Agostino Nani reinvia le lettere\\}
    \oredaa{6}{26 febbraio 1601 & Relazione di Capello a Venezia \\}
  \end{tabular}}}
  \only<12-13>{\small «Giafer havendo  incontrato uno degli Dragomanni
      di  casa,  gli ha  detto,  che  dovesse  dirmi come  ringraziava
      grandemente Vostra Ser[eni]tà del  favore che le aveva fatto col
      coagiuviar    la   ricuperatione    del   suo    credito   dalli
      {\only<13>{\color{red}} heredi del  già Giacomo dei Biasij}, che
      fù  viceconsule  in Cipro,  come  io  li  giorni precedenti  per
      lettere  ricevute dalli  Ill[ustrissi]mi S[igno]ri  Cinque Savii
      sopra la mercanzia feci sapere al suo cheraià.»

  \mbox{ }\hfill{\it Agostino Nani, 24 novembre 1600}
  }

  \only<18->{\small «Hò  dato conto  alli S[igno]ri Cinque  Savij alla
      mercantia  haver   {\only<19>{\color{red}}dissuaso  il  S[igno]r
        Giafer che  fù Bassà di Cipro  di mandar un  Chiaus à Venetia}
      per  il suo credito  con {\only<20>{\color{red}}Marco  dei Aldi}
      procurando anco di interessarne Vostra Serenità, et holo indotto
      à raccomandare il negotio à qualche commesso costì come hà fatto
      {\only<21>{\color{red}}ad uno agente  delli Abudenti hebrei} che
      se  ne  hanno  preso  il  carico.  Onde  così  ho  stimato  bene
      divertire  il  suo  primo  disegno,  così  credo  che  sarà  con
      sodisfazione  di   lei  alla  quale  anco  scrive   una  sua  in
      raccomandazione del prefatto suo procuratore acciò venghi quanto
      più si  possi occorrendo  dalla publica aut[ori]tà  fovorito per
      maggiormente  {\only<22>{\color{red}}facilitar la  essazione del
        denaro  et  consignazione  delle  robbe}, et  certo  che  ogni
      favore, che sarà dimostrato di  quelo modo che parerà alla Somma
      prudentia  di  V[ostra]  S[ereni]tà  verso la  persona  di  esso
      S[ignor] Giaffer sarà ottimamente impiegata in soggetto di molte
      qualità, et essistimatore  delle cose di mare, et  che si mostra
      benissimo disposto in quella S[erenissi]ma Rep[ubbli]ca.»

    \mbox{ }\hfill{\it Agostino Nani, 20 gennaio 1601}
  }

}

