\startnote

Simboli:
        
\begin{itemize*}        
\item[\spzsussume] sussume;
\item[\spzsussunto] è sussunto da;
\item[\spzunify] unificazione;
\item[\spzbnf] una produzione di una grammatica BNF;
\item[\spzder] una flessione (es.  un plurale da un singolare);
\item[\spzlessder] una derivazione (es.  un verbo da un nome);
\item[\spzcompder] una composizione (es. una preposizione articolata);
\item[$s\spzmregsub t$] indica l'applicazione della sostituzione $s$
  (definita come coppia espressione regolare/sostituzione) alla
  stringa $t$;
\item[\spzconcat] nel caso di stringhe indica concatenazione;
%% \item[\spzto] una trasformazione diacronica;
%% \item[\spzloanto] un prestito da un'altra lingua; 
%% \item[\spzadatt] un adattamento di un prestito alla nuova lingua;
%% \item[\spzevol]    una   trasformazione,   modifica    o   variante;
%%   un'evoluzione incipiente, ma non ancora realizzata;
%% \item[\spzradesuffs]  il risultato  dell'unione tra  una radice  e un
%%   suffisso, con modifica del suffisso;
%% \item[\spzradesuffr]  il risultato  dell'unione tra  una radice  e un
%%   suffisso, con modifica della radice;
%% \item[\spzradesufft]  il risultato  dell'unione tra  una radice  e un
%%   suffisso, con modifica di entrambi;
%% \item[\spzradesuff]  il risultato  dell'unione tra  una radice  e un
%%   suffisso, senza modifiche;
\item[\spzradless{{\it x}}] indica che {\it x} è una radice lessicale;
\item[\spzradtema{{\it x}}] indica che {\it x} è una radice tematica;
%% \item[\formasupposta{{\it  x}}]  indica  che  {\it x}  è  una  forma
%%   supposta, ma  non attestata; 
%% \item[\signsupposto{{\it x}}] indica che il significato di {\it x} è
%%   supposto,  ma  la   forma  di  {\it  x}  è   attestata  con  altri
%%   significati.
\end{itemize*}

