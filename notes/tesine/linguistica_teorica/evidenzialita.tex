\documentclass[twoside,stylearticle,11pt,filologia,it,article,bibsection]{spinoza}

\def\titolo{L'evidenzialità nelle lingue uralo-altaiche e il punto di
  vista narrativo: ipotesi ed esempi per la resa in italiano e altre
  lingue europee.}  

\def\marktitolo{L'evidenzialità nelle lingue uralo-altaiche}

\def\autore{Chiara Paci} 

\def\markautore{Chiara Paci}

\def\parttitolo{\titolo}

\def\spzbibdirectory{}

\let\bibstampa\btPrintCited
%\let\bibstampa\btPrintAll

\pdfinfo{
/Title (\marktitolo)
/Author (\autore)
}

%%%% da trasferire in spz*

\newcommand\spzradless[1]{#1\hspace{.2mm}{\scriptsize $\vdash$}}
\newcommand\spzradtema[1]{#1-}

\newcommand\spzmcompder{\rightrightarrows}
\newcommand\spzmsussume{\sqsupseteq}
\newcommand\spzmsussunto{\sqsubseteq}
\newcommand\spzmunify{\cup}
\newcommand\spzmregsub{\star}
\newcommand\spzmconcat{\sum}
\newcommand\spzmbnf{→}

\newcommand\spzcompder{$\spzmcompder$}
\newcommand\spzsussume{$\spzmsussume$}
\newcommand\spzsussunto{$\spzmsussunto$}
\newcommand\spzunify{$\spzmunify$}
\newcommand\spzregsub{$\spzmregsub$}
\newcommand\spzconcat{$\spzmconcat$}
\newcommand\spzbnf{→}

\newenvironment{lingmeq}{\begin{lingeq}\[}{\]\end{lingeq}}

\usepackage{enumitem}
\setlist{nolistsep}

%% rivedere gli spazi pre o post lingeq

\begin{document}

\newcommand\drawover[2]{
  \draw[lineback] ($0.75*#1+0.25*#2$) to ($0.75*#2+0.25*#1$);
  \draw[grid] #1 to #2;
}

\newcommand\drawcontrasto[4]{
  \draw[<->] #1 to #2;
  \draw #1 -- node[fill=white,sloped]{\tiny #3} ($0.5*#1+0.5*#2$);
  \draw ($0.5*#1+0.5*#2$) -- node[fill=white,sloped]{\tiny #4} #2;
}

\def\alphapref{languages}

%% \newcommand\tablerow[6]{
%%   \draw (#1,#2) -- ($(#1+#3,#2)$) -- ($(#1+#3,#2+1)$) -- ($(#1,#2+1)$) -- cycle;
%%   \node (#4#5) at ($(#1+#3*0.25,#2+.5)$) {#5};
%%   \node (#4#5def) at ($(#1+#3*0.75,#2+.5)$) {\it #6};
%% }

\newcounter{dbtablerow}
\setcounter{dbtablerow}{0}

%\c@dbtabley

\newenvironment{dbtabella}[7]{
  \def\dbtablestyle{#1}
  \def\dbtablex{#2}
  \def\dbtabley{#3}
  \def\dbtablelen{#4}
  \def\dbtablenrows{#5}
  \def\dbtablepref{#6}
  \def\dbtablename{#7}
  \setcounter{dbtablerow}{0}
  \filldraw[\dbtablestyle] (\dbtablex,\dbtabley) 
  -- ($(\dbtablex+\dbtablelen,\dbtabley)$) 
  -- ($(\dbtablex+\dbtablelen,\dbtabley-\dbtablenrows-2)$) 
  -- ($(\dbtablex,\dbtabley-\dbtablenrows-2)$)  -- cycle;
  \filldraw[\dbtablestyle top] (\dbtablex,\dbtabley) 
  -- ($(\dbtablex+\dbtablelen,\dbtabley)$) 
  [sharp corners] -- ($(\dbtablex+\dbtablelen,\dbtabley-1)$) 
  -- ($(\dbtablex,\dbtabley-1)$) [rounded corners]  -- cycle;
  \node (\dbtablename) at ($(\dbtablex+\dbtablelen/2,\dbtabley-.6)$) {\bf \dbtablename};
  \draw[\dbtablestyle] ($(\dbtablex,\dbtabley-1)$) -- ($(\dbtablex+\dbtablelen,\dbtabley-1)$);
  \node[anchortab] (\dbtablename id) at ($(\dbtablex,\dbtabley-1.6)$) {};
  \node (\dbtablename id lab) at ($(\dbtablex+\dbtablelen*.3,\dbtabley-1.6)$) {id};
  \node (\dbtablename id def) at ($(\dbtablex+\dbtablelen*.7,\dbtabley-1.6)$) {\it integer};
  \stepcounter{dbtablerow}
}{}

\newcommand\tablerow[2]{
  \draw[\dbtablestyle] ($(\dbtablex,\dbtabley-1-\thedbtablerow)$) 
  -- ($(\dbtablex+\dbtablelen,\dbtabley-1-\thedbtablerow)$);
  \node (\dbtablename #1) at ($(\dbtablex+\dbtablelen*.3,\dbtabley-1.6-\thedbtablerow)$) {#1};
  \node (\dbtablename #1 def) at ($(\dbtablex+\dbtablelen*.7,\dbtabley-1.6-\thedbtablerow)$) {\it #2};
  \stepcounter{dbtablerow}
}

\newcommand\tableflink[2]{
  \tablerow{#1}{f.k. #2}
}

\newcommand\tablelink[3][right]{
  \draw[\dbtablestyle] ($(\dbtablex,\dbtabley-1-\thedbtablerow)$) 
  -- ($(\dbtablex+\dbtablelen,\dbtabley-1-\thedbtablerow)$);
  \node (\dbtablename #2) at ($(\dbtablex+\dbtablelen*.3,\dbtabley-1.6-\thedbtablerow)$) {#2};
  \node (\dbtablename #2 def) at ($(\dbtablex+\dbtablelen*.7,\dbtabley-1.6-\thedbtablerow)$) {\it f.k.};
  \ifthenelse{\equal{#1}{right}}%
  {\draw[->] ($(\dbtablex+\dbtablelen,\dbtabley-1.6-\thedbtablerow)$) to [out=0,in=180] (#3id.west);}%
  {\draw[->] ($(\dbtablex,\dbtabley-1.6-\thedbtablerow)$) to [out=180,in=180] (#3id.west);}
  \stepcounter{dbtablerow}
}

\newcommand\tableolink[3][right]{
  \draw[\dbtablestyle] ($(\dbtablex,\dbtabley-1-\thedbtablerow)$) 
  -- ($(\dbtablex+\dbtablelen,\dbtabley-1-\thedbtablerow)$);
  \node (\dbtablename #2) at ($(\dbtablex+\dbtablelen*.3,\dbtabley-1.6-\thedbtablerow)$) {#2};
  \node (\dbtablename #2 def) at ($(\dbtablex+\dbtablelen*.7,\dbtabley-1.6-\thedbtablerow)$) {\it f.k.};
  \ifthenelse{\equal{#1}{right}}%
  {\draw[<->] ($(\dbtablex+\dbtablelen,\dbtabley-1.6-\thedbtablerow)$) to [out=0,in=180] (#3id.west);}%
  {\draw[<->] ($(\dbtablex,\dbtabley-1.6-\thedbtablerow)$) to [out=180,in=180] (#3id.west);}
  \stepcounter{dbtablerow}
}

\definecolor{bluea}{rgb}{0.6,0.6,1}
\definecolor{blueb}{rgb}{0.8,0.8,1}
\definecolor{reda}{rgb}{1,0.6,0.6}
\definecolor{redb}{rgb}{1,0.8,0.8}
\definecolor{yellowa}{rgb}{1,1,0.6}
\definecolor{yellowb}{rgb}{1,1,0.8}




\mktitolo{\autore}{\titolo}

\section{L'evidenzialità nelle lingue uralo-altaiche}

\section{L'evidenzialità in italiano}

\section{Evidenzialità e punto di vista in narrativa}

\section{Esempi di traduzione}

\afterpage\clearpage

\clearpage

Abstract di \spzcite{squartini2007}:
\begin{quote}
La presenza in alcune lingue del mondo di mezzi grammaticali di tipo
evidenziale, che segnalano cioè la fonte dell'informazione o, più in
generale, il modo in cui il locutore è venuto a conoscenza del
contenuto proposizionale dell'enunciato, è ormai un dato acquisito
negli studi tipologici (dal pionieristico Chafe/Nichols 1986 fino alla
recente rassegna in Aikhenvald/Dixon 2003). Questi stessi studi hanno
però anche dimostrato che soltanto in alcune lingue l'evidenzialità
viene grammaticalizzata, mentre in molte altre le indicazioni sulla
fonte dell'informazione spettano a mezzi lessicali (lessemi verbali,
avverbi etc.), esterni al sistema grammaticale vero e proprio. Anche
nelle lingue romanze non mancano mezzi di espressione lessicale
dell'evidenzialità: come nota Lazard (2000, 214), dit-on, paraît-il, à
ce que je vois, à ce qu'il semble, apparemment, comme on sait si
riferiscono tutti in qualche modo alla fonte dell'informazione. A
partire da questo quadro generale condiviso, che distingue tra lingue
che grammaticalizzano l'evidenzialità e lingue che si limitano a
lessicalizzarla (Ramat 1996, 290–291), si è assistito recentemente a
diversi tentativi (in particolare a partire da Dendale 1993 e
Guentchéva 1994) di estendere la portata dell'evidenzialità
considerandola come un fenomeno grammaticale interno al sistema
verbale romanzo. In questa prospettiva è stato ad esempio proposto di
dare una caratterizzazione evidenziale di ausiliari modali come devoir
+ infinito (Dendale 1994, Dendale/De Mulder 1996, Haßler 2002,
162–164) e pouvoir + infinito (Tasmowski/Dendale 1994), ma anche di
forme analitiche, come il Passé Composé (Guentchéva 1994), o di veri e
propri morfemi flessivi, come il Condizionale in francese (Dendale
1993, Guentchéva 1994), il Futuro e Condizionale in italiano e in
altre lingue romanze (Coseriu 1976, 80, Radanova-Kuševa 1991–1992,
Squartini 2001), l'Imperfetto Indicativo in spagnolo (Reyes 1990,
1994, Haßler 2002, 164–168, Leonetti/Escandell-Vidal 2002) e in
italiano (Berretta 1992, 141, Squartini 2001).
\end{quote}

\clearpage

Abstract di \spzcite{speas2004}:
\begin{quote}
Some languages have evidential morphemes, which mark the Speaker's source for the information being reported in the utterance. Some languages have logophoric pronouns, which refer to an individual whose point of view is being represented. Notions like “source of evidence” and “point of view” have generally been treated as pragmatic, with few interesting repercussions in syntax. In this paper, I examine constraints on the grammaticization of these notions. I argue that a uniform account of these constraints requires a framework in which there are syntactic projections bearing pragmatically-relevant features. In particular, the facts support the claim of Cinque (Cinque, Guglielmo, 1999. Adverbs and Functional Heads: A Cross-linguistic Perspective. Oxford University Press, New York) that there are projections for Speech Act Mood, Evaluative Mood, Evidential Mood and Epistemological Mode.
\end{quote}


%%%%%%%%%%%%%

%\startnote

Simboli:
        
\begin{itemize*}        
\item[\spzsussume] sussume;
\item[\spzsussunto] è sussunto da;
\item[\spzunify] unificazione;
\item[\spzbnf] una produzione di una grammatica BNF;
\item[\spzder] una flessione (es.  un plurale da un singolare);
\item[\spzlessder] una derivazione (es.  un verbo da un nome);
\item[\spzcompder] una composizione (es. una preposizione articolata);
\item[$s\spzmregsub t$] indica l'applicazione della sostituzione $s$
  (definita come coppia espressione regolare/sostituzione) alla
  stringa $t$;
\item[\spzconcat] nel caso di stringhe indica concatenazione;
%% \item[\spzto] una trasformazione diacronica;
%% \item[\spzloanto] un prestito da un'altra lingua; 
%% \item[\spzadatt] un adattamento di un prestito alla nuova lingua;
%% \item[\spzevol]    una   trasformazione,   modifica    o   variante;
%%   un'evoluzione incipiente, ma non ancora realizzata;
%% \item[\spzradesuffs]  il risultato  dell'unione tra  una radice  e un
%%   suffisso, con modifica del suffisso;
%% \item[\spzradesuffr]  il risultato  dell'unione tra  una radice  e un
%%   suffisso, con modifica della radice;
%% \item[\spzradesufft]  il risultato  dell'unione tra  una radice  e un
%%   suffisso, con modifica di entrambi;
%% \item[\spzradesuff]  il risultato  dell'unione tra  una radice  e un
%%   suffisso, senza modifiche;
\item[\spzradless{{\it x}}] indica che {\it x} è una radice lessicale;
\item[\spzradtema{{\it x}}] indica che {\it x} è una radice tematica;
%% \item[\formasupposta{{\it  x}}]  indica  che  {\it x}  è  una  forma
%%   supposta, ma  non attestata; 
%% \item[\signsupposto{{\it x}}] indica che il significato di {\it x} è
%%   supposto,  ma  la   forma  di  {\it  x}  è   attestata  con  altri
%%   significati.
\end{itemize*}



%% \nocite{chomsky1965}
%% \nocite{chomsky2002}
%% \nocite{simone2008}
\nocite{aikhenvald2003}
\nocite{aikhenvald2004}
\nocite{dehaan2005a}
\nocite{dehaan2005b}
\nocite{johanson2000}
\nocite{johanson2003}
\nocite{squartini2007}
\nocite{wiebel1994}
\nocite{speas2003}
\nocite{speas2004}

\startbiblio

\spzbibsection{Biblio}{evidenzialita}{bib:evidenzialita}

\tableofcontents


\end{document}
