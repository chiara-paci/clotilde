\documentclass[twoside,stylearticle,12pt,filologia,it,article,xelatex,bibsection]{spinoza}

\def\titolo{Clotilde, un'applicazione per la gestione e l'analisi di
  corpus di interesse storico-filologico.}  

\def\marktitolo{Clotilde}

\def\autore{Chiara Paci} 
\def\markautore{Chiara Paci}
\def\parttitolo{\titolo} 
\def\spzbibdirectory{}

\let\bibstampa\btPrintCited
%\let\bibstampa\btPrintAll

%\pdfinfo{
%/Title (\marktitolo)
%/Author (\autore)
%}

\hypersetup{
  pdftitle=\marktitolo,
  pdfauthor=\autore,
%  pdfsubject=\Subjekt,
%  pdfkeywords=\Keywords
}

\usepackage{enumitem}
\setlist{nolistsep}

%% rivedere gli spazi pre o post lingeq

\begin{document}

\newcommand\drawover[2]{
  \draw[lineback] ($0.75*#1+0.25*#2$) to ($0.75*#2+0.25*#1$);
  \draw[grid] #1 to #2;
}

\newcommand\drawcontrasto[4]{
  \draw[<->] #1 to #2;
  \draw #1 -- node[fill=white,sloped]{\tiny #3} ($0.5*#1+0.5*#2$);
  \draw ($0.5*#1+0.5*#2$) -- node[fill=white,sloped]{\tiny #4} #2;
}

\def\alphapref{languages}

%% \newcommand\tablerow[6]{
%%   \draw (#1,#2) -- ($(#1+#3,#2)$) -- ($(#1+#3,#2+1)$) -- ($(#1,#2+1)$) -- cycle;
%%   \node (#4#5) at ($(#1+#3*0.25,#2+.5)$) {#5};
%%   \node (#4#5def) at ($(#1+#3*0.75,#2+.5)$) {\it #6};
%% }

\newcounter{dbtablerow}
\setcounter{dbtablerow}{0}

%\c@dbtabley

\newenvironment{dbtabella}[7]{
  \def\dbtablestyle{#1}
  \def\dbtablex{#2}
  \def\dbtabley{#3}
  \def\dbtablelen{#4}
  \def\dbtablenrows{#5}
  \def\dbtablepref{#6}
  \def\dbtablename{#7}
  \setcounter{dbtablerow}{0}
  \filldraw[\dbtablestyle] (\dbtablex,\dbtabley) 
  -- ($(\dbtablex+\dbtablelen,\dbtabley)$) 
  -- ($(\dbtablex+\dbtablelen,\dbtabley-\dbtablenrows-2)$) 
  -- ($(\dbtablex,\dbtabley-\dbtablenrows-2)$)  -- cycle;
  \filldraw[\dbtablestyle top] (\dbtablex,\dbtabley) 
  -- ($(\dbtablex+\dbtablelen,\dbtabley)$) 
  [sharp corners] -- ($(\dbtablex+\dbtablelen,\dbtabley-1)$) 
  -- ($(\dbtablex,\dbtabley-1)$) [rounded corners]  -- cycle;
  \node (\dbtablename) at ($(\dbtablex+\dbtablelen/2,\dbtabley-.6)$) {\bf \dbtablename};
  \draw[\dbtablestyle] ($(\dbtablex,\dbtabley-1)$) -- ($(\dbtablex+\dbtablelen,\dbtabley-1)$);
  \node[anchortab] (\dbtablename id) at ($(\dbtablex,\dbtabley-1.6)$) {};
  \node (\dbtablename id lab) at ($(\dbtablex+\dbtablelen*.3,\dbtabley-1.6)$) {id};
  \node (\dbtablename id def) at ($(\dbtablex+\dbtablelen*.7,\dbtabley-1.6)$) {\it integer};
  \stepcounter{dbtablerow}
}{}

\newcommand\tablerow[2]{
  \draw[\dbtablestyle] ($(\dbtablex,\dbtabley-1-\thedbtablerow)$) 
  -- ($(\dbtablex+\dbtablelen,\dbtabley-1-\thedbtablerow)$);
  \node (\dbtablename #1) at ($(\dbtablex+\dbtablelen*.3,\dbtabley-1.6-\thedbtablerow)$) {#1};
  \node (\dbtablename #1 def) at ($(\dbtablex+\dbtablelen*.7,\dbtabley-1.6-\thedbtablerow)$) {\it #2};
  \stepcounter{dbtablerow}
}

\newcommand\tableflink[2]{
  \tablerow{#1}{f.k. #2}
}

\newcommand\tablelink[3][right]{
  \draw[\dbtablestyle] ($(\dbtablex,\dbtabley-1-\thedbtablerow)$) 
  -- ($(\dbtablex+\dbtablelen,\dbtabley-1-\thedbtablerow)$);
  \node (\dbtablename #2) at ($(\dbtablex+\dbtablelen*.3,\dbtabley-1.6-\thedbtablerow)$) {#2};
  \node (\dbtablename #2 def) at ($(\dbtablex+\dbtablelen*.7,\dbtabley-1.6-\thedbtablerow)$) {\it f.k.};
  \ifthenelse{\equal{#1}{right}}%
  {\draw[->] ($(\dbtablex+\dbtablelen,\dbtabley-1.6-\thedbtablerow)$) to [out=0,in=180] (#3id.west);}%
  {\draw[->] ($(\dbtablex,\dbtabley-1.6-\thedbtablerow)$) to [out=180,in=180] (#3id.west);}
  \stepcounter{dbtablerow}
}

\newcommand\tableolink[3][right]{
  \draw[\dbtablestyle] ($(\dbtablex,\dbtabley-1-\thedbtablerow)$) 
  -- ($(\dbtablex+\dbtablelen,\dbtabley-1-\thedbtablerow)$);
  \node (\dbtablename #2) at ($(\dbtablex+\dbtablelen*.3,\dbtabley-1.6-\thedbtablerow)$) {#2};
  \node (\dbtablename #2 def) at ($(\dbtablex+\dbtablelen*.7,\dbtabley-1.6-\thedbtablerow)$) {\it f.k.};
  \ifthenelse{\equal{#1}{right}}%
  {\draw[<->] ($(\dbtablex+\dbtablelen,\dbtabley-1.6-\thedbtablerow)$) to [out=0,in=180] (#3id.west);}%
  {\draw[<->] ($(\dbtablex,\dbtabley-1.6-\thedbtablerow)$) to [out=180,in=180] (#3id.west);}
  \stepcounter{dbtablerow}
}

\definecolor{bluea}{rgb}{0.6,0.6,1}
\definecolor{blueb}{rgb}{0.8,0.8,1}
\definecolor{reda}{rgb}{1,0.6,0.6}
\definecolor{redb}{rgb}{1,0.8,0.8}
\definecolor{yellowa}{rgb}{1,1,0.6}
\definecolor{yellowb}{rgb}{1,1,0.8}




\mktitolo{\autore}{\titolo}

Lo strumento che si vuole realizzare servirà ad agevolare la creazione
di corpus digitali di testi di interesse storico-filologico e a
estrarre da questi stessi testi informazioni di carattere linguistico
e storico.

L'obbiettivo principale è minimizzare il lavoro ripetitivo. Per questo
è previsto che l'utente fornisca soltanto:
\begin{itemize}
\item un insieme di testi senza tag; al più l'utente dovrà inserire
  dei separatori (spazi, segni di interpunzione, ecc.) e marker di
  formattazione (allineato a destra o sinistra, sottolineato, ecc.);
\item un insieme di regole con cui analizzare i testi;
\item altri dati necessari all'analisi o alla contestualizzazione
  (lemmari, collocazione e datazione dei testi, ecc.).
\end{itemize}

Per l'inserimento sono previsti dei tool per agevolare i compiti
ripetitivi e per l'import e la conversione di dati preesistenti.

Inoltre il programma potrà utilizzare fonti esterne (dizionari,
database bibliografici, altri corpus, ecc.) e interagire con altri
tool di analisi e gestione di corpus. 

\myfig{Schema delle attività dell'utente.}{\begin{scriptsize}
\begin{tikzpicture}
  [x=1mm,y=1mm,node distance=0,
    >=triangle 45,
    stgrid/.style={draw=#1!10,},
    agrid/.style={stgrid=black},
    bgrid/.style={stgrid=cyan},
    anchortab/.style={circle,draw},
    rbox/.style={rounded corners=4mm,
      top color=white, 
      bottom color=#1!80,
      draw=#1!80
    },
    qbox/.style={sharp corners,
      top color=white, 
      bottom color=#1!80,
      draw=#1!80
    },
    data/.style={rbox=red},
    output/.style={rbox=blue},
    action/.style={qbox=green}
  ]

  %\draw[gray!40] (0,0) grid [step=1] (150,200);
  %\draw[cyan] (0,0) grid [step=5] (150,200);
  %\draw[magenta] (0,0) grid [step=10] (150,200);

  %\draw [data] (0,0) rectangle (10,10);

  %\draw (5,5) node {ciao};

  \draw (-5.5,130) node {\resizebox{10mm}{!}{\includegraphics{immagini/arrow.png}}};
  \draw (-5.5,160) node {\resizebox{10mm}{!}{\includegraphics{immagini/arrow.png}}};

  \clobox{output}{0,0}{globo}{visualizzazione con altri tool}
  \clobox{action}{0,30}{globo}{altri tool}
  \clobox{data}{40,30}{computer}{input ad altri tool}
  \clobox{data}{0,60}{globo}{ouput di altri tool}
  \clobox{data}{0,90}{globo}{altre fonti di informazione}
  \clobox{data}{0,150}{omino}{dati forniti dall'utente}
  \clobox{data}{0,120}{omino}{testi}
  \clobox{action}{80,30}{computer}{moduli di conversione verso altri formati}
  \clobox{action}{40,90}{computer}{moduli di ricerca e normalizzazione}
  \clobox{action}{40,60}{computer}{moduli di conversione da altri formati}
  \clobox{output}{80,180}{computer}{visualizzazione tramite Clotilde}
  \clobox{action}{40,180}{omino}{verifica del risultato}

  \draw[action] (40,120) rectangle ($(40,120)+(30,50)$);
  \draw ($(40,120)+(15,10)$) node {\resizebox{!}{8mm}{\includegraphics{immagini/computer.png}}};
  \draw ($(40,120)+(15,40)$) node {\resizebox{!}{8mm}{\includegraphics{immagini/omino.png}}};
  \draw ($(40,120)+(15,25)$) node {\parbox{19mm}{\scriptsize\centering moduli di ausilio al data entry}};

  \draw[action] (80,60) rectangle ($(80,60)+(30,110)$);
  \draw ($(80,60)+(15,20)$) node {\resizebox{!}{8mm}{\includegraphics{immagini/computer.png}}};
  \draw ($(80,60)+(15,55)$) node {\parbox{19mm}{\scriptsize\centering motore interno}};

  \draw[data] (120,60) rectangle ($(120,60)+(30,110)$);
  \draw ($(120,60)+(15,20)$) node {\resizebox{!}{8mm}{\includegraphics{immagini/computer.png}}};
  \draw ($(120,60)+(15,55)$) node {\parbox{19mm}{\scriptsize\centering database di Clotilde}};

  \draw[->] (15,30)--(15,20);
  \draw[->] (15,50)--(15,60);
  \draw[->] (40,40)--(30,40);
  \draw[->] (80,40)--(70,40);
  \draw[->] (95,60)--(95,50);
  \draw[->] (110,115)--(120,115);
  \draw[->] (30,70)--(40,70);
  \draw[->] (30,100)--(40,100);
  \draw[->] (30,130)--(40,130);
  \draw[->] (30,160)--(40,160);
  \draw[->] (70,70)--(80,70);
  \draw[->] (70,100)--(80,100);
  \draw[->] (70,145)--(80,145);
  \draw[->] (80,115)--(60,115)--(60,110);

  \draw[->] (55,180)--(55,170);
  \draw[->] (95,170)--(95,180);

  \draw[->] (80,190)--(70,190);

\end{tikzpicture}
\end{scriptsize}
}{fig:ridotto}

\myfig{Schema del funzionamento.}{\begin{sideways}
\begin{scriptsize}
\begin{tikzpicture}
  [x=1mm,y=1mm,node distance=3,
    >=triangle 45,
    stgrid/.style={draw=#1!10,},
    agrid/.style={stgrid=black},
    bgrid/.style={stgrid=cyan},
    anchortab/.style={circle,draw},
    rbox/.style={rounded corners=4mm,
      top color=white, 
      bottom color=#1!80,
      draw=#1!80
    },
    qbox/.style={sharp corners,
      top color=white, 
      bottom color=#1!80,
      draw=#1!80
    },
    input/.style={rbox=red},
    output/.style={rbox=blue},
    action/.style={qbox=green}
  ]

  %\draw[gray!40] (0,0) grid [step=1] (230,170);
  %\draw[cyan] (0,0) grid [step=5] (230,170);
  %\draw[magenta] (0,0) grid [step=10] (230,170);

  \node[input]  (testi) {\cloparv{omino}{testi}};
  \node[action] (trascrizione) [below=of testi] {\clopar{omino}{trascrizione}}; 
  \node[action] (verifica)     [right=of trascrizione] {\clopar{omino}{verifica e introduzione di marker}}; 
  \node[action] (ocr)          [above=of verifica] {\clopar{computer}{OCR}};

  \draw[->] (testi) to (trascrizione);
  \draw[->] (testi) to (ocr);
  \draw[->] (trascrizione) to (verifica);
  \draw[->] (ocr) to (verifica);

  \node[input] (dizionari online) [right=of ocr] {\cloparv{globo}{dizionari on-line e/o di terzi}};
  \node[input] (altri database) [right=of dizionari online] {\cloparv{globo}{altri database}};
  \node[input] (dizionari) [right=of altri database] {\cloparv{omino}{dizionari}};
  \node[input] (altre fonti) [right=of dizionari] {\cloparv{omino}{altre fonti e ricerche}};
  \node[input] (grammatiche) [right=of altre fonti] {\cloparv{omino}{grammatiche}};

  \puntomediosouth{dizionari online}{altri database};
  \puntomediosouth{dizionari}{altre fonti};
  \puntomediosouth{grammatiche}{altre fonti};

  %\node (medio dizionari altrefonti) at ($.5*(dizionari)+.5*(altrefonti)-(0,3.5)$) { };
  %\node (medio grammatiche altrefonti) at ($.5*(grammatiche)+.5*(altrefonti)-(0,3.5)$) { };

  \node[action] (ricerca lessico) at ($(medio dizionari online altri database.south)+(0,-12)$) {\clopar{computer}{moduli di ricerca (lessico)}};
  \node[action] (regole) at ($(medio grammatiche altre fonti.south)+(0,-12)$) {\clopar{omino}{inserimento di regole (presunte)}};
  \node[action] (compilazione) at ($.5*(ricerca lessico)+.5*(regole)$) {\clopar{omino}{compilazione e/o trascrizione di dizionari e lemmari}};

  \draw[->] (altre fonti) to (compilazione);
  \draw[->] (altre fonti) to (regole);
  \draw[->] (dizionari) to (compilazione);
  \draw[->] (grammatiche) to (regole);
  \draw[->] (altri database) to (ricerca lessico);
  \draw[->] (dizionari online) to (ricerca lessico);

  \puntomediosouth{ricerca lessico}{compilazione};

  \node[action] (conversione lessico) [below=of medio ricerca lessico compilazione] {\clopar{computer}{moduli di conversione (lessico)}};
  \node[action] (conversione regole) at ($(conversione lessico)+(40,0)$) {\clopar{computer}{moduli di conversione (regole)}};

  \draw[->] (ricerca lessico) to (conversione lessico);
  \draw[->] (compilazione) to (conversione lessico);
  \draw[->] (regole) to (conversione regole);

  \node[output] (testi digitalizzati) at ($(verifica.south east)+(10,-20)$) {\clobigbox{testi digitalizzati}};

  \draw[->] (verifica) to (testi digitalizzati);

  \node[action] (analisi statistica) at ($(conversione lessico.south)+(-20,-22)$) 
       {\clopar{computer}{moduli di analisi morfologica, sintattica, lessicale basati su dati statistici}};
  \node[action] (analisi regole) at ($(conversione regole.south)+(-20,-22)$) {\clopar{computer}{moduli di analisi morfologica, sintattica, lessicale basati su regole}};

  \draw[->] (conversione lessico) to (analisi statistica);
  \draw[->] (conversione lessico) to (analisi regole);
  \draw[->] (conversione regole) to (analisi regole);
  \draw[->] (testi digitalizzati) to (analisi statistica);
  \draw[->] (testi digitalizzati) to [out=-30,in=150] (analisi regole.north);
  \draw[->] (analisi regole) to (analisi statistica);

  \node[output] (non verificati) [below right=of conversione regole] {\clobigbox{oggetti non verificati}};
  \node[output] (regole verificate) [below=of non verificati] {\clobigbox{regole grammaticali verificate}};
  \node[output] (glossari) [below=of regole verificate] {\clobigbox{glossari, dizionari con esempi}};
  \node[output] (testi analizzati) [below=of glossari] {\clobigbox{testi analizzati}};

  \draw[->] (analisi regole) to (non verificati);
  \draw[->] (analisi regole) to (regole verificate);
  \draw[->] (analisi regole) to (glossari);
  \draw[->] (analisi regole) to (testi analizzati);

  \draw[<-] (regole.east) to [out=0,in=0] (non verificati.east);
  \draw[<-] (regole.east) to [out=0,in=0] (regole verificate.east);
  \draw[<-] (regole.east) to [out=0,in=0] (glossari.east);
  \draw[<-] (regole.east) to [out=0,in=0] (testi analizzati.east);

  \node[input] (informazioni strutturate) [left=of testi digitalizzati.north west] {\cloparv{omino}{informazioni strutturate}};
  \node[input] (informazioni non strutturate) [left=of informazioni strutturate] {\cloparv{omino}{informazioni non strutturate}};

  \puntomediosouth{informazioni strutturate}{informazioni non strutturate}

  \node[action] (annotazione documentale) [below=of medio informazioni strutturate informazioni non strutturate] {\clopar{omino}{annotazione documentale}};
  \draw[->] (informazioni strutturate) to (annotazione documentale);
  \draw[->] (informazioni non strutturate) to (annotazione documentale);

  \node[action] (catalogazione) at ($(annotazione documentale.south)+(-5,-20)$) {\clopar{computer}{moduli di ricerca, catalogazione e conversione}};
  \node[input] (altri corpus) [below left=of catalogazione] {\cloparv{globo}{altri corpus}};
  \node[input] (fonti online) [above left=of catalogazione] {\cloparv{globo}{fonti online}};
  \node[input] (database bibliografici) [left=of catalogazione.west] {\cloparv{globo}{database bibliografici}};

  \draw[->] (annotazione documentale) to (catalogazione);
  \draw[->] (fonti online) to (catalogazione);
  \draw[->] (database bibliografici) to (catalogazione);
  \draw[->] (altri corpus) to (catalogazione);

  \puntomediosouth{analisi regole}{analisi statistica}

  \node[action] (analisi semantica) at ($(medio analisi regole analisi statistica.south)+(0,-16)$) {\clopar{computer}{moduli di analisi semantica e pragmatica}};
  \node[action] (annotazione) at ($(analisi semantica.west)+(-50,0)$) {\clopar{computer}{moduli di annotazione}};

  \draw[->] (annotazione documentale) to [out=0,in=90] (annotazione);
  \draw[->] (catalogazione) to (annotazione);
  \draw[->] (testi digitalizzati) to (annotazione);
  \draw[->] (analisi regole) to [in=30,out=-140] (annotazione);
  \draw[->] (analisi statistica) to (annotazione);



  \draw[->] (catalogazione) to (analisi semantica);
  \draw[->] (annotazione) to (analisi semantica);
  \draw[->] (analisi regole) to (analisi semantica);
  \draw[->] (analisi statistica) to (analisi semantica);


  \node[output] (corpus annotato) at ($(annotazione.south)+(0,-20)$) {\clobigbox{corpus annotato}};
  \node[output] (enciclopedia) at ($(corpus annotato.east)+(20,0)$) {\clobigbox{enciclopedia di fatti, luoghi, persone, ecc.}};
  \node[output] (dati quantitativi) at ($(enciclopedia.east)+(20,0)$) {\clobigbox{estrazione di dati quantitativi}};

  \node[action] (annotazione manuale) [above left=of corpus annotato] {\clopar{omino}{regole di annotazione manuale}};

  \draw[->] (analisi semantica) to (enciclopedia);
  \draw[->] (analisi semantica) to (dati quantitativi);
  \draw[->] (annotazione) to (corpus annotato);
  \draw[->] (corpus annotato) to [out=180,in=-90] (annotazione manuale);
  \draw[->] (annotazione manuale) to [out=90,in=180] (annotazione);


\end{tikzpicture}
\end{scriptsize}
\end{sideways}
}{fig:completo}

\section{Interfaccia utente}

L'interfaccia utente è composta di due parti, una di inserimento e una
di visualizzazione.

\subsection{Inserimento dati}

L'inserimento dati riguarda i testi, le regole e i dizionari inseriti
dall'utente, manualmente o con l'ausilio di altri tool.

Per l'inserimento di testi sono previste due modalità: 
\begin{itemize}
\item {\it manuale}, in cui l'utente trascrive a mano il testo (o si
  avvale di un OCR e verifica l'output);
\item {\it importazione}, in cui appositi tool convertono testi già
  trascritti (ad esempio in formato html o odt o provenienti da altri
  corpus) e li inseriscono nel database.
\end{itemize}

Anche per l'inserimento di metadati (collocazione, date, produttore,
ecc.) sono previste le due modalità manuale e importazione. 

Per l'importazione sarà creata un interfaccia che consentirà di
supportare nuovi formati tramite plugin anche di terze parti.

Le regole grammaticali e i dizionari sono salvati nel database in un
formato utile al programma per l'analisi e uguale per tutte le lingue.

All'utente sono forniti dei tool di inserimento (specifici per ogni
lingua) per agevolarlo nell'operazione di data entry e di definizione
delle regole e per consentirgli di importare i dati in modo
massivo. Questi dati vengono poi convertiti nel formato voluto dal
programma.

Anche qui verrà creata un'interfaccia che consenta di sviluppare
plugin per supportare le diverse lingue o diversi criteri di
definizione delle regole.

\subsection{Visualizzazione ed estrazione dei dati}

La visualizzazione riguarderà, di base, gli oggetti ricavati
dall'analisi, quindi:

\begin{itemize}
\item testi analizzati e taggati;
\item regole verificate e glossari;
\item informazioni ricavate dai documenti, di tipo enciclopedico o
  quantitativo.
\end{itemize}

Sarà comunque possibile aggiungere dei moduli per personalizzare la
visualizzazione e l'estrazione.

Sempre tramite plugin sarà possibile esportare i dati anche in formati
adatti ad essere utilizzati con altri tool.

\section{Interfaccia verso l'esterno}

In un'ottica di riuso, il programma sarà in grado di ricercare
autonomamente informazioni, per esempio in dizionari online, su
Wikipedia, in database bibliografici, in progetti opendata, ecc.

Inoltre potrà interfacciarsi con tool esistenti di terze per parti sia
per lo scambio di dati che per demandare l'esecuzione di alcune
analisi.

\section{Motore interno}

Internamente, il programma è composto di due parti: un database e un
motore di analisi. 

Il database contiene i dati forniti dall'utente (dizionari, regole,
testi e metadati), e le strutture di appoggio dei moduli di analisi
del programma (cache, tabelle intermedie, ecc.).

Il motore di analisi è composto da moduli indipendenti di analisi,
pensati come filtri in cascata. Possono essere di tre tipi:
\begin{itemize}
\item di analisi morfologico-sintattica basati su regole grammaticali:
  per compiere l'analisi applicano le regole impostate dall'utente in
  modo deterministico;
\item di analisi morfologico-sintattica basati su dati statistici e
  algoritmi adattativi: eseguono l'analisi cercando di individuare
  delle regole in base a quanto appreso in precedenza;
\item di analisi semantico-pragmatica: estraggono informazioni di tipo
  enciclopedico (luoghi citati, persone, identificazione di eventi,
  ecc.) o di tipo quantitativo, cercando di comprendere l'argomento e
  di inserirlo in un contesto.
\end{itemize}

L'utente potrà scegliere quali moduli inserire o escludere per il tipo
di analisi che gli interessa. I moduli potranno essere aggiunti come
plugin di terze parti sia come componenti creati ad hoc, sia come
interfacce verso tool esterni.

\tableofcontents


\end{document}
