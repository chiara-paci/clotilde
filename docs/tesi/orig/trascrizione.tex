\section{Trascrizione\trascrtitoloadd}

\newcommand\prova[1]{\RL{#1}}

\novocalize

\setcounter{righetesto}{0}
\begin{testo}{righedoc}
%1
\itemtestoarabo{0.96}{\basedoc/rows/arow1.png}{
ifti_hAru    al-umarA'^i   al-`a.zAmi    bi-al-millaTi   al-masI.hIyaTi
\LR{\tref{a1}} `umdaTu al-kubarA'^i al-fi_hAm fI al-.t.tAyfaTi al-`IsUyaTi
\LR{\tref{a2}} mu.sli.hu  mu.sAli.hu ^gamAhiri al-firqatI al-na.srAnIyaTi
\LR{\tref{a3}}  sA.hibu a_dyAli al-.hi^samaTi  wa al-waqAri\LR{\tref{a4}}
.sA.hibu dal'ayli al-^gidi wa al-i`tibAri\LR{\tref{a5}}}
%2
\itemtestoaraboturco{0.96}{\basedoc/rows/arow2.png}{            :vened.Ik
  d.O^z.I\LR{\tref{a6}} _hatamat `awAqibuhu bi-al-_hayri\LR{\tref{a7}}}{
  qawAfil-i mawaddat  wa rawA.hil-i mu.habbat  .Jl.H dUstlu.g:H lAyiq
  .Ql^An darUn.I  du`Alardan .so^nr:H  mu.hibbAnah in.hA .Qlun^An  b.U dur
  k^H  ^ganAb-i  ^sarIfi^niz   .haqqind:H  b.U  mu.hiblar.I}{  Kavafil-i
  meveddet   ve  ravahil-i   muhabbet   ile\tref{1b}  dostluğa   layık
  olan\tref{1a} deruni dualardan  sonra,\tref{1} muhibbane inha olunan
  budur ki\tref{2} cenap-i şerifiniz hakkında\tref{aa} bu muhibları}
%3
\itemtestoturco{0.96}{\basedoc/rows/arow3.png}{
mu.habbat     wa    i`tibAr    `alAmatlar.In     i.zhAr    :Etd.Ukumuz:H
binA'^a  dAymA qiyAs wa  mar^gU wa mAm'ulumuz
b.U  .Jd.Ik^H _hulU.s  .tawiyatt :Uzer.H  .Ql^An  dUstl.U.gumuz muqAbalah||sind:H 
^ganAb-i muruwwat||ma'ablar.I da_h.I}{
muhabbet    ve   i'tibar    alametlerin\tref{bb}    izhar   ėtdüğümüze
binaen\tref{a}  daima kıyas ve  mercu ve  memulumuz bu  idi ki\tref{3}
hulus  taviyet üzere  olan dostluğumuza  mukabelesinde\tref{b} cenab-i
mürüvvetmeabları dahi\tref{4d}}
%4
\itemtestoturco{0.96}{\basedoc/rows/arow4.png}{
b.U  mu.hibb-i  .sAdiq-||al-fuwAdlar.In:H  kamAl  dere^g.H  nI:k _hwAhliq  wa
nihAyat  martabah   _hayir  ^gawIliq  :Uzer.H  .Ql:Hlar  b.U  a^gildan
astAn^h^i  sa`Adat:H irsAl :Ed.H  g:aldikl:ar.I  b^Ayl.Os||lar:H
wufUr-i mawaddat}{
bu    muhibb-i   sadık-al-fuadlarına\tref{4c}    kemal    derece   nik
hahlık\tref{4b}  ve  nihayet   mertebe  hayır  cevilik\tref{4a}  üzere
olalar.\tref{4}    Bu   ecilden\tref{5a}   Astane-i    Saadete   irsal
ėde\tref{5b} geldikleri bayloslara\tref{5c} vüfur-i meveddet}
%5
\itemtestoturco{0.96}{\basedoc/rows/arow5.png}{
wa kamAl-i  mu.habbat i.zhAr :Etd.Ukumuzden  mA`adA ra`AyA wa  barAyA gUz
dA_h.I asUd:H .hAl  wa mar.d.I||-al-bAl qilinmalar.I
_hu.sU.sund:H  mahmA amkan wa  mAtayassur maqdUrumuz  dirI.g 
  .QlunmAdU.g.I}{
ve kemal-i  muhabbet izhar ėtdüğümüzden maada\tref{5}  reaya ve beraya
göz  dahi  asude hal  ve  merza-al-bal kılınmaları  hususunda\tref{6b}
mehmaemken ve mateyessür makdurumuz\tref{6a} diriğ olun-maduğu\tref{6}}
%6
\itemtestoturco{0.96}{\basedoc/rows/arow6.png}{
.hu.dUr-i  pur .hubUrah  .zAhir wa  rU^sun  dur wa  _dAt karImah  sutUdaT
  am.hu.d-ilah  _hafI  wa  nihAn  buyUrulmayah  k^H  b.Undan  aqdam  b.U
  mu.hibblarI wilAyat qubris ayAlatind:H iykan yAqumU byAzI nAm}{
huzur-ı pür hubura zahir  ve ruşun dur.\tref{7} Ve zat kerime\tref{7b}
sütude emhuz  ile\tref{c} hafî ve nihan  buyurulmaya ki\tref{8} bundan
akdem\tref{8bis}    bu    muhibları    vilâyet    Kıbrıs    eyaletinde
iken\tref{8ter} Yaqumu Biazii nam}
%7
\itemtestoturco{0.96}{\basedoc/rows/arow7.png}{
ra`iyati^nizah .husn  i`itiqAdimiz .Qlma.gIn wanadIkah  alUb gIdUb bay`i
:Etm:a:k  iy^cUn yadinah uw^c bI^n yad.I
yUz  sikkah  altUnliq panbah  wIrUb  ba`d  al-qab.d  akar yUld:H  murd
.QlmAmi^s}{
ra`iyetinize hüsn i'tikadımız olmağın\tref{9} Venediğe alup gidup beyi
ėtmek  içün\tref{10}  yedine üç  bin  yedi  yüz  sikke-i altunluk  penbe
verup\tref{11} bad al-kabz\tref{12} eğer yolda murd olmamış}
%8
\itemtestoturco{0.96}{\basedoc/rows/arow8.png}{.Ql.Id.I  nI:k
  _halq wa iyU mustaqIm kimasnah .Qlma.gilah i`itiqAd wa i`timAdimiz b.U
  .Jd.Ik^H  har wa^gihlah  b.U  mu.hibbi^nizah anfa`  wa .Ql.I
  .Ql^An   ma`nAni^n  .zuhUr   bUlmas.I  bAbind:H   diqat  wa
  ihtimAmd:H}{olaydı\tref{13} nik halk  ve ėyu müstakim kimesne
  olmağıla\tref{14}  i'tikad  ve i'timadımız  bu  idi ki\tref{15}  her
  vecihile\tref{15bis}  bu  muhibbinize   enfa\tref{16}  ve  ölü  olan
  ma'nanın zuhur bulması\tref{16bis} babında dikkat ve ihtimamda}
%9
\itemtestoturco{0.96}{\basedoc/rows/arow9.png}{
daqIqah fawt :Etm.Iyad.I lakin _dikir .Qlun^An panbah miz k^H
saksAn  bir qi.ta`ah-i  ^cuwAl  .Jd.Ik^H qubris  qan.tArIlah saksAn  ba^s
qan.tAr wa awn .tuqUz ludraH dur _tamanin.I nafs wanIdikd:H}{
dakika   fevt    ėtmeyedi.\tref{17}   Lakin\tref{18a}   zikir   olunan
penbemiz\tref{18} ki  seksen bir kıt'a-ı çuval  idi\tref{19} ki Kıbrıs
kantarıla  seksen beş kantar\tref{20}  ve on  dokuz lodradır\tref{21b}
semenini\tref{21c} nefis Venedikte\tref{21d}}
%10
\itemtestoturco{0.96}{\basedoc/rows/arow10.png}{murd-i        mizbUri^n
  :v.ere_t.Hs.I wayA_hUd mArqU  d:H^Ald.I nAm kimasn.Hni^n ^Elllerind.H
  qAlinmA.glah   b.U  mu.hiblar.I   b.Undan  aqdam   astAn^h^i  dawlat
  madArad:H   b^Ayl.Os   .Ql^An   ^sukUr  qAbilU   nAm}{murd-i   mezburin
  veresesi\tref{21}  veyahut  Marco  d'Aldi  nam  kimesnenin\tref{22a}
  ell?lerinde   kalınmağla\tref{22}   bu  muhibleri\tref{22b}   bundan
  akdem\tref{22bis}   Astane-i   devletmedarda  baylos   olan\tref{23}
  şükur\tref{24} Kabilu nam}
%11
\itemtestoturco{0.96}{\basedoc/rows/arow11.png}{_hAli.s      dUstUmuzah
  murA^ga`at :EdUb b.U _hu.sU.s.I andan istifsAr itdUkumuzd:H
  bilmAsin  murAd   :EdindUkmiz  _hibar.I   wIrdikda-n.su^nraH
  kandUlerI  dawlatlaH  .Ql  ^gAnib-i dawlat-i  ^ganAbah}{halis  dustumuza
  müracaat   ėdup;\tref{25}   bu   hususi\tref{na1}   andan   istifsar
  itdiğimizde\tref{26}  bilmesin   murad  ėdindiğimiz\tref{27}  haberi
  verdikten  sonra\tref{28} kendüleri\tref{na4}  devletle\tref{na3} ol
  canib-ı devlet-i cenaba\tref{na5}}
%12
\itemtestoturco{0.96}{\basedoc/rows/arow12.png}{diwAnah       .Qlma.gIn
  .hAlAdir   dawlatd:H  .Ql^An   b^Ayl.Os^nizi^n   far.t-i  mawaddat   wa
  mu.habbatinah  ilti^gA'^a wa  dUstlU.ginah i`itiqAd^a  ^cinqU s^A:v.I
  nAm  baglari^n .husn  `adAlatlar.I wa  qUt}{divana olmağın\tref{na2}
  hâlâ   dır.\tref{V2}  Devlette   olan\tref{29}  baylosnuzun\tref{30}
  fart-ı    meveddet   ve    muhabbetine    iltican   ve    dostluğuna
  itikadan\tref{31} Cinqu Savii  nam beylerin\tref{32} husn adaletleri
  ve kut}
%13
\itemtestoturco{0.96}{\basedoc/rows/arow13.png}{.hukUmatlar.I .Jl.H  _tamin-i mazbUr mutawa^g^gih  .Ql^An yarlardan
  ^gam` wa  ta.h.sIl :Etd.Urulm:as.I  iy^cUn kindUyah sipAri^s
  :Etm:akY murAd :Edinmi^s  .Jdi:k ammA mu^sArun-ilayH b^Ayl.Os
  b.U       mu.hibbi^nizah}{hükûmetleri      ile\tref{33}      semen-i
  mezbur\tref{nb1} müteveccih olan\tref{34} yerlerden\tref{nb4} cem ve
  tahsil\tref{nb2} ėtdurulmesi içün\tref{35} kendüye\tref{nb3} sipariş
  itmeği\tref{36}    murad   ėdinmiş    idik,\tref{V1}   amma\tref{37}
  müşarünileyh baylos\tref{38} bu muhibiniza\tref{nc1}}
%%%%%%% fine del primo foglio
%14
\itemtestoturco{0.96}{\basedoc/rows/arow14.png}{_hulU.s   wa  .sadAqat
  :Uzer.H mu.habbat  :Etm:akl.H b.U amird:H  bizah anfa` wa  a.hra.I wa
  alyaq  .QlmasI^cUn wanadIkd:H bir  mu`tamaduN `aliyh  kimasnah wakIl
  na.sb  wa ta`yIn  :Etm:ak.H bizah  anwA`a tar.gIb  :Etm:ak.In}{hulus ve
  sadakat     üzere\tref{nc3}     muhabbet    ėtmekle,\tref{39}     bu
  emirde,\tref{40}  bize\tref{nc2}  enfa  ve ehray  ve  elyak\tref{42}
  olmasıçün\tref{79bis}  Venedikte\tref{nc4}   bir  mu'temedun  aleyhe
  kimesne\tref{nc7}  vekil nasp  ve ta`yın  ėtmeğe\tref{47}  bize enva
  tergib ėtmeğin\tref{46}}
%15
\itemtestoturco{0.96}{\basedoc/rows/arow15.png}{biz   dA_hI   _hu.sU.s-i
  mazbUr  iy^cUn  .Ql  ^ganibah mustaqil^a  bir  ^cAwu^s
  gUndarmakY niyatt itmi^s iykan  yinah mu^sAruN ilayhu b^Ayl.Os ^cAwu^s
  gUndarilmIwb  hamAn   bir  yarAr  mu`tamaduN   `alyah  kimasnah}{biz
  dahi\tref{45}  husus-i   mezbur  için\tref{44}  ol  canibe\tref{nc8}
  müstakillen bir çavuş göndermeği\tref{43} niyet itmiş iken,\tref{41}
  yine,\tref{nc5}       müşarünileyh       baylos\tref{48}       çavuş
  gönderilmiyüb\tref{49}    hemen\tref{nc10}    bir    yarar\tref{nc6}
  ma'(h)hed aleyhe kimesne\tref{nc9}}
%16
\itemtestoturco{0.96}{\basedoc/rows/arow16.png}{wakIl     na.sb     wa
  ta`yiyin .Qlunmaq  kAfIdir diyU tafhIm  :Etm:ak.In ^Ab.Ud.Int.I
  nAm yahUdIlar  b.U mu.hibi^nizi^n mu^garrad  _hA.tirIn murA`At iy^cUn
  wa _hi_dmat yanA^sdirmaq a^gilI^cUn qA'il wa rA.d.I}{vakil
  nasp  ve  ta'yin   olunmak\tref{52}  kâfidir  deyü.\tref{50}  Tefhim
  ėtmeğin\tref{53}     Abudenti      nam     yahudiler\tref{54}     bu
  muhibinizin\footnote{Così  nel   testo.}   mücerred  hatırın  müraat
  içün\tref{55} ve hizmet yanaştırmak eciliçün\tref{56} kail ve razı}
%17
\itemtestoturco{0.96}{\basedoc/rows/arow17.png}{.QldIlar    k^H   nafs
  wanadIkd:H .Ql^An  m.Us^A ma^gA|Awd nAm  ^Adamlarin.I panbaH-i
  mazbUrah _tamanindan harnak^H ta.h.sIl :Edirsah ta.h.sIl 
    miqdAr.I b.Undan b.U mu.hibbi^nizaT  bir wa^gh naqd sikkaH .hasanah
  `ad}{oldiler\tref{57} ki\tref{58}  nafs venedikte olan\tref{59} Musa
  Magiaod    nam   adamlarını\tref{ne1}    penbe-i   mezbura\tref{ne2}
  semeninden herneki  tahsıl ederse\tref{ne3} tahsıl mikdarı\tref{ne4}
  bundan\tref{ne5}  bu  muhibiniza\tref{ne6}  bir vech  nakd\tref{ne9}
  sikke hasene\tref{ne7} ad}
%18
\itemtestoturco{0.96}{\basedoc/rows/arow18.png}{wa   taslIm  :EdUb  wa
  sAbiqA  irsAl .Qlun^An  daftarImiz  mU^gibin^gah sipAri^s  :Etd.Ukimiz
  ispAbdan  harnak^H  muzdUr   mUsAyah  taslIm  .QlunUrsah
    kend:Us.I bir  bAr^cah .Jl.H b.U  ^gAnibah}{ve teslim ėdup\tref{ne8}
  ve     sabıkan\tref{nf2}    irsal     olunan\tref{nf3}    defterimiz
  mucibince\tref{nf4}  sipariş  ėtdiğimiz\tref{nf5} espabdan\tref{nf9}
  herneki\tref{nf7} müzdur Musa'ya\tref{nf6} teslim olunursa\tref{nf8}
  kendüsü\tref{nf10} bir barça ile\tref{nf1} bu caniba\tref{nf12}}
%19
\itemtestoturco{0.96}{\basedoc/rows/arow19.png}{gUndarUb    wa   bizim
  nAmimizah .sin.gUryah  :EdUb wa  bizah bi-al-tamAm taslIm  wa iy.sAl
  aylayah lar .Jl.H .Ql^An _dikir .Qlun^An m.Us^A ma^gA.Qd.I|.I
  b.U _hu.sU.sd:H qab.dah wakIl}{gönderip\tref{nf11} ve\tref{ng3} bizim
  namımıza\tref{ng2}     senguriye     ėdup\tref{ng1}     ve\tref{nh7}
  bize\tref{nh8} bittamam\tref{nh1}  teslim ve ėsal  eyleyeler ile
  olan\tref{nh2}    zikir    olunan    Musa   Macaodii\tref{nh3}    bu
  hususta\tref{nh4} kabza vekil\tref{nh5}}
%20
%\setdocimages
\itemtestoturco{0.96}{\basedoc/rows/arow20.png}{na.sb    wa   ta`yiyin
  .Jldi:k   multamasdir  k^H  mazbUri^n   uwzarInah  na.zar
  `inAyati^niz   mab_dUl wIr dirI.g buyUrIlmah  k^H   _dikir  .Ql^An
  .haqqimizi^n   .hu.sUl.I   muyassir   .Qlah   in||-^sA'a||-al-lah}{nasp  ve  ta'yin  ildik.\tref{nh6} Mültemesdir  ki  mezburun
  üzerine       nazar      inayetiniz      mebzul       vir      diriğ
  buyurulmaya\footnote{Errore  di  Cafer,  ha scritto  \RL{buyUrIlmah}
    invece   di  \RL{buyUrilmayah};   senso,  posizione,   ecc.  tutto
    coincide,  per cui  ha  probabilmente scambiato  due lettere.}  ki
  zikir olan hakkımızın husulu müyessir ola. Inşallah,}
%21
\itemtestoturco{0.96}{\basedoc/rows/arow21.png}{ta`Al.I  b.U
  da^nlU lu.tuf  wa muruwwat qalim miwadat birlah  law.h dilah ta.hrIr
  .QlUr k^H  rUz .ha^srah dakIn ^gAykIr wa  ta.hrIfdan `Ar.I
  wa  bir.I wa firAmU^s qilanmAq _hUd amir-i mihAl }{taali bu
  deŋlü lûtuf ve mürüvvet kalemî  meveddet birle levh dile tahrir olur
  ki  ruz-i haşre değin  caygir ve  tahriftan âri  ve beri  ve feramûş
  kılanmak hod emir-i mıhal}
%22
\itemtestoturco{0.96}{\basedoc/rows/arow22.png}{ayr  wa ginah  riyb wa
  gumAn buyUrulmayah wa  mAdAmk^H sa`AdatlU pAdi^sAah `AlimpinAh .Jl.H
  `ahid wa wi_tAq awzanah  siz in||-^sA'a||-al-lah ta`Al.I awwlukIdan nI^cah
  martabah miz  yAnah b^Ayl.Oslar  wa da_hI ra`AyA}{er  ve gine  reyb ve
  güman buyurulmaya ve madamki  Saadetlû Padişah Alempenah ile ahit ve
  visak  euzine siz  Inşallah  taali evvelkiden  niçe mertebemiz  yana
  bayloslar ve dahi reaya}
%23
\itemtestoturco{0.96}{\basedoc/rows/arow23.png}{wa      barAyA     gUz
  .haqind:H  mazId  mu.habbat  wa dU^san _hi_dmatd:H sa`I  wa  himat  wa  bizl  qudrat
  .Qlmanmaq  muqarrirdir  _dikir   .Qlun^An  yahUdayah  kandU  nafsimiz
  .haqqind:H itdUki^niz `inAyati^n .zuhUr yumnat}{ve beraya göz
  hakkında  mezid  muhabbet ve düşen hizmette sa'i  ve himmet  ve  bizl-i  kudret
  olmanmak  mukarrerdir  zikir  olunan yahudaya  kendü  nafsimiz
  hakkında itdiğiniz inayetin zuhur yümnet}
%24
\itemtestoturco{0.96}{\basedoc/rows/arow24.png}{_habirinah    munta.zir
  wa mutiwaqqifUz  umi_d dur k^H sAyah'i .himAyat^nizd:H  asUd:H .hAl wa
  mar.d.I||-al-bAl .Qlah bAqY ^cUn .gar.d  `ar.d mawaddat wa
  ri^gA.I  .himAyat wa  `inAyatdir}{haberine  muntazir ve  mütevakkıfız
  ümizdir ki  saye-i himayetinizde asude kal ve  merza-al-bal ola baki
  çün garz arz mevedet ve rica-i himayet ve inayetdir}
%25
\itemtestoturco{0.8}{\basedoc/rows/arow25.png}{ta.twIl  .QlunmayUb  b.U
  miqdAr  .Jl.H iktifA .Qlund.I hamsah  .zill wa  ^gUd `AlI
  mamdUd  tAbAn  wa dir_ha^sAn  bAd}{tatvil  olunmayub  bu mikdar  ile
  iktifa olundu hemşe zill ve cud ali memdud taban ve dirahşan bad.}
%firma
\itemtestoturco{0.3}{\basedoc/firma.png}{mu.hibb             mu_halli.s
  bi-al-a_hlAqyy   ^g`afir   myrimyrlar   qubris  sAbiqAm}{muhib
  muhallis bi-l-ahlâkı Cafer mîri-mîrler-i Kıbrıs sabiqam}
\setnodocimages
\end{testo}

