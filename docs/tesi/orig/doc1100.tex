\section{Trascrizione del documento 1100 (traduzione in italiano)}

\label{sec:dragomanno}

Al  glorioso fra  i Principi  della  nation Christiana:  eletto tra  i
grandi nella generazione dei credenti al Messia: Mediatore dei negotii
del Popolo Harmeno: ornato di pomposo manto di gravità e grandezze: Il
Doge di Venezia, il cui fine sia felice.

Doppo le molte, et affettuose salutazioni, che si convengono alla vera
amicizia, et  amore, le si  fà amichevolmente sapere, che,  havendo io
sempre  cercato d'haver  occasione di  palesare l'interno  affetto, et
osservanza, che  porto alla  sua felice et  honorata persona,  hò anco
sempre  sperato, et presupposto,  che in  contraccambio di  questa mia
bona, e  sincera volontà, et  affezione, lei ancora desideri  ogni mio
honore, et  utile, come l'un  suo cordialissimo, et sincero  amico, et
per tanto io oltre li molti  segni di perfetta amicizia, et amore, che
hò mostrato in  diversi tempi alli signori Baili,  che vengono mandato
all'ecc.sa  porta, hò  anco usato  ogn'opera per  mè possibile  in dar
satisfazione, e contento alli suoi  sudditi, e vasalli, si come è noto
alla sua molta  sapienza; hora dunque no le  sarà nascosto che essendo
io suo amatore per il passato Beiler Bey di Cipro, consegnai à Giacome
di  Biasii suo  suddito,  per la  confidenza  che haveva  in lei,  una
quantità di  bombaso per l'importanza di 3700  c[ecchi]ni d'oro, accio
che  lo conducesse seco  a Venezia  per vendere,  il quale  dopò haver
ricevuti  esso  bombaso, se  non  fosse  missiaggio sopragionto  della
morte,  io haveva  fermo  sperato d'haver  ogni  utile, et  vantaggio,
mediante la sua bona servitù, et diligenza, perché son sicuro, che non
haverebbe maneato punto di questo servizio, per esser egli huomo reale
et di boni costumi, mà essendo per la sua morte capitato, il tratto di
esso bombaso,  che fù  appunto sachi 81,  i quali pesavano  cantara 89
cipriotti, et L[odra]  19 in mano delli soi heredi, ò  vero in mano di
Marco  d'Aldi suo  parente,  io  ricorsi dal  G.   Bailo Cappello  mio
amorevolissimo amico,  che si ritrovava  quì all'ecc.sa Porta,  et gli
ragionai  di questo  fatto, il  qual G.   Bailo dopò  haver  scritto à
Venezia per  informazione, si partì poi  con felice augurio,  et se ne
venne in quelle parti, et  io confidato nell'amore, et amicizia del G.
Bailo presente  presi per ispediente di far  raccomandar senz'altro il
negozio all ss.ni cinque savii  in Venezia, accio che mediante la loro
bona  et  essemplar  giustizia,  fosse scosso,  et  recuperato  questo
dananro da quelli, che si ritrovavano haver in mano, mà il suddetto G.
Bailo, per l'affezione,  che mi porta, et per il  desiderio, che hà di
giovarmi m'ha essortato per  maggior faccilità, et utilità del negozio
à dover  creare, et  instituire un commesso  à Venezia, che  sia huomo
reale  et fidato; onde  se bene  io haveva  deliberato di  mandare per
questo  effetto un Chiaus  [Çavuş] apposta,  non dimeno  accettando il
conseglio del sudetto G.  Bailo,  il quale m'hà persuaso esser meglio,
et più espediente creare più  tosto un commesso, che mandare un Chiaus
[Çavuş] aposta,  hò eletto, et instituito per  mio legittimo commesso,
et procuratore in questo negozio  Moiso Masò hebreo, che si ritrova al
penta.ni (??)  v.a  et è agente di Abondanti hebrei,  i quali per amor
mio,  et  per  il  desiderio,  che  hanno  di  farmi  servizio,  hanno
consentito  et contentato che  io dia  a lui  questo carico;  il quale
doverà dunque  ricevere tutto quello,  che sarà scosso,  e recuperarlo
per conto di esso bombaso, et rimetter quì il danaro scosso tanti c.ni
d'oro; et di  più doverà mandare con qualche  Nave tutte quelle robbe,
et  gli  seranno consegnate,  conforme  all'inventario  mandato da  mè
\mancante{6} il passato,  faccendo fare la sicurtà in  nome mio, accio
che il tutto  sia consegnato quì à mè medesimo:  per tanto io desidero
che  il sudetto  hebreo sia  racc.to alla  sua protezione,  et gratia,
acciò che mediante il suo benigno ajuto, si possa scotere et recuprare
questo  credito. La qual  grazia, e  favore, oltreche  sarà eternam.te
impresso et scolpito nel mio animo,  senza esser mai in alcun tempo da
mè scordato, la  si rendi anco certa, e sicura,  che mentre che durerà
questa amicizia, et pace fra l'Imp.e M.tà del nostro feliciss.mo s.re,
et la  sua repub.ca, io ancora  piacendo a Dio non  resterò di portare
sempre ogni sorte d'amore et affetto  alli suoi Baili, et anco à tutti
li suoi sudditi,  e vasalli; prestando ogni servizio  per mè possibile
nelle sue accorrenze, et bisogni.

Staremo dunque  aspettando d'intender nova, et avviso  dal favore, che
s'haverà usato per mio conto ad esso hebreo, il quale desidero, che le
sia anco raccom.to per altro  conto, accio che sotto l'ombra della sua
protezione possa  viver contento,  e sicuro, et  per non le  dar lungo
tedio, s'ha  scritto questo poco,  il quale servirà et  per testimonio
del mio amore,  et per supplicarla di protezzione  et agiuso in quella
mia \mancante{6} occorenza (accorenza?).

Del resto poi l'ombra della  sua vita sia sempre risplendente sopra la
faccia della terra.

Senza data.

La sottoscrizione  dice: Sinceriss.mo et real amico  Giafer già Bayler
Bey di Cipro.

Tradotta da mè Giacomo de Noves interp.te della Ser.ma Sig.ria.

