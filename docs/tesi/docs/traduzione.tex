\section{Traduzione}

[Al]la più grande gloria dei  principi della nazione cristiana, il più
splendente  capo  dei   grandi  nella  comunità  cristiana,  mediatore
pacificatore del popolo di fede cristiana, colui che estende il limite
della modestia  e della sobrietà, al  Doge di Venezia, la  sua fine si
concluda nel bene.

Dopo le  preghiere spirituali che si  convengono all'amicizia, insieme
con i  saluti d'amore  e le onoreficienze  d'affetto, quello di  cui è
fatta la  lettera amichevole è  questo, che, in conesguenza  del fatto
che  abbiamo mostrato  segni di  questo amorevole  amore e  stima [che
  provo]  nei riguardi  della  vostra maestà  di  principe, sempre  la
nostra   ipotesi,  attesa  e   speranza  è   stata  questa,   che,  in
contraccambio  alla nostra  amicizia che  si basa  sulla  sincerità di
pensiero, anche le [vostre]  maestà piene di generosità fossere basate
sul desiderio  del bene di  perfetto grado e sull'ardente  amore delle
fortuna di estremo grado per questo amico dal cuore onesto.

Per questo  motivo, in  aggiunta ad aver  noi mostrato segni  di tanta
amicizia e  perfetto amore  ai baili che  venivano a fare  gli inviati
alla porta, è chiaro e  manifesto al cospetto di numerosi sapienti che
non è  mai stato  negato il meglio  che potevamo  compiere e il  più a
lungo  possibile a  proposito  dello stato  felice dell'interesse  dei
reaya e  dei beraya  e del non  permanere dell'essere malati  di cuore
(=avere preoccupazioni).  E alla  [vostra] nobile persona con lodabile
purezza  non può  essere nascosto  e  segreto che,  siccome il  vostro
suddito  di  nome Giacomo  Biasii,  essendo  nella  regione di  Cipro,
provincia di questi amici (=degli amici che scrivono) prima di adesso,
aveva la  nostra buona  fiducia, diedi nelle  sue mani del  cotone del
valore  in oro  di  3700  monete perché  [lo]  vendesse andando[lo]  a
portare a Venezia e (diedi=verup)  dopo averlo preso, se non fosse che
è morto durante  il viaggio, siccome era una  buona persona e qualcuno
molto onesto,  la nostra  convinzione e fede  era questa, che  in ogni
modo l'utile di  questi nostri amici non avrebbe  perso alcunché nella
cura  e nell'attenzione  del  suo ufficio,  anche  avendo scoperto  la
lettera del significato che era morto.

Ma,  poiché il  corrispettivo del  nostro cotone  che è  oggetto della
preghiera, che  erano sacchi  in numero di  ottantuno, che  in cantara
ciprioti  [erano] ottantacinque  cantara e  19 lodra,  si  trova nella
stessa Venezia, nelle mani di  un erede del summenzionato morto oppure
di  uno di  nome  Marco d'Aldi,  in  precedenza questi  amici si  sono
rivolti al nostro sincero amico di nome Capello, che sia ringraziato!,
che  era bailo  al centro  di grandezza  della Porta;  questo distinto
[bailo],  dopo  aver  mandato   notizia  di  quello  che  desideravamo
conoscere nell'inchiesta  che avevamo sollecitato da lui,  è in questo
momento che è stato richiamato davanti a [voi] stessi (che faccia buon
viaggio!) \mancante{6}. 

Essendo  nello  stato  di  fiducia  verso l'esagerazione  di  amore  e
amicizia e di confidenza verso  l'amicizia del vostro bailo che era al
``governo'',  avevo  espresso  allo  stesso [bailo]  il  desiderio  di
inviare un ordine perché fosse fatto fare, con la squisita giustizia e
i felici governi  dei principi di nome Conque Savii,  il recupero e la
raccolta dei luoghi che sono disposti verso il (=dove si trova il) (?)
suddetto  denaro, ma  il  summenzionato Bailo,  con una  conversazione
amichevole  per  la sincerità  e  lealtà  verso  questi vostri  amici,
siccome avevamo espresso  dal canto nostro il proposito  di inviare un
{\it  çavuş} velocemente  in quella  direzione, perché  stimolasse per
ogni ogni sorta  di cose per nominare e designare  a venezia un agente
che fosse qualcuno fidato, invece il summenzionato Bailo disse che non
[era necessario] inviare un {\it çavuş} e che era sufficiente solo che
fosse nominato e designato a  Venezia un agente stipendiato, che fosse
qualcuno fidato.

Dopo averglielo chiesto, gli ebrei di nome Abudenti, unicamente per il
rispetto dell'influenza di  questi nostri amici e per  il desiderio di
procurare utilità, sono stati contenti  e compiaciuti che un loro uomo
di nome  Musà Magiaod che è  a Venezia centro, noi  [lo] nominassimo e
designassimo come  procuratore plenipotenziario in  questa faccenda il
suddetto Musà Magiaod
