\mode
<article>

\newpage

\mode
<all>

\mysection{La quantità}

\myframe{Unità di misura}{frame:unitamisura}{
  \begin{center}
    \begin{tabular}{rl}
      \includegraphics[scale=0.15]{documenti/Bilancioimg-000.png}
      &\includegraphics[scale=0.15]{documenti/Bilancioimg-001.png}
    \end{tabular}
  \end{center}
}


\myframe{Conversione}{frame:conversione}{
  \begin{center}
    {\color{evidenzia}\bf 85 cantara di Cipro e 19 lodra}
    
    \vfill
    
    \only<1>{\scriptsize
      \rule[-1em]{0pt}{6cm}\begin{tabular}[b]{rcccl}
        1 &cantara di Cipro &$\cong$& 750 &libbre sottili veneziane\\
        1 &libbra sottile veneziana&$\cong$& 300 &g\\
        1 &cantara &$\cong$ &225 &kg \\[1cm]
        1 &cantara &$=$&44& okka\\
        1 &okka&$=$&4& lodra\\
        1 &cantara&$=$&176& lodra\\[0.5cm]
        {\it oppure}\\[0.5cm]
        1 &cantara&$=$&100& rotoli\\
        1 &rotolo&$=$&1& lodra\\
        1 &cantara&$=$&100& lodra\\
      \end{tabular}
    }
    \only<2>{\scriptsize
      \rule[-1em]{0pt}{6cm}\begin{minipage}[b]{\textwidth}
        \begin{eqnarray*}
          85\,\text{cantara} + 19\, \text{lodra} 
          &=& 85+\frac{19}{176}\, \text{cantara} \\
          &=& \left(85+\frac{19}{176}\right)225\, \text{kg}\\
          &=& 19149.29\, \text{kg} \cong 20\,\text{t}\\[1cm]
          85\,\text{cantara} + 19\, \text{lodra} 
          &=& 85+\frac{19}{100}\, \text{cantara} \\
          &=& \left(85+\frac{19}{100}\right)225\, \text{kg}\\
          &=& 19167.75\, \text{kg} \cong 20\,\text{t}
        \end{eqnarray*}
      \end{minipage}
    }
  \end{center}
}

\myincslide{frame:conversione}{2}

\mode
<article>

Per  stimarne  il  peso  in  chilogrammi, utilizziamo  un  manuale  di
commercio del  XVIII secolo. Questo manuale  è più tardo  di circa 150
anni, ma  il valore indicato per  il cantara è coerente  con quello di
altre fonti \cite[p. 40]{triulzi1766}.

Secondo questo manuale,  un cantara di Cipro corrisponde  a 750 libbre
sottili veneziane.   Una libbre sottile veneziana varia  a seconda del
periodo intorno ai 300 g (301.2 g), quindi il cantara di Cipro sarebbe
più o meno 225 kg \cite{websizes}.

Per quanto rigurda la lodra, può  essere intesa in due modi. In alcuni
casi è un  sinonimo del rotolo e quindi corrisponde  a un centesimo di
cantara \cite{websizes}. In  altri casi è un quarto  di un'okka, che a
sua volta è un quarantaquattresimo di cantara, il che dà 176 lodra per
cantara \cite[voce {\it lodra}]{okyanus}.

Questo valore è  comunque ininfluente, dato che in  entrambi i casi si
arriva ad una stima di poco meno di 20 tonnellate.

La  parola   lodra  è  formata   sull'arabo  {\it  rotl}   (rotolo)  e
sull'italiano {\it libra} \cite[voce {\it lodra}]{okyanus}.

\mode
<all>

\myframe{I sacchi}{frame:sacchi}{
  \begin{center}
    {\color{evidenzia}\bf 81 sacchi}
    
    \vfill
    
    \parbox[c]{0.5\textwidth}{\centering
      \begin{equation*}
        \frac{20\,\text{t}}{81\,\text{sacchi}}\cong 247 \,\text{kg per sacco}
      \end{equation*}
    }
    \hfill\parbox[b]{0.4\textwidth}{
      \begin{tabular}{c}
        \only<1>{\cparbox{0.38\textwidth}{6cm}{\includegraphics[scale=0.3]{documenti/luzerner.png}}}
        \only<2>{\cparbox{0.38\textwidth}{6cm}{\includegraphics[scale=0.4]{documenti/luzerner-sacco.png}}}\\
                {\tiny {\sc Diebold Schillieg}, {\it Luzerner Chronik}, 1517}\\
    \end{tabular}}
  \end{center}
}

\myincslide{frame:sacchi}{2}

\myframe[3]{Il volume del cotone}{frame:dimensione}{
  \begin{center}
    \input{frame/camioncotone}
  \end{center}
  
  {\footnotesize
    {\it Bales  of cotton being  transported from  Bakersfield to  the cotton
      compress on Terminal Island}, 
    ca. 1932,
    Long Beach Public Library,
    Long Beach, California}
}

\mode
<article>

La dimensione  delle balle  che si vedono  nella riproduzione  è molto
simile a quelle di questa foto.  Nella foto ci sono circa 70 balle, un
valore molto vicino a quello della lettera.

\mode
<all>

\myframe{Galeone veneziano}{frame:galeone}{
  \begin{center}
    \parbox[c]{0.7\textwidth}{\includegraphics[scale=0.8]{navi/santissimamadre.png}}
    \hfill
    \parbox[c]{0.28\textwidth}{\tiny {\it Santissima Madre}, Venezia, circa 1550}
  \end{center}
}

\mode
<article>

Da \cite[p. 48]{cucari2004}

Generalmente, i grossi traffici veneziani avvengono con navi di grosso
dislocamento  (722 t).  Tuttavia,  alla fine  del  secolo (dal  1573),
nessuno è più in grado di  costruire una nave del genere senza l'aiuto
dello stato \cite[p. 325]{braudel1982}.

\mode
<all>

\myframe{Marciliana}{frame:marciliana}{
  \begin{center}
  \includegraphics[scale=0.4]{navi/marciliana.png}
  \end{center}
}

\mode
<article>

Compaiono quindi navi  di stazza più piccola, sulle  150 t, dette {\it
  marciliane}. Ma l'incapacità (o la  non volontà) di Venezia a varare
più  navi  di  questo  tipo  favorisce la  navi  Francesi,  Inglesi  o
Olandesi.  Questi  mercanti ``stranieri'' invadono  i porti e  sono in
grado  di pagare  di più  le merci,  anche lo  stesso cotone  di Cipro
\cite[p. 329]{braudel1982}.

\mode
<all>

\myframe{Susan Constant}{frame:susanconstant}{
  \begin{center}
    \includegraphics{navi/susanconstant.png}

    \vspace{1cm}

    {\tiny {\it Susan Constant (ricostruzione)}, London, 1605}
  \end{center}
}

\myframe{Mary Rose}{frame:maryrose}{
  \begin{center}
  \input{frame/maryrose.tex}

  {\tiny {\it Mary Rose}, Portsmouth, 1511}
  \end{center}
}

\mode
<article>

Mary     Rose:    da     \cite[p.     97]{marsden2003}    (foto)     e
\cite[p. 93]{marsden2003} (disegno).

Susan Constant: da \cite[p. 88]{lavery2005}.

\mode
<all>


