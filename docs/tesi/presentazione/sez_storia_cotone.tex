\mysection{Storia del cotone}

\myframe{Erodoto}{frame:erodoto}{

\textgreek{
τὰ δὲ δένδρεα τὰ ἄγρια αὐτόθι φέρει καρπὸν εἴρια καλλονῇ τε προφέροντα καὶ ἀρετῇ τῶν ἀπὸ τῶν ὀίων: καὶ ἐσθῆτι Ἰνδοὶ ἀπὸ τούτων τῶν δενδρέων χρέωνται.}

{\mbox{ }\hfill  Erodoto, {\it Storie, 3.106.3}, V sec. a.C.}

\vspace{1cm}

Laggiù le piante selvatiche producono come frutto una lana che supera
in bellezza e qualità quella delle pecore: e per le vesti gli Indiani
si riforniscono proprio con queste piante.

}

\myframe{Prospero Alpino}{frame:alpino}{
  \begin{tabular}{cp{5cm}}
    \multirow{3}{*}{\includegraphics[scale=0.65]{documenti/alpino.jpg}}&
    {\sc Prosper Alpinus},\\& {\it De Plantis Aegypti Liber},\\& Venezia 1592\\[6cm]
  \end{tabular}
}

\myframet{Agnello vegetale}{A. vegetale}{frame:agnellovegetale}{
  \setfarsi
  \novocalize
  \tiny
  \begin{tabular}{cp{0.5\textwidth}}
    \includegraphics[scale=0.5]{documenti/Mandeville_cotton.jpg}&
    {\centering\includegraphics[scale=0.15]{documenti/Vegetable_lamb.jpg}}\\[0.4cm]
    Jehan  de Mandeville, 1357-1371 & {\centering  Johann Zahn's, {\it
        Specula Physico-Mathematico-Historica Notabilium ac Mirabilium
        Sciendorum}, Norinberga, 1696}\\
  \end{tabular}
}

\myframet{Il viaggio del cotone}{V. cotone}{frame:viaggiocotone}{
  \begin{center}
    \includegraphics[scale=0.3]{atlas/viaggio-cotone.png}
  \end{center}
}


%% \myframe{Sacchi}{frame:sacchi}{
%%   \begin{tabular}{p{\textwidth}}
%%     \includegraphics[scale=0.3]{navi/lavoratoriporto.jpg}\\  {\it Lavoratori del
%%       porto di  Genova tra  sacchi di cotone  (a sinistra) e  balle di
%%       lane  (a destra)}, fine 800\\
%%   \end{tabular}
%% }

%% \myframe{Cotone imbarcato}{frame:imbarcato}{
%%     \begin{center}
%%       \input{frame/cotoneimbarcato}
%%       {\it Bales of cotton on a steamboat dock ready for shipping}
%%     \end{center}
%% }
