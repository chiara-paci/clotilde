\mysection{Suffissi}

\myframe{Accusativo}{frame:accusativo}{
\spzsuffottomano{.I}{-I}{I}{accusativo}

\begin{tabular}{rll}
\spzrltab{:Adamlarin.I} & adamlarını
\end{tabular}
}

\myframe{3a singolare (nom.)}{frame:terzasingnom}{
\spzsuffottomano{-sin}{-(s)I(n)}{I}{terza persona singolare (nominale)}

Può essere  scritto anche  \spzrl{s.In}.

\vspace{1cm}

\begin{tabular}{rll}
\spzrltab{:Adamlarin.I} & adamlarını \\
\spzrltab{d.O^z.I} & dojı \\
\spzrltab{kend:Us.I} & kendüsi \\
\spzrltab{kend:Uler.I} & kendüleri \\
\spzrltab{muqAbal:H||sind:H} & mukabelesinde \\
\end{tabular}
}


\myframe{Prima persona plurale (verb.)}{frame:primaplurverb}{
\spzsuffottomano{.Uz}{-Wz}{II}{prima persona plurale (verbale)}

\begin{tabular}{rll}
\spzrltab{mutawaqifUz} & muteväkifüz  \\
\end{tabular}
}

\myframe{Converbio}{frame:converbio}{
\spzsuffottomano{y.Ub}{-(y)Wb}{II}{converbio}

\begin{tabular}{rll}
\spzrltab{^Al.Ub} & alup  \\
\spzrltab{g.Id.Ub} & gidüp  \\
\spzrltab{.QlunmayUb} & olunmayup  \\
\end{tabular}
}


\myframe{Nome/aggettivo verbale}{frame:duk}{
\spzsuffottomano{d.Uk}{-dWḰ}{IIIa}{nome/aggettivo verbale}

Può essere  scritto anche \spzrl{duk}.  

\vspace{1cm}

\begin{tabular}{rll}
\spzrltab{:Etd.Ukümüz} & ėtdükümüz  \\
\spzrltab{:v.Erd:ukden} & vėrdükden  \\
\spzrltab{.QlunmAdU.g.I} & olunmadu$\gamma$ı  \\
\end{tabular}

}

\myframe{Fattitivo}{frame:fattitivo}{
\spzsuffottomano{-|dur}{-DXr}{IIIa}{fattitivo}

\begin{tabular}{rll}
\spzrltab{:Etd:Urulm:as.I} & ėtdürülmesi \\
\spzrltab{yan^A^sd:ur|-} & yanaştür-\\
\end{tabular}
}

\myframe{Aggettivo}{frame:lu}{
\spzsuffottomano{l.U}{-lW$^{\text{ə}}$}{IIIa}{aggettivale denominale}

  \spzrl{sa`Adatl:U}, \spzpr{seadetlü}
}

\myframe{1a e 2a plurale (nom.)}{frame:unodue}{ 
\spzsuffottomano{-|umuz}{-(X)mWz}{IV-II}{prima persona plurale (nominale)}

\spzsuffottomano{-|u^nuz}{-(X)ŋWz}{IV-II}{seconda persona plurale (nominale)}

\only<2->{\spzsuffottomano{-|um}{-(X)m}{IV}{}}
\only<2->{\spzsuffottomano{-|u^n}{-(X)ŋ}{IV}{}}
\only<3->{\spzsuffottomano{-|uz}{-Wz}{II}{plurale}}

\begin{tabular}{rll}
\spzrltab{dUstUmuzah} & dostumuza  \\
\spzrltab{daftarImüz} & defterimüz  \\
\spzrltab{penb.Hmüz} & pembemüz  \\
\spzrltab{b^Ayl.Osu^nuzu^n} & baylosuŋuzuŋ  \\
\spzrltab{mu.hibbi^niz.H} & muhıbınıza  \\
\end{tabular}
}

\myframe{Postposizione (ile)}{frame:post}{
\spzsuffottomano{Il.H}{-ile}{0}{postposizione \spztr{con}, \spztr{e}}

\begin{tabular}{rll}
\spzrltab{.Qlma.gil.H} & olma$\gamma$ile \\
\end{tabular}
}







