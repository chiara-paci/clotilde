\mode
<article>

\newpage

\mode
<all>

\mysection{Il valore}

\myframe{Altun}{frame:altun}{
  \begin{center}
    \begin{tabular}{c}
      \includegraphics{monete/marino_grimani.jpg}\\
      {\it ducato veneziano} coniato sotto Marino Grimani (1595-1605)\\
      22 mm, 3.50 g, 500 \EUR\\[0.5cm]
      \includegraphics[scale=0.25]{monete/sultani_mehmedIII.png}\\
      {\it sultanî} coniato ad Amasya sotto Mehmed III (1003/1597) \\
      19-22 mm, 3.28-3.50 g\\
    \end{tabular}

  \end{center}
}

\mode
<article>

Nella slide si vedono due  monete d'oro contemporanee alla lettera. La
prima è un ducato veneziano, la seconda un sultanî.

La prima immagine è tratta dal  sito di una casa d'aste, che valuta la
moneta 500 \EUR. I valori di diametro e peso sono quelli della moneta.

La  seconda da  un  sito di  numismatica  ottomana, per  cui i  valori
indicano  il  range che  poteva  avere  questa  moneta.  Gli  ottomani
coniavano  monete  in  più  zecche  sparse  per  l'impero.  Questa  in
particolare viene da Amasya.

\mode
<all>

\myframe{Akçe}{frame:akce}{
  \footnotesize
  \begin{center}
    \begin{tabular}{p{0.48\textwidth}p{0.48\textwidth}}
      {\centering\includegraphics[scale=0.25]{monete/13-1003-dirhem-haleb.png}}&
      {\centering\includegraphics[scale=0.25]{monete/akce2.png}}\\
      {\centering{\it dirhem} coniato a Aleppo sotto Mehmed III (1003/1597), 19 mm, 2.02-2.70 g} &
      {\centering{\color{evidenzia}\it akçe} coniato a Cipro sotto Mehmed III (1003/1597),
        11 mm, 0.35 g}\\
      {\centering 100 dirhem = 800 akçe} & {\centering 1 altun = 120 akçe} \\[5mm]
    \multicolumn{2}{c}{\includegraphics[scale=0.25]{monete/1587-8reales-felipe2-segovia.jpg}}\\
    \multicolumn{2}{c}{{\it real de a ocho} coniato in Spagna sotto Filippo II (1597), 40 mm, 27.2 g}\\
    16 real = 1 altun & 2 real de a ocho = 1 altun \\
    
    \end{tabular}
  \end{center}
}

\mode
<article>

Per  avere  un'idea  del  valore  di 3700  monete  d'oro,  è  comunque
necessario   cercare  un'equivalenza  con   il  valore   delle  monete
d'argento, dato che salari, prezzi  e bilanci erano espressi in questa
unità di misura.

In questo momento, nell'impero ottomano  si usa due tipi di monete: la
moneta d'oro ({\it altun}), usata  per gli scambi internazionali, e la
moneta d'argento ({\it akçe}), usata invece per le spese correnti.

Mentre l'{\it altun}  ha una quantità d'oro e  un valore relativamente
stabile  nel tempo e  uniforme nel  Mediterraneo, la  moneta d'argento
subisce una forte e veloce svalutazione a partire dall'inizio del XVI.
Questo fenomeno  è chiamato {\it  Rivoluzione dei prezzi}  e colpisce,
anche se in modo diverso, tutto  il mondo europeo in senso molto largo
\cite{braudel1982,barkan1975,pamuk2001}.

\mode
<all>

\myframe{La svalutazione delle monete d'argento}{frame:svalargento}{
  \includegraphics[scale=0.9]{schemi/svalutazione.png}
}

\myframe[5]{Il valore del cotone in akçe}{frame:valakce}{
  \begin{center}
    \begin{tabular}{>{\it}rccccc|ccc}
     &&&&&& \multicolumn{3}{l}{3700 monete d'oro}\\[1cm]
     \visible<2-6>{ufficiale} & \visible<2-6>{1} & \visible<2-6>{altun} & \visible<2-6>{=} & \visible<2-6>{120} & \visible<2-6>{akçe} & 
     \visible<3-6>{=} &\visible<3-6>{444\hspace{0.2em}000} & \visible<3-6>{akçe}\\[1cm]
     \visible<4-6>{reale} & \visible<4-6>{1} & \visible<4-6>{altun} & 
     \visible<4-6>{$\cong$} & \visible<4-6>{180} & \visible<4-6>{akçe}&  
     \visible<5-6>{=} & \visible<5-6>{666\hspace{0.2em}000} & \visible<5-6>{akçe}\\[1cm]
    \end{tabular}

    \vspace{1cm}

    \visible<6>{ {\it Narh Defteri}, 15 novembre 1600 } 
  \end{center}
}

%\myincslide{frame:valakce}{5}

\mode
<article>

Negli anni tra il 1585 e il 1599, il valore dell'{\it akçe} continua a
calare drammaticamente e soprattutto  si discosta dal valore ufficiale
di cambio \cite[p. 12-14]{barkan1975}.

L'amministrazione interverrà il 15 novembre  del 1600 con il {\it Narh
  Defteri}, che regola il  valore dell'{\it akçe} e contemporaneamente
il prezzo delle merci, riportando il valore di cambio a 120 {\it akçe}
per un  {\it altun}  \cite[p. 13]{barkan1975}. Quest'evento  è tuttavia
successivo al documento.

\mode
<all>

\myframe[4]{Indice dei prezzi}{frame:indice}{
  \begin{center}
    \input{frame/indiceprezzi}

    {\footnotesize  Indice dei prezzi  delle vettovaglie  dai registri
      degli {\it imaret}, 1490-1655. La  linea solida è in {\it akçe},
      quella tratteggiata in grammi d'argento.}
  \end{center}
}

\mode
<article>

Il grafico  è una rielaborazione  da \cite[p. 15]{barkan1975}.  I dati
dell'argento vengono rivisti al ribasso in \cite[p. 80]{pamuk2001}.

\mode
<all>

\def\coldue{\only<2->{\color{evidenzia}}}

\myframe[4]{Salari degli operai edili}{frame:salari}{
  \begin{center}
    \begin{tabular}{r*{4}{r@{.}l}}
      \hline
      \multicolumn{9}{c}{\bf Salari giornalieri}\\
      \hline
      &\multicolumn{4}{c}{non esperti} & \multicolumn{4}{c}{esperti}  \\
      anni & \multicolumn{2}{c}{akçe} & arg& (g)  
      & \multicolumn{2}{c}{akçe} & arg& (g) \\
      \hline
      1580-1589 &  8&1 & 3&5 & 12&4 & 5&4 \\
      \coldue 1590-1599 &\coldue  11&\coldue 7 &\coldue  2&\coldue 6 &\coldue  20&\coldue 7 &\coldue  4&\coldue 6 \\
      \color{black}1600-1609 & 13&9 & 4&0 & 22&5 & 6&5 \\
      \hline
    \end{tabular}

    \vspace{1cm}

    \visible<3->{\parbox[8]{0.6\textwidth}{Operaio esperto all'anno (su 200 giorni) nel periodo 1590-1599:}\hfill 4140 akçe}

    \vspace{1cm}

    \visible<4->{Con 8-12 akçe: 8 kg di pane o 2.5 kg di riso o 2 kg di montone} 

  \end{center}
} 

\mode
<article>

Il grafico viene da \cite[p. 301]{ozmucur2002}.

All'epoca di Süleyman, l'{\it  ağa} dei Giannizzeri aveva una pensione
di 300 akçe al giorno e un alto  ufficiale tra i 120 e i 150, oltre ad
alloggio,    parte    del    vitto    e    parte    dell'abbigliamento
\cite[p. 191]{horniker1944}.

In  ogni caso, la  paga di  un operaio  esperto gli  consentiva, anche
lavorando meno di  duecento giorni l'anno, di avere  un buon tenore di
vita \cite[p. 306-7]{ozmucur2002}.

\mode
<all>

\myframe{Prezzi di alcune merci}{frame:alcunemerci}{
  \small
  \begin{center}
    \begin{tabular}{>{\it}lcc}
      & prima del & dopo il \\
      &{\it Narh Defteri} 
      &{\it Narh Defteri}  \\
      & (in akçe) & (in akçe)  \\
      \hline
      100 {\it dirhem} di pane & 0.87 & 0.50  \\
      100 {\it dirhem} di pane di alta qualità & 1.25  & 0.83  \\
      1 {\it kile} di farina & 120 & 80  \\
      1 {\it kile} di farina di bassa qualità & 75 & 50  \\
      1 {\it kile} di riso & 56 & 39  \\
      1 {\it okka} di miele & 20 & 13 \\
      1 {\it okka} di burro & 26 & 19 \\
      1 cubito di velluto fine francese & 1200 & 550 \\
      1 cubito di velluto genovese & 880 & 400 \\
      1 cubito di lana pesante {\it nev-peyda} & 300 & 120 \\
      \hline
    \end{tabular}
  \end{center}

  \vspace{0.5cm}

  {\footnotesize
  \begin{tabular}{rcccl@{\hspace{1cm}}rcccl}
    100 & dirhem & $\cong$ & 300 & g &
    1 & kile   & $\cong$ & 36 & l \\
    1 & okka   & $\cong$ & 1.2 & kg &
               && $\cong$ & 10 & kg di farina \\ 
    1 & cubito & $\cong$ & 45 & cm &
               && $\cong$ & 15 & kg di riso \\ 
  \end{tabular}
  }
}

\mode
<article>

Da \cite[p. 13-14]{barkan1975}.

\mode
<all>

\myframe{Andamento dei salari}{frame:trend}{
  \begin{center}
  \input{frame/salari}
  \end{center}
}

\mode
<article>

Da  \cite[p.  306]{ozmucur2002}.   L'andamento dei  salari  continua a
calare e bisognerà aspettare il 1850 perché i salari tornino ai valori
del 1500.

A differenza di quanto succede  in occidente, dove è ancora largamente
diffuso il  baratto e  il pagamento in  natura (anche  delle imposte),
nell'impero  ottomano  l'uso  della   moneta  è  molto  diffuso  nella
popolazione urbana e in larghe  fasce di quella rurale. Quindi il calo
relativo  dei salari,  l'aumento dei  prezzi e  la  svalutazione della
moneta  hanno   un  impatto   enorme  virtualmente  su   tutti  quanti
\cite[p. 307-8]{ozmucur2002}.

Il problema è  se possibile maggiore per chi  ha uno stipendio fissato
per  legge  (soldati, impegati  dell'amministrazione,  ecc.). E'  però
anche  per  questo che  le  politiche  finanziare  come il  {\it  Narh
  Defteri} hanno successo.

\mode
<all>

\myframe[3]{Il prezzo della seta grezza a Bursa}{frame:prezzoseta}{
  \begin{center}
    \begin{tabular}{rr@{.}l}
      1595 & 197&06 \\
      \coldue 1597 & \coldue 224&\coldue 79 \\
      1603 & 351&05 \\
      1607 & 233&05 \\
    \end{tabular}

  \vspace{0.4cm}
  
  in {\it akçe} per {\it lodra}

  \vspace{1cm}
  \visible<3->{
    \begin{tabular}{rcccl}
      1 & cantara &$=$&176&lodra \\
      1 & cantara &$\rightleftarrows$&39\hspace{0.2em}563 &akçe\\[1cm]
      1 & cantara &$=$&100&lodra \\
      1 & cantara &$\rightleftarrows$&22\hspace{0.2em}479 &akçe\\
    \end{tabular}
  }
  \end{center}

}

\mode
<article>

Da \cite[p. 536]{cizakca1980}.

\mode
<all>

\myframe{Confronto}{frame:confronto}{
  \scriptsize
  \parbox[c]{0.4\textwidth}{\includegraphics[scale=0.15]{documenti/Bilancioimg-002.png}}
  \hfill\parbox[c]{0.55\textwidth}{
    \begin{center}
    \begin{tabular}{r>{\centering}p{1cm}ccp{1cm}}
      1 &cantara di Cipro &$\cong$& 750 &libbre sottili\\[2mm]
      1 &cantara di Bursa &$\cong$& 176 &libbre sottili\\[2mm]
      1 &cantara di Cipro &$\cong$& 4.26 &cantara di Bursa\\[4mm]
      85 & cantara di Cipro & $\cong$& 362 & cantara di Bursa\\[4mm]
      362 & cantara di Bursa di seta grezza & $\rightleftarrows$ 
      & 14\hspace{0.2em}321\hspace{0.2em}806 & akçe\\[2mm]
          &                                 & $\rightleftarrows$ 
      & 8\hspace{0.2em}137\hspace{0.2em}398 & akçe\\
    \end{tabular}

    \vspace{1cm}
    {\normalsize 444\hspace{0.2em}000 $\div$ 666\hspace{0.2em}000 akçe}
    \end{center}

  }
}

