\subsection{Documento 7 - Il bailo Girolamo Cappello al Doge (Costantinopoli, 22 marzo 1600)}

[Il documento è tutto cifrato]

\begin{center}
Ser.mo Prencipe
\end{center}

Carlo Cigala hà scritto da  Ragusi al Capitano suo fratello quello che
vostra   Ser.tà   intendera  dagli   acclusi   capitoli  mandati,   et
all'Ambasciatore di Francia, et à  me in molta confidenza, da quali si
scoprono li suoi fini e pensieri, et il sospetto che egli tiene di noi
onde l'Amb.re [...]

Si stà in aspettatione dell'arrivo di Simon giorgiano se bene doggi mi
è stato  riferito da buona  banda, che egli  non sia stato  per ancora
posto  in camino,  havendo Giafer  Bassà scritto  che per  la stagione
horida del verno, et  per la grave età di Simon che  è di 80 anni egli
ha differito la missione sua à  stagione più comoda se così sua maestà
comanderà.

[...]

Dalle Vigne di Pera a 22 di Marzo 1600

\begin{raggedright}
Girolamo Cappello Bailo
\end{raggedright}


