\subsection{Documento 22 - Il bailo Girolamo Cappello al Doge (Costantinopoli, 30 giugno 1600)}

\begin{center}
Ser.mo Prencipe
\end{center}

Giunse hieri mattina qua da Gallipoli i scrivano della nave Mamolo (?)
la quale cinque  giorni prima era arrivata alli ...,  et sopra essa ho
...   ritrovarsi  molte  robbe  di  ragione dell'  Ill.mo  Bailo  Nani
dell'elettione del quale non ho havuto a quest'hora nessuno particolar
avviso da  Vinetia, se non  per una di  Cattaro per i consigli  ... da
quel Ill.mo  Rettore, et  per una di  Candia dal Ill.mo  Rettore della
Canea; ...  che  posso dire di crederlo incamminato  in viaggio, prima
che siano  arrivate qui  le lettere della  sua elettione; et  come gli
resto con molto obbligo  della prontezza ch'egli dimostra di venirsene
tosto qua, consì m'incresce di  non vedere lettere sue, che m'avvisino
la  risoluzione   del  suo  viaggio  per  poterlo   servire  nei  suoi
particolari bisogni, et nel  procurargli altra comodità, quand'egli si
risolvesse  al  viaggio di  mare,  nel  qual  caso (quando  egli  così
desiderasse) mandarei  alcuno de Dragomanni  ccome nuovi comandamenti,
perch'egli fosse  condotto qua',  o' dove gli  tornasse comodo  con le
galee promessemi dal cap.no, come  per alter mie diedi conto a' vostra
ser.tà, et certo, che la ... delle lettere che con sommo desiderio sto
aspettando, essendo  di già  passati due mesi  dopo la  sua elettione,
m'apporta  grandissimo travaglio, per  il dubbio,  in ch'ella  mi pone
della cagione di questa lunga dimora, con la quale pare che la fortuna
vogli  anco impedire la  perfetta consolazione,  che sò  riceverei, et
dall'avviso  dell'elettione, et  dalla prontezza  di questo  sig.re di
venirsene  à  liberarmi di  qua.  Hò  ricevute  dal medesimo  scrivano
lettere  delli sig.ri  di  Corfù, dell'Ill.mo  Provveditore del  Iance
(??), et sig. Cap.no del  Golfo, per le quali son'avvisato dell'andata
di  esso Ill.mo  Cap.no  nelle acque  di  santa Maura,  e Prevesa  per
impedire i  passi à Deliali,  et altri corsari, retiratesi  per quello
che mi scrivono alla Prevesa:

[inizio cifrato]

Spero però che all'arrivo di Mustafà  di Santa Maura, il quale partì 8
giorni sono con Suliman di  Catanea haverà esso Ill.mo Cap.o occasione
di  esercitare  il suo  valore  perché Mustafà  se  ne  era con  animo
ressoluto di  fare il debito suo  et hauendo io mandato  seco un huomo
dal ... con lettere a quel Ill.mo Provv.re et a li sig.ri di Corfù, et
cap.no in Golfo  con l'avviso dell'andata di esso  Mustafà, et insieme
con  li  comm...  ...   il  Bei alla  Morea,  spero,  che quei  sig.ri
intendendosi con Mustafà sudetto potranno in questo [...]

Gli parlai con il C??? di Giafer Bassà per il reg.o delli .. gruppi(?)
di denari  delli ...  mercanti, da lui  levati(?) mentre era  Bassà di
Cipro, il qual C??? mi promette di far ogni buon officio, et opera, ma
non me  ... a quest'ora data  nessuna risposta dubito  che la risposta
sarà  molto difficile, come  difficile è  il levar  danari di  mano di
queste genti, et ... credo che ... vederne il fine, bisognerà prendere
... più risoluti ...

[fine cifrato]

Dalle ... di Pera a 30 di giugno 1600

\begin{raggedright}
Girolamo Cappello Bailo
\end{raggedright}

