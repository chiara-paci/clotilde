\section{Fusta 52}

\subsecdocument{Il bailo Girolamo Cappello al Doge}{Costantinopoli}{?? settembre 1600}

[..]

\startcifrato

[..]

Simon  Georgiano si  trova a  Scutaretto gi`a  alcuni giorni,  et sono
state varie le  opinioni intorno à casi suoi,  ma finalmente si dubita
che  l'habbino à  mandare à  ??? torri  del Mar  Negro, ò  nelle sette
torri,  facendone la condanna?  senza nessun  concorso del  popolo, mà
tacitamente,  si  bene s'è  sempre  veduto,  che  egli dovesse  essere
condotto pubblicamente in Divano

[...]

\stopcifrato

[...]

Dalle Vigne di Pera a 22 di Marzo 1600

\testoright{Girolamo Cappello Bailo}

\subsubsecallegato{Messaggio cifrato ricevuto a Costantinopoli in 5 settembre}{Ragusa}{22 luglio 1600}

\subsubsecallegato{Al Beglerbei et al Defterdaro di Cipro}{Costantinopoli}{Metà della luna di Sefer l'anno 1008}

\begin{center}
Al Beglerbei et al Defterdaro di Cipro
\end{center}

Al giunger  dell'ecc.so segno Imp.le  saprete, come il Bailo  di Ven.a
??? mandato alla mia fe. Porta  mi ha fatto intendere, che la Saetia?,
over Nave de  Zuan Ant.o? Vidali? Ven.no venendo  da Tripoli di Soria,
carica  di mercantie  per Ven.a,  mentre seguitava  il suo  camino per
Ven.a s'incontrò  in un Galeone  Inglese, del quale havendo  havuto il
patrone di essa  Nave paura, lasciò doi marinari in  Nave, et esso con
tutto il resto  delli suoi compagni entrato nella  barca, se ne fuggì,
et dopo lì in Nave restati i sudetti doi Marinari, condussero quella à
sakvam.to nell'Isola di Cipro alla scalla di Limossa?  onde il Console
Venetiano,  ch'è in  quel luogo  in virtù  della buona  pace,  e degli
ecc.si?  Cap.li ricerc`o da  Dilanei?  Bei Luogotenente del Beglierbei
di Cipro licenza di mandare essa  Nave con tutto il carico à ven.a, ma
egli oltre che  non gli l'ha coluta concedere,  fece di più pigliargli
le velle, et trattenere essa Nave,  dicendo io non vi darò la licenza,
se prima non mi verrà  commesso con nobil comm.ne?  dall'ecc.sa Porta,
che ve la dia, et in  questo modo hà intertenuta?, et fatto tardare la
sudetta Nave. Ha  però supplicato il mio Imp.le  Comm.to, che la detta
Nave insieme  con tutta  la mnercantia,  che è in  essa sia  mandata à
Venetia.  Però vedrete  se  in effetto  è  certo, che  à tal  vassello
venetiano no sia stata data  licenza, et che s'è stata intertenuta, et
sequestrata, non  è ciò conveniente, né lecito,  Pertanto comando, che
senza  alcuna dimora,  et senza  differire punto  mandiate  il sudetto
vassello con tutta la mercantia,  velle, Armizi, et robbe, che in essa
?? è  a Vinetia,  et vi guarderete  molto bene  di far cosa  contra la
buona pace,  et gli ecc.si cap.li  et che sia più  bisogno di rimandar
altro  ordine  per  tal   causa.  Così  sapete,  et  prestarete?  fede
all'honorato Imperial segno.

Di Cost.li alla mettà della Luna di Sefer l'anno 1008.

\testoright{tradotto dal Borisi}

\subsubsecallegato{Al Beglerbei et al Defterdaro di Cipro}{Costantinopoli}{primi di settembre 1600}

\begin{center}
Al Beglerbei et al Defterdaro di Cipro
\end{center}

Al giunger dell'ecc.so segno Imperiale  saprete, che il Bailo di Ven.a
??? mandato alla mia felice Porta  mi ha fatto intendere, che in virtù
de  gli  ecc.si cap.li,  essendo  soliti  li  mercanti Ven.i  che  ivi
negotiano, et quelli che vengono  in quelle parti con le Navi comprare
à loro beneplactio, et pigliare à pretij correnti li gottoni, et altre
sorti di  mercantie, et  non ostante  che anco già  sia stato  con mio
honorato Reggio  Comm.to? probito, che  le li homini?  Appaltatori, né
altri  dovessero far  pigliare per  forza alla  natione  venetiana, ne
dargli à maggior pretio de  pretio correnti li gottoni: hora alcuni di
quei ministri contra gli ecc.si capitoli, et honorato mio ordine fanno
per forza  comprar dal  console, et mercanti  Venetiani li  gottoni de
sig.ria? à  maggior pretio  del pretio corrente:  onde in  cio havendo
richiesto il mio Imp.le Comm.to accioche sia ovviato?, et non lasciato
far più  siml torto  alla detta natione,  commando che giunto,  che vi
sarà  questo mio  Reggio, et  Imperial Comm.to  dobbiate vedere  se in
effetto essendo soliti  li mercanti che habitano in  Cipro, et quelli,
che vengono con le navi pigliare, et comprare à loro beneplacito, et à
pretio  corrente   li  gottoni,  et  altre   mercantie  dalli  communi
venditori, et  che hora  alcuni ministri per  forza gettando  sopra li
gottoni, fanno chel'Console et altri mercanti Veneziani li piglini, et
comprino à maggior  pretio del pretio corrente, non  è cosa honesta né
lecita. Però voi efficacem.te prohibirete à quei tali, che non debbino
per l'avvenire  gettare sopra il  Console, et altri mercanti  Ven.i li
gottoni de ???, ne altra mercantia,  ne lasciarete che li siano dati à
maggior pretio del pretio corrente, et così non lasciarete, che contra
gli ecc.si cap.li, et buona pace,  et già, et hora concesso mio Imp.le
Comm.to le  sia data per questa,  ne per altra  causa minima molestia,
così   saprete,  et  dopo   visto  questo   mio  Imperial   segno,  lo
restituirete.

Scritto in Cost.li alle 2 della Luna di Rebialevel? l'anno 1009.

cioè circa li primi di 7mbre 1600

\testoright{tradotto dal Borisi}

\subsubsecallegato{Al Beglerbei et Cadi di Cipro}{Costantinopoli}{circa 10 settembre 1600}

\begin{center}
Al Beglerbei et Cadi di Cipro
\end{center}

\docnota{Va trascritto}

\subsubsecallegato{Al Beglerbei, Defterdaro et Cadi di Cipro}{Costantinopoli}{2 della luna di Rabilevel 1009}

\begin{center}
Al Beglerbei et Cadi di Cipro
\end{center}

\docnota{Va trascritto}


\setcounter{docnumber}{3}

\subsecdocument{I baili Girolami Cappello e Agostino Nani  al Doge}{Costantinopoli}{30 settembre 1600}

[...]

E' stato finalmente condotto  qua Simon Giorgiano, che fu accompagnato
da 20 Chiaus con 200 soldati  di quelli che sono venuti con il Cheara?
di Giafer  Bassà che fu  quello, che lo  fece preggione. Fu  il povero
Principe  condotto in  Divano  à cavallo,  et  nell'entrar alli  Bassà
levarono tutti in piedi honorandolo, et facendolo sedere in scabelo al
dirimpetti di essi.  Il Vezir lo consol`o con  parole cortesi, et dopo
haver parlato  seco lo  mandarono nelle stesse  torri con  alcuni suoi
huomeni  che lo  servino dove  si crede  che egli  finirà la  sua vita
essendo  vecchio di  80 anni,  se  bene robusto,  et nell'aspetto  non
dimostra  l'età tingendosi  la  barba; è  venuto  seco un  figliuolino
nipote  suo nominato  David insieme  con alcuni  altri mandato  da ???
figlio di Simon  per procurare di liberar il  padre con questi ostaggi
promettendo fideltà at di mandare al gran sig.re tributo di sette anni
corti?, che dicesi essere in tutto 700 somme di seta, mà pare, che quì
non diano orechie alle instanze loro, se bene sono in conformità della
parola data da Giafer Bassà, et  si teme che possano insieme con Simon
ritenereanco questo  figliuolino con gli  altri, che sono con  lui per
assicurarsi con questi mezzi maggiormente da qualche moto, che potesse
fare il figliuolo di Simon favorito dal Persiano.

[...]

La regolazione delle monete, che pareva fosse differita per l'instanza
fatta da  Ibraim Genero? come io  Capello scrissi per  le precedenti è
stata ??mente pubblicata,  et con l'uscita dei nuovi  Aspri fù ridotto
il cechino  da 220 a  120, et  il talero da  140 ad ottanta  Aspri, la
quale regolation sin hora non ha apportato altro beneficio, se non che
non  corrono Aspri  cosi cativi  come prima,  ma nello  resto  torna à
pregiudicio de poveri  perche le robbe, et le  vettovagli se ne stanno
quasi  nell'istessa condizione  di  prima anzi  in maggior  carrestia?
perché parlandosi in  ragione d'Aspri non è diminuito  il prezzo delle
robbe,  con  tutto che  l'Aspro  vagli hora  quasi  il  dopio più  del
passato,  è vero  che  Gemisei? Assan  Bassà  (al quale  è stato  dato
particolar carico  dal gran sig.re di questa  regolatione con assoluta
potestà  di castigar  nella vita  li trasgressori,  come fece  anco il
medesimo giorno della pubblicazione con  la morte fatta dare à due che
cambiavano monete)  stà procurando  in ogni maniera  di dare  prezzo à
tutte le  robbe di ogni  condizione per appareggiarle alla  valuta del
cecchino, mà non sarà cosi facile il farlo, et mantenire il decretato,
et  tanto più  quanto  egli vuole  abbracciare indifferentemente  ogni
condizione di  robba non solo delle  vettovaglie, ma di  qual si vogli
merce, di maniera che egli  dissegnava anco dar prezzo alla panina, et
à pani di seta de nostri  mercanti, il che vedendo? à gran pregiudicio
del  negotio Noi  nell'audienza del  Bassà facessimo  instanzia perche
fosse  sospeso l'ordine  fino  à tanto  che  io Nani  andarò à  dargli
particolar informatione del pregiudicio  de mercanti, come farò quanto
prima.  L'istesso ufficio faressimo  con Gemisei  Assan Bassà  che n'è
l'autore, et l'uno,  et l'altro cortesemente promissero di  non ne far
altro, ma di aspettare d'intendere  gli aggravij de mercanti, li quali
altro  volte sotto Sinan  Bassà furono  liberati da  simile monatione?
introdotta da lui che alla fine se ne chiarì, et non hebbe effetto, et
io non mancarò con ogni  poter mio di procurare l'annulatio di decreto
tanto pregiudiciale, come lo istesso farò in tutto lo resto di negotij
di vostra Ser.tà

[...]

Dalle Vigne di Pera il di ultimo di Settembre 1600

\testoright{Girolamo Cappello Bailo}


\newpage

% Senato, dispacci costantinopoli

Fusta 52

n. 14 Agostino  Nani al  Doge,  dalle  Vigne  di Pera,  1  novembre
1600. cc. 83-100

{alla fine di settembre Saban Pascià di Cipro si trova nel Danubio con
  le  galee  alla  obbedienza  di  Mahumet?  generale  contro  Micali}

[95v-96v] all. al n. 14

n.  22  Agostino  Nani al  Doge,  dalle  Vigne  di Pera,  24  novembre
1600. 171-175 cifr., 167-170 dec.

[169r] [..] Di  Mar Negro sono ritornati con  quattro galee Memi Corso
Giafer Bassà fu già Bassà di  Cipro, et il Bei di Scio essendo restato
Saban con sei altre, et Giafer havendo incontrato uno degli Dragomanni
di casa, gli ha detto,  che dovesse dirmi come ringraziava grandemente
V.stra  Ser.tà  del  favore  che  le aveva  fatto  col  coagiuviar  la
ricuperatione del suo credito dalli  heredi del già Giac.o dei Biasij,
che fù viceconsule in Cipro,  come io li giorni precedenti per lettere
ricevute dalli Ill.mi S.ri Cinque Savii sopra la mercanzia feci sapere
al  suo cheraià.  Dice  il p.to  Giafer che  un giorno  sperava essere
ancora lui  cap.o dell'Armata  di Sua Maestà,  et che mentre  esso, et
Memi  corso  che  si  trovava   presente  saranno  di  un  volere  non
lascieranno mai operar alcuna cosa  al Cigala in pregiudizio di V.stra
Ser.tà anzi lo impediranno, con aggiungere, che cosa hora faceva Carlo
Cigala à Scio, et per dir il vero con tutto che li turchi tenghino che
il  Cap. Cigala  sia più  vero  Turco di  ogni altro,  ?renze di  meno
dicono, che li suoi parenti cristiani lo fanno riputar mal Musulmano.

n.  33 Agostino  Nani al  Doge,  Dalle Vigne  di Pera  6 gennaio  1600
(mv=1601), cc.  296-299

[297v] [..] Essendomi incontrato con Giafer che fu Bassà di Cipro egli
mi hà  con affetto ricercato ad  assicurar la Ser.tà  V.tra, che tiene
una buona volontà  verso di Lei, et dove  conoscerà poter giovare alli
suoi interessi  lo farà con molta  prontezza, et io oltra  di ciò sono
maggiormente  tenuto  à renderle  questo  testimonio  poiche egli  con
estraordinaria cortesia mi  ha donato un suddito di  quel S.mo Dominio
che teneva  per suo  volontario debitore di  cechini cento  vinti, che
haveva à lui, et ad altri  esborsati per conto suo valendosi della sua
persona  per  far  lavorar  di favro\footnote{Fabbro,  ossia  operaio,
  artigiano.},  et è  figlio di  maestro Zaneto  favro  di cappelaria,
costui per la  sud.a somma eragli obbligato come  schiavo fino che gli
restituisse  esso denaro onde  per virtù  di capitolazioni  non potevo
ricercarglielo in  conto alcuno, della cui dispotione  non debbo dirle
altro se non,  che se la fortuna di esso Giafer  lo portasse al carico
di Generale  dell'Armata in luoco di  chi serve al  presente la Ser.tà
v.stra farebbe un grande, et fruttuoso cambio.

n. 36  Agostino Nani ai  Cinque Savi, dalle  Vigne di Pera  18 gennaio
1600 (mv=1601), cc. 322-323 in parte cifrato senza decifratura.

[cifrato]

[322v] [cifrato] Con  questa occasione non debbo restar  di dirle, che
essendo stato scritto da Ven.a da un particolar amico al S.r Britio(?)
Giustiniano  Console delle Smirne  [correzzione aggiunta]  in Siria(?)
per quanto  mi viene riferito,  che da non  sò chi costà non  [? forse
  cancellato] procurato  senza altra sua saputa di  levargli il carico
per esserne in esso investito,  in tutto che egli sia stato confirmato
in  detto Consulato  in vita  per parte  dell'Ill.mo Senato,  et senza
alcun suo  demerito, se  ne risente molto,  non sapendo ciò  venghi da
alcun  suo mancamento.   Dove io  non  ho voluto  credere simil  cosa,
perche sono  sicuro, che VV.SS.  Ill.me si  haverebbero compiaciuto di
haverne informatione  dall'Ill.mi miei predecessori, et  da me ancora,
poiche  io mi  trovo qui  al presente.  Et havendo  egli  prestato, et
prestando  tuttavia utiliss.o  et honorato  servitio, pregole  à darmi
parte  della  [sua]  volontà,  accioche  non  essendo  vero  io  possi
consolarlo; et se fosse io ne resti illuminato di quella per poter con
più recente informatione  darle notitia di quanto le  VV.  SS.  Ill.me
desiderassero.  Et loro S.rie Ill.me baccio di core le mani.


n.  37 Agostino  Nani al  Doge, dalle  Vigne di  Pera 20  gennaio 1600
(mv=1601), cc.  324-335

[325v] [..]  Hò dato conto alli S.ri Cinque Savij alla mercantia haver
dissuaso il  S.r Giafer che  fù Bassà di  Cipro di mandar un  Chiaus à
Venetia  per il  suo credito  con Marco  dei Aldi  procurando  anco di
interessarne  V.stra  Serenità,  et  holo indotto  à  raccomandare  il
negotio à  qualche commesso  costì come hà  fatto ad uno  agente delli
Abudenti hebrei che se ne hanno  preso il carico. Onde così ho stimato
bene  divertire  il  suo  primo  disegno,  così  credo  che  sarà  con
sodisfazione

[326r] di  lei alla quale anco  scrive una sua  in raccomandazione del
prefatto suo  procuratore acciò venghi quanto più  si possi occorrendo
dalla publ.ca aut.tà fovorito  per maggiormente facilitar la essazione
del denaro et consignazione delle robbe, et certo che ogni favore, che
sarà  dimostrato di  quelo modo  che  parerà alla  Somma prudentia  di
V. S.tà verso la persona di esso S. Giaffer sarà ottimamente impiegata
in soggetto di molte qualità, et essistimatore [incerto] delle cose di
mare, et che si mostra benissimo disposto in quella S.ma Rep.ca.

n. 45  Agostino Nani al  Doge, dalle Vigne  di Pera, 20  febbraio 1600
(mv=1601), cc. 414-424

[416r] [..] Il Console di Cipro mi scrive delle

[416v]  difficoltà  che  ha  incontrato  [..] et  delle  mangiarie  et
estorsioni,  che  da  molti  di  quelli ministri  vengono  fatte  alli
Mercanti Venetiani,  et che  particolarmente sforzavano li  patroni de
Vasselli à  caricare le mercantie  de Turchi sopra le  Navi Venetiane,
per il che  quelle de Suditi convenivano restar in  terra, si come era
successo  per la  violenza usata  da  Mustafà Bassà  che haveva  fatto
metter  cento  cinquanta sachi  di  gotoni  sopra  la nave  Ferra,  et
cinquecento sopra la  Pigna, et che il medesimo  voleva levar tutti li
gotoni per suo conto, et prohibire così alli mercanti di Venetia, come
Cipriotti, et fino  alli Marinai di caricar gotoni  se non se pagavano
doi cechini per cantaro.
