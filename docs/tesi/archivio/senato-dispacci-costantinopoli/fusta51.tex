\section{Fusta 51}

foto: 7, 8, 11, 12, 13, 22, 25

\setcounter{docnumber}{6}

\subsecdocument{Il bailo Girolamo Cappello al Doge}{Costantinopoli}{22 marzo 1600}

\docnota{Il documento è tutto cifrato}

\begin{center}
Ser.mo Prencipe
\end{center}

Carlo Cigala hà scritto da  Ragusi al Capitano suo fratello quello che
vostra   Ser.tà   intendera  dagli   acclusi   capitoli  mandati,   et
all'Ambasciatore di Francia, et à  me in molta confidenza, da quali si
scoprono li suoi fini e pensieri, et il sospetto che egli tiene di noi
onde l'Amb.re [...]

Si stà in aspettatione dell'arrivo di Simon giorgiano se bene doggi mi
è stato  riferito da buona  banda, che egli  non sia stato  per ancora
posto  in camino,  havendo Giafer  Bassà scritto  che per  la stagione
horida del verno, et  per la grave età di Simon che  è di 80 anni egli
ha differito la missione sua à  stagione più comoda se così sua maestà
comanderà.

[...]

Dalle Vigne di Pera a 22 di Marzo 1600

\testoright{Girolamo Cappello Bailo}

\setcounter{docnumber}{10}
\subsecdocument{Il bailo Girolamo Cappello al Doge}{Costantinopoli}{6 aprile 1600}

\docnota{Il documento è tutto cifrato}

\begin{center}
Ser.mo Prencipe
\end{center}

[...]

ho parlato con  il C?? di Giaffer Bassà  in proposito dell'eredità dei
?aguri(??), et ho posto seco ordine di ??? ?? Giaffer passato il B???,
nel qual negotio procederei in  ogni celerità, et in maniera, per quel
Bassà si resterà molto soddisfatto di me. [...]

Dalle Vigne di Pera a 6 di Aprile 1600

\testoright{Girolamo Cappello Bailo}

\setcounter{docnumber}{21}

\subsecdocument{Il bailo Girolamo Cappello al Doge}{Costantinopoli}{30 giugno 1600}

\begin{center}
Ser.mo Prencipe
\end{center}

Giunse hieri mattina qua da Gallipoli i scrivano della nave Mamolo (?)
la quale cinque  giorni prima era arrivata alli ...,  et sopra essa ho
...   ritrovarsi  molte  robbe  di  ragione dell'  Ill.mo  Bailo  Nani
dell'elettione del quale non ho havuto a quest'hora nessuno particolar
avviso da  Vinetia, se non  per una di  Cattaro per i consigli  ... da
quel Ill.mo  Rettore, et  per una di  Candia dal Ill.mo  Rettore della
Canea; ...  che  posso dire di crederlo incamminato  in viaggio, prima
che siano  arrivate qui  le lettere della  sua elettione; et  come gli
resto con molto obbligo  della prontezza ch'egli dimostra di venirsene
tosto qua, consì m'incresce di  non vedere lettere sue, che m'avvisino
la  risoluzione   del  suo  viaggio  per  poterlo   servire  nei  suoi
particolari bisogni, et nel  procurargli altra comodità, quand'egli si
risolvesse  al  viaggio di  mare,  nel  qual  caso (quando  egli  così
desiderasse) mandarei  alcuno de Dragomanni  ccome nuovi comandamenti,
perch'egli fosse  condotto qua',  o' dove gli  tornasse comodo  con le
galee promessemi dal cap.no, come  per alter mie diedi conto a' vostra
ser.tà, et certo, che la ... delle lettere che con sommo desiderio sto
aspettando, essendo  di già  passati due mesi  dopo la  sua elettione,
m'apporta  grandissimo travaglio, per  il dubbio,  in ch'ella  mi pone
della cagione di questa lunga dimora, con la quale pare che la fortuna
vogli  anco impedire la  perfetta consolazione,  che sò  riceverei, et
dall'avviso  dell'elettione, et  dalla prontezza  di questo  sig.re di
venirsene  à  liberarmi di  qua.  Hò  ricevute  dal medesimo  scrivano
lettere  delli sig.ri  di  Corfù, dell'Ill.mo  Provveditore del  Iance
(??), et sig. Cap.no del  Golfo, per le quali son'avvisato dell'andata
di  esso Ill.mo  Cap.no  nelle acque  di  santa Maura,  e Prevesa  per
impedire i  passi à Deliali,  et altri corsari, retiratesi  per quello
che mi scrivono alla Prevesa:

\startcifrato

Spero però che all'arrivo di Mustafà  di Santa Maura, il quale partì 8
giorni sono con Suliman di  Catanea haverà esso Ill.mo Cap.o occasione
di  esercitare  il suo  valore  perché Mustafà  se  ne  era con  animo
ressoluto di  fare il debito suo  et hauendo io mandato  seco un huomo
dal ... con lettere a quel Ill.mo Provv.re et a li sig.ri di Corfù, et
cap.no in Golfo  con l'avviso dell'andata di esso  Mustafà, et insieme
con  li  comm...  ...   il  Bei alla  Morea,  spero,  che quei  sig.ri
intendendosi con Mustafà sudetto potranno in questo [...]

Gli parlai con  il C??? di Giafer Bassà per il  reg.o delli dui gruppi
di  denari delli  ...  mercanti,  da lui  levati mentre  era  Bassà di
Cipro, il qual C??? mi promette di far ogni buon officio, et opera, ma
non me ...ndo a quest'ora data nessuna risposta dubito che la risposta
sarà  molto difficile, come  difficile è  il levar  danari di  mano di
queste  genti,  et però  credo  che  ...  vederne il  fine,  bisognerà
prendere ... più risoluti ...

\stopcifrato

Dalle Vigne di Pera a 30 di giugno 1600

\testoright{Girolamo Cappello Bailo}

\setcounter{docnumber}{28}

\subsecdocument{Il bailo Girolamo Cappello al Doge}{Costantinopoli}{1 luglio 1600}

\tuttocifrato

\begin{center}
Ser.mo Prencipe
\end{center}

[...]

Hanno medesimamente destinati per Mar  Negro Giaffer fu Bassà di Cypro
et Memy(?)  .. Bassà di  Tripoli, li quali  haveranno seco in  ... 14?
galee  per servirsene  in diversi  affari per  l'effetto  suddetto; Ma
perché  dei progressi  di Michali  si  ragiona variamente  io mando  a
V. Sig.ria  un sommario di alcuni  avvisi venuti à  persone di qualche
credenza.

[...]

\stopcifrato

Dalle Vigne di Pera à primo di Luglio 1600

\testoright{Girolamo Cappello Bailo}

\setcounter{docnumber}{34}

\subsecdocument{Il bailo Girolamo Cappello al Doge}{Costantinopoli}{1 luglio 1600}

\tuttocifrato

\begin{center}
Ser.mo Prencipe
\end{center}

[...]

Gli Abudanti Hebrei che desiderano di venire a capitare in Vinegia, mi
hanno  presentato l'accluso  loro memoriale,  di quanto  supplicano la
S.ra V. di concedergli, perché  se ne possino venire sicuramente; però
ella prudentemente  si risolverà à quello che  stimerà magior servitio
suo.

\stopcifrato

Dalle Vigne di Pera à 15 di Luglio 1600

\testoright{Girolamo Cappello Bailo}

\subsubsecallegato{Lettera del console francese in Egitto}{}{5 giugno 1600}

Copia  di un  capitolo contenuto  in lettere  di Mons.  di Coquerel(?)
console per  sua Maestà Christianissima in  Egitto, scritta all'Ill.mo
S. di  Breves Amb.re di detta Maestà  à questa porta alli  5 di Giugno
1600.

[...]

\subsubsecallegato{Lettera degli Abudenti al Bailo}{Costantinopoli}{}

\begin{center}
Ill.mo sig. Bailo
\end{center}

Desideriamo noi juda  e david Abudente frati Ritirarsi  per V.a con le
famiglie et facultamente per Vivere et Morire sotto le Ale dela Ser.ma
Republica, Mentre pero potiamo  starsicuri de non esser molestatti per
la  Inquisitione o  da  altro per  le  cose passatte;  pero pregamo  a
V.S.Ill.ma  che voglia suplicar  in nome  nostro a  sua ser.ta  che si
degni consedersi, salvo  condoto ne la qui soto  scrita man.?? che noi
frati potiamo vivere et habittare per  V.a et in tutte le altre terre,
et loche  de sua Ser.sa  con le nostre  famiglie et faculta,  et esser
sicuri cosi in tempo di paxe come de guera de qualsivoglia molestia et
travaglio come sono tutti li  altri suditti di sua Ser.ta et l'istesso
si intenda in la faculta de il ??? nostro patto che Reta incost??.

Che  per ocasione  de inquisitione  de  cose passate  cossi de  giosa?
lopez? nostro frate di Roma  et altrove come per altri acidenti ocorse
nele persone  nostre per  ocasione de Religione  et per  esser Vinutti
come Cristiani per Ven.a o altrove non potiamo Ricever molestia alcuna
ma esser liberi et asolutti in  ogni tempo, et che sua Ser.ta si degni
protegierci per  questi Rispetti et  che non potiamo per  ocasione del
sudetto lomez nostro frate  esser astretti a qualsivoglia Istantia che
fusse fatta contra di noi da chi si sia.

Che sua Ser.ta si degni comandar che sia datta casa nel ghetto pagando
il nostro fito come gli altri hebrei et che non potiano esser astretti
a nisuna graveza deli hebrei di ghetto se non per quello che paresse a
sua serenitta, ala quale si remetemo,  et che habiamo da goder tuti li
previlegi che godeno li hebrei levantini e ponentini.

Che  il salvo  condotto se  intenda almenoper  Anni dieci  continuj et
tanto ??? piaxera  a Sua Ser.ta desiderando noi di  viver sotto le Ale
sue, ma se per qualche acidente  paresse a sua Ser.ta che noi ne altri
hebrei  se  fermassero in  V.a  in questo  caso  si  degni sua  Ser.ta
consedersi salvo condotto per Anni tre ascio potiamo acomodare le cose
nostre  et che  siamo  poi fatte  condure  con le  nostre famiglie  et
faculta in levante sicuramente, o con galie o altre vaseli sicuri.

Questo è quanto suplicamo che si sia dechiaritto nel salvo condotto et
previlegio che si degni sua ser.ta de consedersi il qualle subitto che
haveremo  havuto  dela  Benignitta  et  gratia sua  si  ritiraremo  ad
habitare nel inclitta citta di ven.a con le nostre famiglie et faculta
per goder di quella securta et  liberta che godeno li altri suditti et
citadini de quela Ser.ma Republica, et a V.S. Ill.ma Basciamo le mane.

\testoright{de V.S.Ill.ma}

\testoright{aff.mi s.ri Juda et david Abudente}


\setcounter{docnumber}{41}

\subsecdocument{Il bailo Girolamo Cappello al Doge}{Costantinopoli}{?? agosto 1600}


\begin{center}
Ser.mo Prencipe
\end{center}

[...]

\startcifrato

[...]

\stopcifrato

[...]

\startcifrato

E'  stato da  Ibraim generale  dato il  Bassalich de  Cipro  à Mustafà
Flangini? cugino di  Osman Capigì Bassà, con questo  duenqu havendo io
molta familiarità  mentre Ibrai  si trovava quì  et con il  suo mezzo?
?cevuti dal mned.o Bassà molti favori, egli nel suo passar dal Campo à
Cipro, mi  ha scritto di  ???, nella quale  crderà la Ser.ta  V.ra non
solo la prontezza,  che egli dimostra nel servizio  di V.ra Ser.tà, et
di quei  mercanti in Cipro: ma  quello che egli mi  acena ancora della
buona volontà d'Igraim, et della memoria, che egli conserva di quello,
che io trattai  ??, se bene me ne parla alla  sfuggita, di maniera che
bisogna, che il  Bassà gli habbia communicato il  suo pensiero, anzi è
da credere,  che cosi sia poiche egli  fu uno delli mandati  da lui al
Campo Imperiale  per la trattativa di  pace, et credo anco  che ne sia
consapevole  Osman Flangini  Capigì Bassà,  il quale  mi mandò  à dire
ultimemente, che voleva trovarsi meco, quanto prima havessi potuto per
parlarmi  di  negotij  commessigli   dal  suo  patrone,  et  io  starò
attendendo  la  sua  venuta,  la  quale  se  ritardasse  procurerò  di
sollecitare, et dargliene  occasione. A Mustafa Bassa di  Cipro poi ho
risposto  con  quei  termini  di  cortesi  che si  conviene  e  gli  ò
raccomandati  li  mercanti  di  Cipro,  che  sono  sino  à  quest'hora
grandemente tiraneggiati  da gli agenti  di Saban Bassà, come  da quel
Consule  ne vengo  avvisato, et  per quest'effetto  anco  ricercato ad
ottenere diversi commandamenti, che procurerò inviargli quanto prima.

[...]

\stopcifrato

Dalle Vigne di Pera à ?? di Agosto 1600

\testoright{Girolamo Cappello Bailo}

\subsubsecallegato{Lettera di Mustafà Bassà di Cipro al Bailo}{}{agosto 1600}

\begin{center}
Traduzione di lettera arrivata all'Ill.mo G. Bailo Capello da Mustafà Bassà di Cipro

All'Honoratissimo sig.r Bailo
\end{center}

Dopo le  honorevoli, et amichevoli  salutazioni si fà sapere  come per
gratia di Dio stò bene, et  che l'Ill.mo mio signore in ogni occasione
dimostra  grata memoria  delli sig.ri  di Venetia,  et di  V.S. ancora
commemorando  ben  spesso la  buona,  et  sincera  amicitia, che  quei
signori conservano con il ser.mo Gran Sig.re.

\startcifrato

[qualcosa che  non c'entra -  o ce l'hanno  infilato, o Mustafà  è una
  spia dei Veneziani]

\stopcifrato

Di me adunque non si scordarà  V.S. quando Io?  sarò in Cipri, perche,
et per lei,  et per gli amici suoi Io mi  adoperarò ne' servitij tutti
che occorreranno,  et massime in  quelli de' mercanti, per  li negotij
de'  quali mi scriverete  con le  Navi che  verranno in  quelle parti,
perche Io mi impiegarò con  quanto forze avrò facendoli rispettare con
le facoltà loro. Hò mandato Pialì Bei nostro insieme con Mehmet Chiaus
in quelle  parti, et però Io  prego V.S. che voglia  esser contenta di
farmi havere  per diversi miei  negotij costì dalli suoi  mercanti dui
ovvero tre  mille cecchini, perche  detti mercanti non diranno  di non
alla parola sua; et il guadagno loro quanto sarà da Vicenzo Negroponte
mio commesso  in Venetia al qualo  hò dato lettere,  et chiarezze sarà
pagato intieramente. Io spero, che  il suo aiuto mi sarà prestato come
si deve. Ho dato simile sigillo à quello ch'è nella lettera di V.S. in
carta  bianca  nelle  mani di  Mehmet  Chiaus,  di  Piali Beg,  et  di
M. Zuanne acciò che faccino lettere  di cambio di contro à quelli, da'
quali si  prenderanno li denari,  et V.S. ancora potrà  affermare come
questo  bollo è  il mio,  perché crederanno  all'attestazione  sua. In
conclusione è servitio d'amico, et io hò bisogno del suo aiuto. Inanzi
che hora  l'Ill.mo mio Patrone havea  ordinato in Venetia  per lui, et
per li miei  Casali, alcuni palli di ferro, zappe,  et ferri da carro,
li  quali fin'hora  non sono  venuti, et  però piacendo  à  Dio quando
V.S.  sarà gionta con  saluti à  Venetia potrà  aiutar il  negotio, et
favorire    mio   Cognato    acciò   che    possa    mandarmi   questi
strumenti. ... che  nelli nostri paesi si trovano  di questi ferri; ma
quelli de Vinetia sono troppo buoni,  ne si può far senz'essi. et così
saprà V.S. la quale mi potrà  far tener fornito di questi ferri d'anno
in anno, acciò  non mi manchino, perché in effetti  li casali di Cipro
sono molto bisognosi delli suddetti strumenti di Venetia, et perciò mi
è necessaria la sua gratia. V.S. mi scriverà in ogni occasione, perché
se in Cipri  le bisognerà alcuna cosa Io  la sodisfarò compitamente et
ella intenderà il rimanente della bocca di Pialì Beg. .. vostro

il Povero? Mustafà Bassà di Cipro
 
