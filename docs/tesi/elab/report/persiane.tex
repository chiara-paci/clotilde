\begin{glossario}{Origine: persiano}
\item[{\color{colorlowref}\spzrl{:CsitAn.H}},] {\sf asitane},\ n.\ p.:\ soglia,   porta, corte; Istanbul; nel testo compare una  volta   seguito  da  \RL{sa`Adat}   e  una  volta   seguito  da \RL{dawlat}.
\begin{subvocedue}
\item[Pron. (1.0):] \spzcite{redhouse1997}
\item[Rif.:] \spzcite[31]{kiefferbianchi18351}
\end{subvocedue}
\begin{subvocedue}
\item[\subglossariobullet] \spzrl{:CsitAn^h sa`Adat||a^syAn.H}, {\sf Asitane-i  Saadetaşiyane},\ np.\ p.:\ la  Porta della         Felicità,        la         residenza        imperiale (v. Mantran).
\begin{subvocedue}
\item[Pron. (1.0):] \spzcite{redhouse1997}
\item[Rif.:] \spzcite[31]{kiefferbianchi18351}
\end{subvocedue}
\item[\subglossariobullet] \spzrl{:CsitAn^h      sa`Adat}, {\sf Asitane-i Saadet},\ np.\ p.:\ stesso significato.
\begin{subvocedue}
\item[Pron. (1.0):] \spzcite{redhouse1997}
\item[Rif.:] \spzcite[31]{kiefferbianchi18351}
\end{subvocedue}
\item[(simil:1.0)] \spzrl{:AsitAn^h}  4 (16) 10 (13)
\end{subvocedue}
\item[{\color{colorlowref}\spzrl{:CsUd.H}},] {\sf asude},\ agg.\ p.:\ quieto, tranquillo.
\begin{subvocedue}
\item[Pron. (1.0):] \spzcite{redhouse1997}
\item[Rif.:] \spzcite[42]{kiefferbianchi18351}
\end{subvocedue}
\begin{subvocedue}
\item[\subglossariobullet] \spzrl{:CsUd.H .hAl}, {\sf asude hal},\ agg.\ p.:\ che gode della tranquillità.
\begin{subvocedue}
\item[Pron. (1.0):] \spzcite{redhouse1997}
\item[Rif.:] \spzcite[42]{kiefferbianchi18351}
\end{subvocedue}
\item[(simil:1.0)] \spzrl{asUd.H}  5 (11) 24 (9)
\end{subvocedue}


Contesti:
\begin{subvocedue}
\item[(riga 5)] \spzrl{asUd.H .hAl wa mar.di_Y||-al-bAl q:ilinmalar.I}
\item[(riga 24)] \spzrl{asUd.H .hAl wa mar.diy||-al-bAl .Ql:H}
\end{subvocedue}
\item[{\color{colorlowref}\spzrl{eker}},] {\sf eğer},\ cong.\ p.:\ se.
\begin{subvocedue}
\item[Pron. (1.0):] \spzcite[328]{redhouse1997}
\item[Rif.:] \spzcite[78]{kiefferbianchi18351}
\end{subvocedue}
\begin{subvocedue}
\item[(var)] \spzrl{eger}, {\sf eğer}\begin{subvocedue}
\item[Rif.:] \spzcite[328]{redhouse1997}
\end{subvocedue}
\item[\subglossariobullet] \spzrl{eker bu eker o}, {\sf eğer bu eğer o}:\ sia questo sia quello.
\begin{subvocedue}
\item[Rif.:] \spzcite[78]{kiefferbianchi18351}
\end{subvocedue}
\item[(simil:1.0)] \spzrl{eger}  7 (20)
\end{subvocedue}
\item[{\color{colorlowref}\spzrl{aydir}},] {\sf eider, ider},\ avv.\ p.:\ sempre, qui, allora.
\begin{subvocedue}
\item[Rif.:] \spzcite[154]{kiefferbianchi18351}
\end{subvocedue}
\begin{subvocedue}
\item[\subglossariobullet] \spzrl{aydirs.H}, {\sf eiderse}:\ ovunque?? \verificare.
\item[(radice)] \spzrl{aydirs.H}  17 (13)
\end{subvocedue}
\item[{\color{colorlowref}\spzrl{bAd}},] {\sf bad},\ v.\ p.:\ imperativo: sia, fiat.
\begin{subvocedue}
\item[Rif.:] \spzcite[167]{kiefferbianchi18351}
\end{subvocedue}
\begin{subvocedue}
\item[(simil:1)] \spzrl{bAd}  25 (16)
\end{subvocedue}
\item[{\color{colorlowref}\spzrl{pAdi^sAah}},] {\sf Padişah},\ n.\ p.:\ Padishah, Sultano dell'impero Ottomano.
\begin{subvocedue}
\item[(simil:1)] \spzrl{pAdi^sAah}  22 (10)
\end{subvocedue}
\item[{\color{colorlowref}\spzrl{pur}},] {\sf pür},\ avv./agg.\ p.:\ molto; pieno, abbondante, numeroso.
\begin{subvocedue}
\item[Rif.:] \spzcite[197]{kiefferbianchi18351}
\end{subvocedue}
\begin{subvocedue}
\item[(simil:1)] \spzrl{pur}  6 (1)
\end{subvocedue}
\item[{\color{colorlowref}\spzrl{penb.H}},] {\sf penbe},\ n.\ p.:\ cotone.
\begin{subvocedue}
\item[Rif.:] \spzcite[230]{kiefferbianchi18351}
\end{subvocedue}
\begin{subvocedue}
\item[(simil:1)] \spzrl{penb.H}  7 (17)
\item[(simil:1.0)] \spzrl{penb^h}  17 (8)
\item[(radice)] \spzrl{penb.Hmiz}  9 (6)
\end{subvocedue}
\item[{\color{colorlowref}\spzrl{tAbAn}},] {\sf taban},\ agg.\ p.:\ luminoso, risplendente; meglio (sostantivo).
\begin{subvocedue}
\item[Rif.:] \spzcite[269]{kiefferbianchi18351}
\end{subvocedue}
\begin{subvocedue}
\item[(simil:1)] \spzrl{tAbAn}  25 (13)
\end{subvocedue}
\item[{\color{colorlowref}\spzrl{^gAygIr}},] {\sf caygir},\ agg.\ p.:\ che occorre, stabilito, posto, che occupa, tiene, penetra.
\begin{subvocedue}
\item[Rif.:] \spzcite[363]{kiefferbianchi18351}
\end{subvocedue}
\begin{subvocedue}
\item[(simil:1)] \spzrl{^gAygIr}  21 (17)
\end{subvocedue}
\item[{\color{colorlowref}\spzrl{^gUy}},] {\sf cuy},\ agg.\ p.:\ che vuole, che desidera.
\begin{subvocedue}
\item[Rif.:] \spzcite[405]{kiefferbianchi18351}
\end{subvocedue}
\begin{subvocedue}
\item[\subglossariobullet] \spzrl{^gUy nik}, {\sf cuy nik}:\ che vuole il bene di qualcuno.
\begin{subvocedue}
\item[Rif.:] \spzcite[405]{kiefferbianchi18351}
\end{subvocedue}
\item[(radice)] \spzrl{^gUyliq}  4 (11)
\end{subvocedue}


Contesti:
\begin{subvocedue}
\item[(riga 4)] \spzrl{_hayir ^gUyliq}
\end{subvocedue}
\item[{\color{colorlowref}\spzrl{^cu:vAl}},] {\sf çuval},\ n.\ p.:\ sacco.
\begin{subvocedue}
\item[Rif.:] \spzcite[398]{kiefferbianchi18351}
\end{subvocedue}
\begin{subvocedue}
\item[(simil:1)] \spzrl{^cu:vAl}  9 (11)
\end{subvocedue}
\item[{\color{colorlowref}\spzrl{_h_wAh}},] {\sf hah},\ agg.\ p.:\ colui che desidera; desiderio, volontà.
\begin{subvocedue}
\item[Rif.:] \spzcite[490]{kiefferbianchi18351}
\end{subvocedue}
\begin{subvocedue}
\item[\subglossariobullet] \spzrl{_h_wAhliq}, {\sf hahlık},\ n.\ pt.:\ volonta, disposizione d'animo.
\begin{subvocedue}
\item[Rif.:] \spzcite[490]{kiefferbianchi18351}
\end{subvocedue}
\item[(radice)] \spzrl{_h_wAhliq}  4 (6)
\end{subvocedue}
\item[{\color{colorsologlossario}{\bf (g)}}] {\color{colorsologlossario}\spzrl{_hUd}}, {\sf hod},\ aggp.\ p.:\ stesso; stesso, sé stesso, proprio.
\begin{subvocedue}
\item[Rif.:] \spzcite[491]{kiefferbianchi18351}
\end{subvocedue}
\begin{subvocedue}
\item[\subglossariobullet] \spzrl{b:an _hUd}, {\sf ben hod}:\ me stesso, quanto a me.
\item[\subglossariobullet] \spzrl{_hUd bIn}, {\sf hod bin}:\ che non vede che sé stesso, egoista.
\item[\subglossariobullet] \spzrl{_hUd b_hUd}, {\sf hod behod}:\ sua sponte, senza essere autorizzato.
\begin{subvocedue}
\item[Rif.:] \spzcite[491]{kiefferbianchi18351}
\end{subvocedue}
\end{subvocedue}
\item[{\color{colorlowref}\spzrl{dira_h^sAn}},] {\sf dirahşan},\ agg.\ p.:\ chiaro, brillante, lampeggiante.
\begin{subvocedue}
\item[Rif.:] \spzcite[515]{kiefferbianchi18351}
\end{subvocedue}
\begin{subvocedue}
\item[\subglossariobullet] \spzrl{dira_h^sAn .QlB}, {\sf dirahşan ol-}:\ illuminare, lampeggiare, brillare.
\begin{subvocedue}
\item[Rif.:] \spzcite[515]{kiefferbianchi18351}
\end{subvocedue}
\item[(simil:1.0)] \spzrl{dir_ha^sAn}  25 (15)
\end{subvocedue}
\item[{\color{colorlowref}\spzrl{darUnI}},] {\sf deruni},\ agg.\ p.:\ interno, spirituale.
\begin{subvocedue}
\item[Rif.:] \spzcite[520]{kiefferbianchi18351}
\end{subvocedue}
\begin{subvocedue}
\item[(simil:1)] \spzrl{darUnI}  2 (16)
\end{subvocedue}
\item[{\color{colorlowref}\spzrl{dirI.g}},] {\sf diriγ},\ n.\ p.:\ rifiuto, repulsione; mancanza, omissione.
\begin{subvocedue}
\item[Rif.:] \spzcite[522]{kiefferbianchi18351}
\end{subvocedue}
\begin{subvocedue}
\item[\subglossariobullet] \spzrl{dirI.g :EtB}, {\sf diriğ et-}:\ rifiutare.
\item[\subglossariobullet] \spzrl{.husni na.zar^nzY dirI.g bIwr mAm^nz ma'amUldir}, {\sf husni nazariŋizi dirig buyurmamaŋiz me'emuldir}:\ speriamo che non smettiate di tenerci nelle vostre grazie.
\item[(simil:1)] \spzrl{dirI.g}  5 (21) 20 (12)
\end{subvocedue}
\item[{\color{colorlowref}\spzrl{dil.H}},] {\sf dile},\ n.\ p.:\ cuore, anima.
\begin{subvocedue}
\item[Rif.:] \spzcite[541]{kiefferbianchi18351}
\end{subvocedue}
\begin{subvocedue}
\item[(simil:1)] \spzrl{dil.H}  21 (10)
\end{subvocedue}
\item[{\color{colorlowref}\spzrl{dUst}},] {\sf dost},\ n.\ p.:\ amico.
\begin{subvocedue}
\item[Rif.:] \spzcite[553]{kiefferbianchi18351}
\end{subvocedue}
\begin{subvocedue}
\item[\subglossariobullet] \spzrl{dUstluq}, {\sf dostluq}:\ amicizia.
\item[(radice)] \spzrl{dUstlu.g:H}  2 (13)
\item[(radice)] \spzrl{dUstl.U.gumuz:H}  3 (19)
\item[(radice)] \spzrl{dUstlU.gin:H}  12 (12)
\item[(radice)] \spzrl{dUstUmuzah}  11 (1)
\end{subvocedue}
\item[{\color{colorlowref}\spzrl{rUz}},] {\sf ruz},\ n.\ p.:\ giorno, dì.
\begin{subvocedue}
\item[Rif.:] \spzcite[604]{kiefferbianchi18351}
\end{subvocedue}
\begin{subvocedue}
\item[\subglossariobullet] \spzrl{rUzi ^gezA}, {\sf ruzi ceza}:\ giorno del Giudizio.
\begin{subvocedue}
\item[Rif.:] \spzcite[604]{kiefferbianchi18351}
\end{subvocedue}
\item[(simil:1)] \spzrl{rUz}  21 (14)
\end{subvocedue}


Contesti:
\begin{subvocedue}
\item[(riga 21)] \spzrl{rUz .ha^sr:H}
\end{subvocedue}
\item[{\color{colorlowref}\spzrl{rU^sin}},] {\sf ruşen},\ agg.\ p.:\ chiaro, manifesto, luminoso, splendente.
\begin{subvocedue}
\item[Rif.:] \spzcite[606]{kiefferbianchi18351}
\end{subvocedue}
\begin{subvocedue}
\item[(simil:1.0)] \spzrl{rU^sun}  6 (5)
\end{subvocedue}
\item[{\color{colorlowref}\spzrl{sAyah}},] {\sf saye},\ n.\ p.:\ ombra, protezione, favore.
\begin{subvocedue}
\item[Rif.:] \spzcite[644]{kiefferbianchi18351}
\end{subvocedue}
\begin{subvocedue}
\item[(simil:1.0)] \spzrl{sAya^h}  24 (7)
\end{subvocedue}


Contesti:
\begin{subvocedue}
\item[(riga 24)] \spzrl{sAya^h .himAyat^nizd:H}
\end{subvocedue}
\item[{\color{colorlowref}\spzrl{sipAri^s}},] {\sf sipariş},\ n.\ p.:\ ordine, commissione, raccomandazione.
\begin{subvocedue}
\item[\subglossariobullet] \spzrl{sipAri^s :EtB}, {\sf sipariş ėt-}:\ raccomandare, commissionare.
\begin{subvocedue}
\item[Rif.:] \spzcite[644]{kiefferbianchi18351}
\end{subvocedue}
\item[(simil:1)] \spzrl{sipAri^s}  13 (13) 18 (9)
\end{subvocedue}


Contesti:
\begin{subvocedue}
\item[(riga 18)] \spzrl{sipAri^s :Etd.Ukimiz}
\item[(riga 13)] \spzrl{sipAri^s :Etm:akY}
\end{subvocedue}
\item[{\color{colorlowref}\spzrl{sutUd.T}},] {\sf sütude},\ agg.\ p.:\ lodabile,lodato.
\begin{subvocedue}
\item[Rif.:] \spzcite[651]{kiefferbianchi18351}
\end{subvocedue}
\begin{subvocedue}
\item[\subglossariobullet] \spzrl{sutUdaT am_hu.d}, {\sf sütude emhuz}:\ di lodabile purezza.
\item[(simil:1.0)] \spzrl{sutUdaT}  6 (10)
\end{subvocedue}
\item[{\color{colorlowref}\spzrl{sikk.H}},] {\sf sikke},\ n.\ p.:\ moneta.
\begin{subvocedue}
\item[Rif.:] \spzcite[683]{kiefferbianchi18351}
\end{subvocedue}
\begin{subvocedue}
\item[(simil:1)] \spzrl{sikk.H}  7 (15)
\item[(simil:1.0)] \spzrl{sikk^h}  17 (22)
\end{subvocedue}
\item[{\color{colorlowref}\spzrl{firAmU^s}},] {\sf feramûş},\ n.\ p.:\ dimenticanza.
\begin{subvocedue}
\item[Rif.:] \spzcite[361]{kiefferbianchi18352}
\end{subvocedue}
\begin{subvocedue}
\item[(var)] \spzrl{firAmu^s}, {\sf feramuş}\begin{subvocedue}
\item[Rif.:] \spzcite[361]{kiefferbianchi18352}
\end{subvocedue}
\item[\subglossariobullet] \spzrl{firAmU^s :EtB}, {\sf feramûş ėt-}:\ dimenticare.
\begin{subvocedue}
\item[Rif.:] \spzcite[361]{kiefferbianchi18352}
\end{subvocedue}
\item[(simil:1)] \spzrl{firAmU^s}  21 (24)
\end{subvocedue}
\item[{\color{colorlowref}\spzrl{k^H}},] {\sf ki},\ cong.\ p.:\ che.
\begin{subvocedue}
\item[(simil:1)] \spzrl{k^H}  2 (24) 6 (16) 9 (7) 17 (1) 20 (5,14) 21 (13) 24 (6)
\end{subvocedue}
\item[{\color{colorlowref}\spzrl{gumAn}},] {\sf güman},\ n.\ p.:\ dubbio, incertezza.
\begin{subvocedue}
\item[Rif.:] \spzcite[636]{kiefferbianchi18352}
\end{subvocedue}
\begin{subvocedue}
\item[\subglossariobullet] \spzrl{gumAn :EtB}, {\sf güman ėt-}:\ pensare, dubitare.
\item[(simil:1)] \spzrl{gumAn}  22 (5)
\end{subvocedue}
\item[{\color{colorlowref}\spzrl{murd}},] {\sf murd},\ n.\ p.:\ morto.
\begin{subvocedue}
\item[Rif.:] \spzcite[864]{kiefferbianchi18352}
\end{subvocedue}
\begin{subvocedue}
\item[\subglossariobullet] \spzrl{murd .QlB}, {\sf murd ol-}:\ morire.
\begin{subvocedue}
\item[Rif.:] \spzcite[864]{kiefferbianchi18352}
\end{subvocedue}
\item[(simil:1)] \spzrl{murd}  7 (22)
\item[(simil:0.5)] \spzrl{murd-i}  10 (0)
\end{subvocedue}
\item[{\color{colorlowref}\spzrl{nAm}},] {\sf nam},\ n.\ p.:\ nome, fama, onore.
\begin{subvocedue}
\item[Rif.:] \spzcite[1084]{kiefferbianchi18352}
\end{subvocedue}
\begin{subvocedue}
\item[(simil:1)] \spzrl{nAm}  6 (26) 10 (5,20) 12 (15) 16 (10) 17 (6)
\item[(radice)] \spzrl{nAmimiz.H}  19 (3)
\end{subvocedue}
\item[{\color{colorlowref}\spzrl{nihAn}},] {\sf nihan},\ agg.\ p.:\ segreto, interno.
\begin{subvocedue}
\item[Rif.:] \spzcite[1150]{kiefferbianchi18352}
\end{subvocedue}
\begin{subvocedue}
\item[(simil:1)] \spzrl{nihAn}  6 (14)
\end{subvocedue}
\item[{\color{colorlowref}\spzrl{nIk}},] {\sf nik},\ agg./avv.\ p.:\ buono, eccellente, fortunato; bene, fortuna; come avverbio, molto.
\begin{subvocedue}
\item[Rif.:] \spzcite[1155]{kiefferbianchi18352}
\end{subvocedue}
\begin{subvocedue}
\item[(simil:1)] \spzrl{nI:k}  4 (5) 8 (1)
\end{subvocedue}
\item[{\color{colorlowref}\spzrl{her}},] {\sf her},\ aggp.\ p.:\ ogni, ognuno.
\begin{subvocedue}
\item[Rif.:] \spzcite[1212]{kiefferbianchi18352}
\end{subvocedue}
\begin{subvocedue}
\item[\subglossariobullet] \spzrl{hern.H k^H}, {\sf herneki}:\ qualunque che, ogni cosa che \verificare.
\begin{subvocedue}
\item[Rif.:] \spzcite[1212]{kiefferbianchi18352}
\end{subvocedue}
\item[\subglossariobullet] \spzrl{hern.H}, {\sf herne}:\ qualunque che, ogni cosa che.
\begin{subvocedue}
\item[Rif.:] \spzcite[1212]{kiefferbianchi18352}
\end{subvocedue}
\item[\subglossariobullet] \spzrl{hernek^H}, {\sf herneki}:\ qualunque che, ogni cosa che.
\item[(simil:1)] \spzrl{her}  8 (13)
\item[(radice)] \spzrl{hernek^H}  17 (11) 18 (12)
\end{subvocedue}
\item[{\color{colorlowref}\spzrl{himAn}},] {\sf hemen},\ avv.\ p.:\ solamente, non più; anche, lo stesso, esattamente; sempre; adesso.
\begin{subvocedue}
\item[Rif.:] \spzcite[1224]{kiefferbianchi18352}
\end{subvocedue}
\begin{subvocedue}
\item[(simil:1.0)] \spzrl{hamAn}  15 (19)
\end{subvocedue}
\end{glossario}
