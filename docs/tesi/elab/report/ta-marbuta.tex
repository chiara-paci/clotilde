\begin{glossario}{Ta Marbuta}
\item[{\color{colorlowref}\spzrl{ayAlat}},] {\sf eyalet},\ n.\ a.:\ principato; prefettura, governo di una provincia.
\begin{subvocedue}
\item[Rif.:] \spzcite[151-152]{kiefferbianchi18351}
\end{subvocedue}
\begin{subvocedue}
\item[\subglossariobullet] \spzrl{ayAlat wilAyat iznikmId ^A^n^A tafwI.d .Qlund.Y}, {\sf eyaleti vilaieti iznikmid aŋa tefviz olundı}:\ gli fu conferito il governo della provincia di Nicomedia \verificare[trascrizione].
\begin{subvocedue}
\item[Rif.:] \spzcite[151]{kiefferbianchi18351}
\end{subvocedue}
\item[\subglossariobullet] \spzrl{ayAlat :EtB}, {\sf eyalet et-},\ v.\ at.:\ governare una provincia.
\begin{subvocedue}
\item[Rif.:] \spzcite[152]{kiefferbianchi18351}
\end{subvocedue}
\item[(radice)] \spzrl{ayAlatind:H}  6 (23)
\end{subvocedue}
\item[{\color{colorlowref}\spzrl{barIyaT}},] {\sf beriya},\ n.\ a.:\ vassallo; popolo, creatura; cittadino libero dello stato musulmano?.
\begin{subvocedue}
\item[Rif.:] \spzcite[209]{kiefferbianchi18351}
\end{subvocedue}
\begin{subvocedue}
\item[(var)] \spzrl{barAyA}, {\sf beraya}, plurale\begin{subvocedue}
\item[Rif.:] \spzcite[199]{kiefferbianchi18351}
\end{subvocedue}
\item[\subglossariobullet] \spzrl{ra`AyA wa  barAyA}, {\sf reaya u beraya}:\ vassalli e sudditi.
\begin{subvocedue}
\item[Rif.:] \spzcite[199]{kiefferbianchi18351}
\end{subvocedue}
\item[(simil:1.0)] \spzrl{barAyA}  5 (8) 23 (1)
\end{subvocedue}


Contesti:
\begin{subvocedue}
\item[(riga 5)] \spzrl{ra`AyA wa barAyA}
\item[(righe 22-23)] \spzrl{ra`AyA wa barAyA}
\end{subvocedue}
\item[{\color{colorlowref}\spzrl{.hasin.H}},] {\sf hasene},\ n.\ a.:\ buon  lavoro,   buono,  piacevole.
\begin{subvocedue}
\item[Rif.:] \spzcite[425]{kiefferbianchi18351}
\end{subvocedue}
\begin{subvocedue}
\item[(simil:1)] \spzrl{.hasin.H}  17 (23)
\end{subvocedue}
\item[{\color{colorlowref}\spzrl{.hukUmat}},] {\sf hükûmet},\ n.\ a.:\ governo, amministrazione, giurisdizione, potere.
\begin{subvocedue}
\item[Rif.:] \spzcite[433]{kiefferbianchi18351}
\end{subvocedue}
\begin{subvocedue}
\item[(radice)] \spzrl{.hukUmatlar.I}  13 (0)
\end{subvocedue}
\item[{\color{colorlowref}\spzrl{.himAyat}},] {\sf himayet},\ n.\ a.:\ protezione, difesa, supporto.
\begin{subvocedue}
\item[Rif.:] \spzcite[436-437]{kiefferbianchi18351}
\end{subvocedue}
\begin{subvocedue}
\item[(var)] \spzrl{.himAy.H}, {\sf himaye}\item[(simil:1)] \spzrl{.himAyat}  24 (20)
\item[(radice)] \spzrl{.himAyat^nizd:H}  24 (8)
\end{subvocedue}
\item[{\color{colorlowref}\spzrl{_hidmat}},] {\sf hidmet},\ n.\ a.:\ servizio,  utilità,  lavoro, funzione, occupazione, impiego.
\begin{subvocedue}
\item[Rif.:] \spzcite[458]{kiefferbianchi18351}
\end{subvocedue}
\begin{subvocedue}
\item[(var)] \spzrl{_hi_dmat}, {\sf hizmet \textchi idmet}\begin{subvocedue}
\item[Rif.:] \spzcite[454]{bib:kiefferbianchi18351}
\end{subvocedue}
\item[\subglossariobullet] \spzrl{_hidmat bAqI}, {\sf hizmet baqi}:\ il servizio resta, ossia «sarò riconoscente per il servizio che mi verrà reso».
\begin{subvocedue}
\item[Rif.:] \spzcite[458]{kiefferbianchi18351}
\end{subvocedue}
\item[(simil:1.0)] \spzrl{_hi_dmat}  16 (19)
\item[(radice)] \spzrl{_hi_dmatd:H}  23 (8)
\end{subvocedue}
\item[{\color{colorlowref}\spzrl{dara^g:H}},] {\sf derece},\ n.\ a.:\ grado, rango, passo, step.
\begin{subvocedue}
\item[Rif.:] \spzcite[515]{kiefferbianchi18351}
\end{subvocedue}
\begin{subvocedue}
\item[(simil:1)] \spzrl{dara^g:H}  4 (4)
\end{subvocedue}
\item[{\color{colorlowref}\spzrl{diqqat}},] {\sf dikkat},\ n.\ a.:\ particolare cura, attenzione; precisione.
\begin{subvocedue}
\item[Rif.:] \spzcite[531]{kiefferbianchi18351}
\end{subvocedue}
\begin{subvocedue}
\item[(var)] \spzrl{diqat}, {\sf dikat}\item[\subglossariobullet] \spzrl{diqqat wa ihtimAm eyleyesiz}, {\sf dikkat u ihitmam eyleyesiz}:\ impiegate tutta la vostra precisione e la vostra cura.
\begin{subvocedue}
\item[Rif.:] \spzcite[531]{kiefferbianchi18351}
\end{subvocedue}
\item[(simil:1.0)] \spzrl{diqat}  8 (25)
\end{subvocedue}
\item[{\color{colorlowref}\spzrl{daqIqa_H}},] {\sf dakika},\ n.\ a.:\ minuto, momento, minuzia.
\begin{subvocedue}
\item[Rif.:] \spzcite[531-532]{kiefferbianchi18351}
\end{subvocedue}
\begin{subvocedue}
\item[\subglossariobullet] \spzrl{daqIqa_H fawt :EtmeB}, {\sf dakika fevt ėtme-}:\ non trascurare nulla.
\begin{subvocedue}
\item[Rif.:] \spzcite[531-532]{kiefferbianchi18351}
\end{subvocedue}
\item[(simil:1.0)] \spzrl{daqIq:H}  9 (0)
\end{subvocedue}
\item[{\color{colorlowref}\spzrl{dawlat}},] {\sf devlet},\ n.\ a.:\ fortuna, ricchezza; stato, governo.
\begin{subvocedue}
\item[Rif.:] \spzcite[560]{kiefferbianchi18351}
\end{subvocedue}
\begin{subvocedue}
\item[\subglossariobullet] \spzrl{dawlatl:H}, {\sf devletle}:\ Buona fortuna! (detto all'ospite che parte) \verificare.
\item[\subglossariobullet] \spzrl{dawlatmadAr}, {\sf devletmedar},\ agg.\ ap.:\ centro di fortuna, felicissimo.
\begin{subvocedue}
\item[Rif.:] \spzcite[561]{kiefferbianchi18351}
\end{subvocedue}
\item[(simil:1)] \spzrl{dawlat}  10 (14)
\item[(simil:0.5)] \spzrl{dawlat-i}  11 (18)
\item[(radice)] \spzrl{dawlatd:H}  12 (3)
\item[(radice)] \spzrl{dawlatl:H}  11 (15)
\end{subvocedue}
\item[{\color{colorlowref}\spzrl{_dAt}},] {\sf zat},\ n.\ a.:\ persona, essenza, sostanza; epiteto per varie cose; essenza, natura; la cosa o la persona stessa.
\begin{subvocedue}
\item[Rif.:] \spzcite[575]{kiefferbianchi18351}
\end{subvocedue}
\begin{subvocedue}
\item[\subglossariobullet] \spzrl{_dAt makArim .sifAt}, {\sf zat mekârim sifat}:\ persona dotata di eccellenti qualità \verificare[trascrizione].
\begin{subvocedue}
\item[Rif.:] \spzcite[575]{kiefferbianchi18351}
\end{subvocedue}
\item[(simil:1)] \spzrl{_dAt}  6 (8)
\end{subvocedue}
\item[{\color{colorlowref}\spzrl{rA.hila_H}},] {\sf rahile},\ n.\ a.:\ cose che  viaggiano insieme, carovana, successione di onoreficienze \verificare.
\begin{subvocedue}
\item[Rif.:] \spzcite[580]{kiefferbianchi18351}
\end{subvocedue}
\begin{subvocedue}
\item[(var)] \spzrl{rawA.hil}, {\sf ravahil}, plurale\begin{subvocedue}
\item[Rif.:] \spzcite[580]{kiefferbianchi18351}
\end{subvocedue}
\item[(simil:0.5)] \spzrl{rawA.hil-i}  2 (10)
\end{subvocedue}
\item[{\color{colorlowref}\spzrl{ra`iyat}},] {\sf raiyat},\ n.\ a.:\ suddito, persona sotto il governo di qualcuno.
\begin{subvocedue}
\item[Rif.:] \spzcite[595]{kiefferbianchi18351}
\end{subvocedue}
\begin{subvocedue}
\item[\subglossariobullet] \spzrl{ra`AyA}, {\sf reaya}:\ non    musulmani   (cristiani)   sudditi dell'Impero Ottomano.
\begin{subvocedue}
\item[Rif.:] \spzcite[595]{kiefferbianchi18351}
\end{subvocedue}
\item[(radice)] \spzrl{ra`AyA}  5 (6) 22 (27)
\item[(radice)] \spzrl{ra`iyati^niz.H}  7 (0)
\end{subvocedue}
\item[{\color{colorlowref}\spzrl{sa`Adat}},] {\sf saadet},\ n.\ a.:\ felicità, prosperità.
\begin{subvocedue}
\item[Rif.:] \spzcite[672]{kiefferbianchi18351}
\end{subvocedue}
\begin{subvocedue}
\item[\subglossariobullet] \spzrl{sa`Adatl.U}, {\sf saadetlu}:\ prosperoso, felice, fortunato (anche come titolo).
\begin{subvocedue}
\item[Rif.:] \spzcite[672]{kiefferbianchi18351}
\end{subvocedue}
\item[(radice)] \spzrl{sa`Adatl.U}  22 (9)
\item[(radice)] \spzrl{sa`Adat:H}  4 (17)
\end{subvocedue}


Contesti:
\begin{subvocedue}
\item[(riga 22)] \spzrl{mAdAmk^H sa`Adatl.U}
\item[(riga 4)] \spzrl{:AsitAn^h sa`Adat:H}
\end{subvocedue}
\item[{\color{colorlowref}\spzrl{.sadAqat}},] {\sf sadakat},\ n.\ a.:\ lealtà, fiducia, amicizia.
\begin{subvocedue}
\item[Rif.:] \spzcite[97]{kiefferbianchi18352}
\end{subvocedue}
\begin{subvocedue}
\item[(var)] \spzrl{.sadAqaT}, {\sf sadakat}\begin{subvocedue}
\item[Rif.:] \spzcite[97]{kiefferbianchi18352}
\end{subvocedue}
\item[\subglossariobullet] \spzrl{.sadAqat .Jl.H}, {\sf sadakat ile}:\ sinceramente.
\begin{subvocedue}
\item[Rif.:] \spzcite[97]{kiefferbianchi18352}
\end{subvocedue}
\item[(simil:1)] \spzrl{.sadAqat}  14 (2)
\end{subvocedue}


Contesti:
\begin{subvocedue}
\item[(riga 14)] \spzrl{.sadAqat :Wzer.H}
\end{subvocedue}
\item[{\color{colorlowref}\spzrl{.tawIyat}},] {\sf taviyet},\ n.\ a.:\ proposta,    intenzione,    pensiero, disposizione, fede.
\begin{subvocedue}
\item[Rif.:] \spzcite[205]{kiefferbianchi18352}
\end{subvocedue}
\begin{subvocedue}
\item[(var)] \spzrl{.tawIyaT}, {\sf taviyet}\begin{subvocedue}
\item[Rif.:] \spzcite[205]{kiefferbianchi18352}
\end{subvocedue}
\item[(simil:1.0)] \spzrl{.tawiyyat}  3 (16)
\end{subvocedue}


Contesti:
\begin{subvocedue}
\item[(riga 3)] \spzrl{.tawiyyat :Wzer.H}
\end{subvocedue}
\item[{\color{colorlowref}\spzrl{`adAla_T}},] {\sf adalet},\ n.\ a.:\ giustizia, equità.
\begin{subvocedue}
\item[Rif.:] \spzcite[239]{kiefferbianchi18352}
\end{subvocedue}
\begin{subvocedue}
\item[\subglossariobullet] \spzrl{`adAlat .Jl.H}, {\sf adalet ile}:\ con giustizia, con equità.
\begin{subvocedue}
\item[Rif.:] \spzcite[239]{kiefferbianchi18352}
\end{subvocedue}
\item[(radice)] \spzrl{`adAlatlar.I}  12 (18)
\end{subvocedue}
\item[{\color{colorlowref}\spzrl{`alAma_T}},] {\sf alamet},\ n.\ a.:\ segno.
\begin{subvocedue}
\item[Rif.:] \spzcite[277]{kiefferbianchi18352}
\end{subvocedue}
\begin{subvocedue}
\item[(radice)] \spzrl{`alAmatlar.In}  3 (3)
\end{subvocedue}
\item[{\color{colorlowref}\spzrl{`inAya_T}},] {\sf inayet},\ n.\ a.:\ grazia, favore; sollecitudine, cura.
\begin{subvocedue}
\item[Rif.:] \spzcite[289-290]{kiefferbianchi18352}
\end{subvocedue}
\begin{subvocedue}
\item[\subglossariobullet] \spzrl{`inAyat  :EtB}, {\sf inayet et-}:\ soccorrere, proteggere, fare con benevolenza.
\begin{subvocedue}
\item[Rif.:] \spzcite[289-290]{kiefferbianchi18352}
\end{subvocedue}
\item[(radice)] \spzrl{`inAyatdir}  24 (22)
\item[(radice)] \spzrl{`inAyati^n}  23 (24)
\item[(radice)] \spzrl{`inAyati^niz}  20 (9)
\end{subvocedue}
\item[{\color{colorlowref}\spzrl{fawt}},] {\sf fevt},\ n.\ a.:\ omissione, negligenza.
\begin{subvocedue}
\item[Rif.:] \spzcite[402]{kiefferbianchi18352}
\end{subvocedue}
\begin{subvocedue}
\item[\subglossariobullet] \spzrl{fawt :EtB}, {\sf fevt et-}:\ trascurare, essere negligente.
\begin{subvocedue}
\item[Rif.:] \spzcite[402]{kiefferbianchi18352}
\end{subvocedue}
\item[\subglossariobullet] \spzrl{daqIqa_H fawt .QlunB}, {\sf dakika fevt olun-}:\ non mancare in alcun punto.
\begin{subvocedue}
\item[Rif.:] \spzcite[402]{kiefferbianchi18352}
\end{subvocedue}
\item[(simil:1)] \spzrl{fawt}  9 (1)
\end{subvocedue}


Contesti:
\begin{subvocedue}
\item[(riga 9)] \spzrl{fawt :Etm.Iyad.I}
\end{subvocedue}
\item[{\color{colorlowref}\spzrl{qAfila_H}},] {\sf kafile},\ n.\ a.:\ cose che  si succedono in  fila,  carovana, successione  di  onoreficienze \verificare.
\begin{subvocedue}
\item[Rif.:] \spzcite[422]{kiefferbianchi18352}
\end{subvocedue}
\begin{subvocedue}
\item[(var)] \spzrl{qawAfil}, {\sf kavafil}, plurale\begin{subvocedue}
\item[Rif.:] \spzcite[513]{kiefferbianchi18352}
\end{subvocedue}
\item[(simil:0.5)] \spzrl{qawAfil-i}  2 (7)
\end{subvocedue}
\item[{\color{colorlowref}\spzrl{qab.da_T}},] {\sf kabza},\ n.\ a.:\ manico,  elsa; potere.
\begin{subvocedue}
\item[Rif.:] \spzcite[436]{kiefferbianchi18352}
\end{subvocedue}
\begin{subvocedue}
\item[(simil:1)] \spzrl{qab.dah}  19 (20)
\end{subvocedue}


Contesti:
\begin{subvocedue}
\item[(riga 19)] \spzrl{qab.dah wakIl}
\end{subvocedue}
\item[{\color{colorlowref}\spzrl{qudrat}},] {\sf kudret},\ n.\ a.:\ potere, forza, capacità, natura.
\begin{subvocedue}
\item[Rif.:] \spzcite[448]{kiefferbianchi18352}
\end{subvocedue}
\begin{subvocedue}
\item[(simil:1)] \spzrl{qudrat}  23 (14)
\end{subvocedue}


Contesti:
\begin{subvocedue}
\item[(riga 23)] \spzrl{bi_dl qudrat}
\end{subvocedue}
\item[{\color{colorlowref}\spzrl{qi.ta`a_T}},] {\sf kıt'a},\ n.\ a.:\ pezzo.
\begin{subvocedue}
\item[Rif.:] \spzcite[489]{kiefferbianchi18352}
\end{subvocedue}
\begin{subvocedue}
\item[\subglossariobullet] \spzrl{d:Ort qi.ta`a_T qAdir.ga_H}, {\sf dört kita kadirga}:\ quattro galere.
\begin{subvocedue}
\item[Rif.:] \spzcite[489]{kiefferbianchi18352}
\end{subvocedue}
\item[(simil:1.0)] \spzrl{qi.ta`_H}  9 (10)
\end{subvocedue}
\item[{\color{colorlowref}\spzrl{mu.habba_T}},] {\sf muhabbet},\ n.\ a.:\ amore, amicizia.
\begin{subvocedue}
\item[Rif.:] \spzcite[817]{kiefferbianchi18352}
\end{subvocedue}
\begin{subvocedue}
\item[(var)] \spzrl{mu.haba_T}, {\sf muhabet}\item[\subglossariobullet] \spzrl{mu.habbat :EtB}, {\sf muhabbet ėt-}:\ amare, essere amici.
\begin{subvocedue}
\item[Rif.:] \spzcite[817]{kiefferbianchi18352}
\end{subvocedue}
\item[(radice)] \spzrl{mu.habbat}  3 (0) 5 (2) 14 (4)
\item[(radice)] \spzrl{mu.habat}  2 (11) 23 (5)
\item[(radice)] \spzrl{mu.habatin.H}  12 (9)
\end{subvocedue}


Contesti:
\begin{subvocedue}
\item[(righe 2-3)] \spzrl{^sarIfi^niz .haqqind:H b.U mu.hiblar.I mu.habbat wa ^J`tibAr `alAmatlar.In}
\item[(righe 4-5)] \spzrl{wufUr-i mawaddat wa kamAl-i mu.habbat i.zhAr :Etd.Ukumuzden mA`adA}
\item[(riga 14)] \spzrl{_hulU.s wa .sadAqat :Wzer.H mu.habbat :Etm:akl.H b.U amrd:H}
\item[(riga 2)] \spzrl{qawAfil-i mawaddat wa rawA.hil-i mu.habat .Jl.H dUstlu.g:H lAyiq}
\item[(riga 23)] \spzrl{barAyA g:Oz .haqind:H mazId mu.habat wa d.U^sen _hi_dmatd:H}
\item[(riga 12)] \spzrl{b^Ayl.Osu^nuzu^N far.t-i mawaddat wa mu.habatin.H ilti^gA'^a wa dUstlU.gin:H}
\end{subvocedue}
\item[{\color{colorlowref}\spzrl{murA^ga`a_T}},] {\sf müracaat},\ n.\ a.:\ ricorso, ritorno.
\begin{subvocedue}
\item[Rif.:] \spzcite[856]{kiefferbianchi18352}
\end{subvocedue}
\begin{subvocedue}
\item[\subglossariobullet] \spzrl{murA^ga`at :EtB}, {\sf müracaat ėt-}:\ rivolgersi.
\item[(radice)] \spzrl{murA^ga`at}  11 (2)
\end{subvocedue}


Contesti:
\begin{subvocedue}
\item[(riga 11)] \spzrl{murA^ga`at .JdUb}
\end{subvocedue}
\item[{\color{colorlowref}\spzrl{murA`A_T}},] {\sf müraat},\ n.\ a.:\ azione di mantenere un precetto, osservanza, obbligazione, dovere.
\begin{subvocedue}
\item[Rif.:] \spzcite[858]{kiefferbianchi18352}
\end{subvocedue}
\begin{subvocedue}
\item[(radice)] \spzrl{murA`At}  16 (16)
\end{subvocedue}
\item[{\color{colorlowref}\spzrl{martab:H}},] {\sf mertebe},\ n.\ a.:\ grado, ordine, posizione, rango.
\begin{subvocedue}
\item[Rif.:] \spzcite[860]{kiefferbianchi18352}
\end{subvocedue}
\begin{subvocedue}
\item[\subglossariobullet] \spzrl{martabah-dan martabah-ya}, {\sf mertebeden mertebeye}:\ per gradi, di livello in livello.
\item[\subglossariobullet] \spzrl{tA bU martabah kah}, {\sf ta bu mertebe ki}:\ al punto che, a tal punto che.
\item[\subglossariobullet] \spzrl{kamAl martabah}, {\sf kemal mertebe}:\ perfettamente, al più alto grado.
\item[\subglossariobullet] \spzrl{qAdir awldUnimiz martabah}, {\sf qadyr olu\textgamma ümüz mertebe}:\ tutto quello che possiamo.
\item[(simil:1)] \spzrl{martab:H}  4 (9)
\item[(radice)] \spzrl{martab:Hmiz}  22 (22)
\end{subvocedue}
\item[{\color{colorlowref}\spzrl{muruwwa_T}},] {\sf mürüvvet},\ n.\ a.:\ umanità, pietà, bontà.
\begin{subvocedue}
\item[Rif.:] \spzcite[872]{kiefferbianchi18352}
\end{subvocedue}
\begin{subvocedue}
\item[\subglossariobullet] \spzrl{muruwat||ma'ab}, {\sf mürüvvetmeab},\ n.\ a.:\ pieno di umanità.
\begin{subvocedue}
\item[Rif.:] \spzcite[727]{kiefferbianchi18352}, \spzcite[872]{kiefferbianchi18352}
\end{subvocedue}
\item[(radice)] \spzrl{muruwwat}  21 (5)
\item[(radice)] \spzrl{muruwwat||ma'ablar.I}  3 (22)
\end{subvocedue}
\item[{\color{colorlowref}\spzrl{muqAbal:H}},] {\sf mukabele},\ n.\ a.:\ lo stare  di fronte a  un altro, il confrontarsi, l'essere reciproco, ritorno, opposizione, confronto; incontro, compensazione.
\begin{subvocedue}
\item[Rif.:] \spzcite[969-970]{kiefferbianchi18352}
\end{subvocedue}
\begin{subvocedue}
\item[\subglossariobullet] \spzrl{muqAbal:H||sind.H}, {\sf mukabelesinde}:\ nel suo di fronte.
\begin{subvocedue}
\item[Rif.:] \spzcite[969-970]{kiefferbianchi18352}
\end{subvocedue}
\item[(radice)] \spzrl{muqAbal:H||sind:H}  3 (20)
\end{subvocedue}
\item[{\color{colorlowref}\spzrl{mawadda_T}},] {\sf meveddet},\ n.\ a.:\ amore, amicizia.
\begin{subvocedue}
\item[Rif.:] \spzcite[1043]{kiefferbianchi18352}
\end{subvocedue}
\begin{subvocedue}
\item[(var)] \spzrl{mawada_T}, {\sf mevedet}\item[(radice)] \spzrl{mawadat}  21 (7) 24 (17)
\item[(radice)] \spzrl{mawaddat}  2 (8) 4 (23) 12 (7)
\end{subvocedue}


Contesti:
\begin{subvocedue}
\item[(riga 21)] \spzrl{lu.tuf wa muruwwat qalam mawadat birl.H law.h dil.H}
\item[(riga 24)] \spzrl{.Ql:H bAq_Y||^cUn .gar.d `ar.d mawadat wa ri^gA.I .himAyat}
\item[(riga 12)] \spzrl{dawlatd:H .Ql^An b^Ayl.Osu^nuzu^N far.t-i mawaddat wa mu.habatin.H ilti^gA'^a}
\item[(riga 2)] \spzrl{qawAfil-i mawaddat wa rawA.hil-i mu.habat}
\item[(righe 4-5)] \spzrl{.Jd.H g:aldikl:ar.I b^Ayl.Os||lar:H wufUr-i mawaddat wa kamAl-i mu.habbat}
\end{subvocedue}
\item[{\color{colorlowref}\spzrl{nihAyaT}},] {\sf nihayet},\ n.\ a.:\ fine, estremità, estremo, conclusione, estremamente, molto.
\begin{subvocedue}
\item[Rif.:] \spzcite[1150-1151]{kiefferbianchi18352}
\end{subvocedue}
\begin{subvocedue}
\item[(radice)] \spzrl{nihAyat}  4 (8)
\end{subvocedue}
\item[{\color{colorlowref}\spzrl{niyyaT}},] {\sf niyet},\ n.\ a.:\ soluzione, intenzione, proposito.
\begin{subvocedue}
\item[Rif.:] \spzcite[1153]{kiefferbianchi18352}
\end{subvocedue}
\begin{subvocedue}
\item[\subglossariobullet] \spzrl{niyat :EtB}, {\sf niyet et-}:\ avere intenzione di.
\begin{subvocedue}
\item[Rif.:] \spzcite[1153]{kiefferbianchi18352}
\end{subvocedue}
\item[(radice)] \spzrl{niyyat}  15 (11)
\end{subvocedue}
\item[{\color{colorlowref}\spzrl{wilAyat}},] {\sf vilâyet},\ n.\ a.:\ governo di una provincia, carica.
\begin{subvocedue}
\item[Rif.:] \spzcite[1196]{kiefferbianchi18352}
\end{subvocedue}
\begin{subvocedue}
\item[(simil:1)] \spzrl{wilAyat}  6 (21)
\end{subvocedue}
\item[{\color{colorlowref}\spzrl{himmaT}},] {\sf himmet},\ n.\ a.:\ zelo,   sforzo,   influenza,   auspicio, grazia, favore, aiuto, intenzione.
\begin{subvocedue}
\item[Rif.:] \spzcite[1225]{kiefferbianchi18352}
\end{subvocedue}
\begin{subvocedue}
\item[(var)] \spzrl{himaT}, {\sf himet}\item[\subglossariobullet] \spzrl{himmaT :EtB}, {\sf himmet ėt-}:\ applicarsi, darsi da fare.
\begin{subvocedue}
\item[Rif.:] \spzcite[1225]{kiefferbianchi18352}
\end{subvocedue}
\item[(radice)] \spzrl{himat}  23 (11)
\end{subvocedue}
\item[{\color{colorlowref}\spzrl{yumn.H}},] {\sf yumnet},\ n.\ a.:\ felicità, proseprità.
\begin{subvocedue}
\item[Rif.:] \spzcite[1281]{kiefferbianchi18352}
\end{subvocedue}
\begin{subvocedue}
\item[(simil:1)] \spzrl{yumn.H}  23 (26)
\end{subvocedue}
\item[{\color{colorlowref}\spzrl{yin.H}},] {\sf yine},\ avv.\ a.:\ ancora, di nuovo, ma ancora, inoltre; nonostante.
\begin{subvocedue}
\item[Rif.:] \spzcite[1285]{kiefferbianchi18352}
\end{subvocedue}
\begin{subvocedue}
\item[(simil:1)] \spzrl{yin.H}  15 (14)
\end{subvocedue}
\end{glossario}
