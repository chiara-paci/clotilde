\startnote

\begin{enumerate}


%% \item Ottoman  words are  written in arabic  alphabet.  Pronounciation
%%   indications  are made  with modern  Turkish latin  alphabet  used in
%%   Republic  of  Turkey, with  addition  of  letters \spzchar{ė}  ({\it
%%     closed   e},    \spzipa{e}),   \spzchar{j}   ({\it    french   j},
%%   \spzipa{\textyogh}),  \spzchar{ŋ}   and  \spzchar{\textgamma}~  (for
%%   their phonetic correspondent \spzipa{ŋ} and \spzipa{\textgamma}) and
%%   of  apostrophes   '  and  \TRL{`},  for  arabic   {\it  hamzah}  and
%%   \spzln{`ayn} respectively.  All are indicated on the third column in
%%   \tabella{tab:alfabeto}.  Both  arabic and latin  spellings come from
%%           {\it Redhouse  Türkçe/Osmanlıca-İngilizce Sözlük} dictionary
%%           \spzcite{redhouse1997}  or,  when  absent,  from  Timurtaş's
%%           grammary  \spzcite{timurtas1999}.   Verbs  are written  with
%%           root followed by a dash.

%% \item In suffixes, morfo-phonological variants are written as:

%% \begin{center}
%% \begin{tabular}{*{7}{R=L}}
%% E&a,e&İ&i,ı,u,ü&U&u,ü&C&c,ç&D&d,t&G&g,k (ğ)&P&personal suffix\\
%% \end{tabular}
%% \end{center}

%% Eufonic letters and pronominal \spzchar{n} are written in round brackets.

\item  Il termine {\it  turco} si  riferisce all'insieme  delle lingue
  turche  (approssimativamente come {\it  turkic}), mentre  le singole
  lingue sono indicate, in caso,  per esteso (turco ottomano, turco di
  Turchia,  ecc.).  Lo  stesso  in casi  analoghi  (tataro di  Crimea,
  ecc.).  Quando  non  è  specificato altrimenti,  {\it  ottomano}  si
  riferisce all'ottomano del periodo di cui ci stiamo occupando, ossia
  tra la fine del XVI e l'inizio del XVII secolo.

\item I  termini arabi,  persiani e turchi  ottomani sono  scritti con
  l'alfabeto arabo, la trascrizione in corsivo e la pronuncia in tondo
  senza  grazie    (\vedi   \sezione{sec:notetrascrizione}).
  L'indicazione della pronuncia è data secondo una versione modificata
  dell'alfabeto  del turco di  Turchia.  Per  agevolare la  lettura, è
  stato scelto  di utilizzare  un solo glifo  per grafema,  quindi per
  esempio {\it ḱ}  in luogo del più comune {\it  k$^i$}. L'uso è stato
  esteso per uniformità al resto del lavoro.

\item  Le  parole che  provengono  da una  lingua  che  si scrive  con
  l'alfabeto latino sono riportate  come si scrivono in quella lingua;
  eventualmente  con  indicata la  pronuncia  con l'alfabeto  fonetico
  internazionale se questa si discosta dall'uso attuale.

\item  Le  parole  greche  sono  scritte  con  l'alfabeto  greco,  con
  eventuale  indicazione della  pronuncia tramite  l'alfabeto fonetico
  internazionale, come riportato in \spzcite{simone2008}.

\item Le radici turche antiche  sono scritte secondo le convenzioni di
  \spzcite{clauson1972}.

\item Le  modifiche morfofonologiche  nei suffissi sono  indicate come
  riportato  in \tabella{tab:armonia}  per  quanto riguarda  l'armonia
  vocalica  e  in   \tabella{tab:sandhi}  per  quanto  riguarda  altri
  fenomeni.

\item Simboli:
  \begin{itemize*}        
  \item[\spzto] una trasformazione diacronica;
  \item[\spzder] una  derivazione morfologica  (es.  un plurale  da un
    singolare);
  \item[\spzlessder] una  derivazione lessicale  (es.  un verbo  da un
    nome);
  \item[\spzloanto] un prestito da un'altra lingua; 
  \item[\spzadatt] un adattamento di un prestito alla nuova lingua;
  \item[\spzevol]    una   trasformazione,   modifica    o   variante;
    un'evoluzione incipiente, ma non ancora realizzata;
  \item[\spzradesuffs]  il risultato  dell'unione tra  una radice  e un
    suffisso, con modifica del suffisso;
  \item[\spzradesuffr]  il risultato  dell'unione tra  una radice  e un
    suffisso, con modifica della radice;
  \item[\spzradesufft]  il risultato  dell'unione tra  una radice  e un
    suffisso, con modifica di entrambi;
  \item[\spzradesuff]  il risultato  dell'unione tra  una radice  e un
    suffisso, senza modifiche;
  \item[\spzarmprogr] armonia progressiva;
  \item[\spzarmregr] armonia regressiva;
  \item[({\it x})]  in un suffisso, indica  che {\it x}  è una lettera
    eufonica o una {\it n} pronominale;
  \item[\fonema{{\it   x}}]   indica   che   {\it  x}   è   trascritto
    fonologicamente   (per  fonemi);   
  \item[{\it  x-}]  indica  che  {\it x}  è  una  radice verbale;
  \item[\formasupposta{{\it  x}}]  indica  che  {\it x}  è  una  forma
    supposta, ma  non attestata; 
  \item[\signsupposto{{\it x}}] indica che il significato di {\it x} è
    supposto,  ma  la   forma  di  {\it  x}  è   attestata  con  altri
    significati.
\end{itemize*}

%% \item We subdivide  the parts  of speech in  three categories:
%% \begin{itemize}
%% \item  {\it verbs};
%% \item {\it nouns}, including adjectives  and pronouns; 
%% \item   {\it   particles},   i.e.,  everithing   else   (conjunctions,
%%   pre/postpositions, etc.).
%% \end{itemize}

%% \item      The     examples      \ref{leq:cafer},     \ref{leq:idiki},
%%   \ref{leq:kavafil}, \ref{leq:sadiq} and  \ref{leq:penbe} came from an
%%   ottoman    document    of    the    State    Archive    of    Venice
%%   \spzcitedoc{documento1099}.


\end{enumerate}
